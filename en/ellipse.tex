% !TeX root = orbits.tex

%%%%%%%%%%%%%%%%%%%%%%%%%%%%%%%%%%%%%%%%%%%%%%%%%%%%%%%%%%%%%%%%

\section{Ellipses}\label{s.ellipse}

\subsection{Fundamental properties}

\begin{definition}[Ellipse]
\mbox{}
\begin{itemize}
\item Let $S$ and $H$ be two points in the plane such that $SH=2c\geq 0$ and choose $a$ such that $2a> 2c$ (Figure~\ref{f.ellipse-features1}). An \emph{ellipse} is the geometric locus of all points $P$ such that $SP+HP=2a$. If $c=0$ the geometric locus is a \emph{circle}.
\item Construct $AB$ through $SH$, where $A,B$ are the intersections of the line with the ellipse. $AB$ is  the \emph{major axis} of the ellipse. Let $O$ be the midpoint of $SH$. $AO$ and $OB$ are the \emph{semi-major axes} of the ellipse.
\item Construct the perpendicular to $AB$ at $O$ and let $C,D$ be its intersections with the ellipse. $CD$ is the \emph{minor axis} of the ellipse and $CO$ and $OD$ are the \emph{semi-minor axes} of the ellipse.
\end{itemize}
\end{definition}

\vspace{-3ex}

%%%%%%%%%%%%%%%%%%%%%%%%%%%%%%%%%%%%%%%%%%%%

\begin{figure}[b]
\begin{center}
\begin{tikzpicture}[scale=.7]

% Size and center of the ellipse
\def\a{4.5}
\def\b{2.7}

% Draw an ellipse with center O
\coordinate (O) at (0,0);
%\vertexsm{O};
\node[below left] at (O) {$O$};
\draw[name path=ellipse] (0,0) ellipse[x radius=\a,y radius=\b];

% Draw the two focal points
\coordinate (F1) at ({-sqrt(\a*\a-\b*\b)},0);
\coordinate (F2) at ({+sqrt(\a*\a-\b*\b)},0);
\node[below] at (F1) {$S$};
\node[below] at (F2) {$H$};
%\vertexsm{F1};
%\vertexsm{F2};

% Draw axes whose center is O
\coordinate (L) at +(180:{\a} and {\b});
\coordinate (R) at +(0:{\a} and {\b});
\coordinate (B) at +(-90:{\a} and {\b});
\coordinate (T) at +(90:{\a} and {\b});
\node[left]  at (L) {$A$};
\node[right] at (R) {$B$};
\node[above] at (T) {$C$};
\node[below] at (B) {$D$};
\draw[name path=major] (L) -- (R);
\draw[name path=minor] (B) -- (O) -- (T);
\draw (O) rectangle +(9pt,9pt);

% Label distances from center to foci by c
\path (F1) -- node[below] {$c$} (O) -- node[below] {$c$} (F2);

% Select an arbitrary point on the ellipse and draw lines to the foci
\path[name path=fromF1] (F1) -- +(50:5);
\path [name intersections = {of = ellipse and fromF1, by = {P} }];
\draw (F1) -- (P) -- (F2);

% Label P
\node[above] at (P) {$P$};
%\vertexsm{P};

\end{tikzpicture}
\caption{The definition of an ellipse}\label{f.ellipse-features1}
\end{center}
\end{figure}

%%%%%%%%%%%%%%%%%%%%%%%%%%%%%%%%%%%%%%%%%%%%

\begin{theorem}(Figure~\ref{f.ellipse-features2})\label{thm.ellipses-features}
\begin{enumerate}
\item $SC=HC=a$.
\item $AO=OB=a$.
\item $CO=OD$. (Label $CO=OD$ by $b$.)
\end{enumerate}
\end{theorem}

\begin{proof}
\begin{enumerate}
\item $\triangle SCO\cong\triangle HCO$ by side-angle-side  so $SC=HC$. Since $C$ is on the ellipse, $SC+HC=2a$ and $SC=HC=a$ follows. 
\item Since $A$ is on the ellipse,
\[
2a = AS+AH = (AO-c)+(AO+c)=2AO\,,
\]
so $AO=a$. $OB=a=AO$ follows by symmetry.
\item $CO=OD$ follows since $\triangle SCO\cong\triangle SDO$.\hqed
\end{enumerate}
\end{proof}

%%%%%%%%%%%%%%%%%%%%%%%%%%%%%%%%%%%%%%%%%%%%

\begin{figure}[tb]
\begin{center}
\begin{tikzpicture}[scale=.7]

% Size and center of the ellipse
\def\a{4.5}
\def\b{2.7}

% Draw an ellipse with center O
\coordinate (O) at (0,0);
%\vertexsm{O};
\node[below left] at (O) {$O$};
\draw[name path=ellipse] (0,0) ellipse[x radius=\a,y radius=\b];

% Locate the two focal points
\coordinate (F1) at ({-sqrt(\a*\a-\b*\b)},0);
\coordinate (F2) at ({+sqrt(\a*\a-\b*\b)},0);
\node[below] at (F1) {$S$};
\node[below] at (F2) {$H$};
%\vertexsm{F1};
%\vertexsm{F2};

% Draw axes whose center is O
\coordinate (L) at +(180:{\a} and {\b});
\coordinate (R) at +(0:{\a} and {\b});
\coordinate (B) at +(-90:{\a} and {\b});
\coordinate (T) at +(90:{\a} and {\b});
\node[left]  at (L) {$A$};
\node[right] at (R) {$B$};
\node[above] at (T) {$C$};
\node[below] at (B) {$D$};
\draw[name path=major] (L) -- (R);
\draw[name path=minor] (B) --  
  node[left,yshift=-8pt] {$b$} (O) -- node[left] {$b$} (T);
\draw (O) rectangle +(9pt,9pt);

% Indicates distances along the major axis
\draw[<->] ($(O)+(0,-27pt)$) -- node[fill=white] {$a$} ($(R)+(0,-27pt)$);
\draw[<->] ($(O)+(0,-27pt)$) -- node[fill=white] {$a$} ($(L)+(0,-27pt)$);

% Label distances from center to foci by c
\path (F1) -- node[below] {$c$} (O) -- node[below] {$c$} (F2);

% Draw and label lines from foci to intersection of semi-minor
%   axis and the ellipse
\draw (F1) -- node[above] {$a$} (T) -- node[above] {$a$} (F2);

\end{tikzpicture}
\caption{The semi-major and semi-minor axes of an ellipse}
\label{f.ellipse-features2}
\end{center}
\end{figure}

%%%%%%%%%%%%%%%%%%%%%%%%%%%%%%%%%%%%%%%%%%%%

\begin{theorem}\label{thm.ellipse-equation}
A point $P=(x,y)$ on an ellipse satisfies the equation
\[
\frac{x^2}{a^2}+\frac{y^2}{b^2}=1\,.
\]
\end{theorem}

\begin{proof}
Since $S=(-c,0)$, $H=(c,0)$ and $PS+PH=2a$,
\[
PS+PH=\sqrt{(x-(-c))^2 + y^2}+\sqrt{(x-c)^2+y^2} = 2a\,.
\]
Squaring twice results in
\begin{eqn}
(x+c)^2+y^2 &=& \left(2a - \sqrt{(x-c)^2+y^2}\right)^2\\[4pt]
4xc &=& 4a^2 -4a\sqrt{(x-c)^2+y^2}\\[4pt]
a-\frac{c}{a}x &=& \sqrt{(x-c)^2+y^2}\\[4pt]
a^2 +\frac{c^2}{a^2}x^2 &=& x^2+c^2+y^2\\[4pt]
	\frac{x^2}{a^2}+\frac{y^2}{c^2-a^2} &=& \frac{c^2-a^2}{c^2-a^2}=1\,.
\end{eqn}

By Theorem~\ref{thm.ellipses-features} and Pythagoras's theorem, $b^2=a^2-c^2$ so
\[
\frac{x^2}{a^2}+\frac{y^2}{b^2}=1\,.\fqed
\]
\end{proof}

\subsection{A circle circumscribing an ellipse}

Consider a circle of radius $a$ with the same center as an ellipse (Figure~\ref{f.ellipse-circle}). Choose a point $X$ on the major axis and construct a perpendicular through $X=(x,0)$. Let its intersections with the ellipse and the circle be $P_e=(x,y_e)$ and $P_c=(x,y_c)$, respectively.
\begin{theorem}\label{thm.ellipse-b-over-a}
The perpendicular to the major axis through a point $P_c=(x,y_c)$ on the circle circumscribing an ellipse intersects the ellipse at $P_e=(x,y_e)=\left(x,\displaystyle\frac{b}{a}y_c\right)$.
\end{theorem}
\begin{proof} Using the formulas for points on an ellipse and on a circle we have
\begin{eqnlabels}
\frac{x^2}{a^2}+\frac{y_e^2}{b^2}&=&1\nonumber\\[4pt]
y_e&=& \frac{b}{a} \,\sqrt{(a^2-x^2)}=\frac{b}{a}y_c\,.\label{eq.ye}
\label{eq.point-on-ellipse}\fqed
\end{eqnlabels}
\end{proof}

\begin{figure}[tb]
\begin{center}
\begin{tikzpicture}[scale=.8]

% Draw a hemi-ellipse and a circumscribing hemi-circle
\coordinate (O) at (0,0);
%\vertexsm{O};
\node[below] at (O) {$O$};
\draw[name path=ellipse] (4.5,0) arc(0:180:4.5cm and 2cm);
\draw[name path=circle] (4.5,0) arc(0:180:4.5cm);

% Draw axes through O
\coordinate (L) at (-4.5,0);
\coordinate (R) at (4.5,0);
\coordinate (T) at (0,4.5);
\draw[name path=major] (L) -- 
  node[below] {$a$} (O) -- node[below] {$a$}(R);
\draw[name path=minor] (O) -- (T);
\draw[<->] ($(O)+(-10pt,1pt)$) --
  node[near end,fill=white] {$a$} ($(T)+(-10pt,-1pt)$);

% Choose arbitrary point on major-axis, raise a perpendicular
%   and label its intersections with the ellipse and the circle
\coordinate (X) at (1,0);
\path[name path=atX] (X) -- ($(X)+(0,4.5)$);
\path [name intersections = {of = atX and ellipse, by = {E1} }];
\path [name intersections = {of = atX and circle, by = {C1} }];
\draw (X) -- (C1);

% Label X and the intersections
\node[below] at (X) {$X$};
\node[above right] at (C1) {$P_c\!=\!(x,y_c)$};
\node[above right] at (E1) {$P_e\!=\!(x,y_e)$};

\draw (O) rectangle +(8pt,8pt);
\draw (X) rectangle +(8pt,8pt);

\end{tikzpicture}
\caption{A circle circumscribing an ellipse}\label{f.ellipse-circle}
\end{center}
\end{figure}

%%%%%%%%%%%%%%%%%%%%%%%%%%%%%%%%%%%%%%%%%%%%%%%%%%%%%%%

\subsection{The latus rectum of an ellipse}

\begin{definition}\label{def.ellipse-lr}
Consider a line through a focus of an ellipse that is perpendicular the major axis. Let its intersections with the ellipse be $L_1,L_2$. $L=L_1L_2$ is a \emph{latus rectum}\footnote{Normally, points are denoted by upper-case letters and line segments or lengths by lower-case letters, but $L$ for the latus rectum is the standard notation.} of an ellipse (Figure~\ref{f.ellipse-latus-rectum}).
\end{definition}
\begin{theorem}\label{thm.ellipse-lr}
$L$, the length of the latus rectum of an ellipse, is 
$\displaystyle\frac{2b^2}{a}$.
\end{theorem}
\begin{proof}
By Equation~\ref{eq.ye} and Pythagoras's theorem,
\[
L=2L_1=2\cdot\frac{b}{a}\sqrt{a^2-c^2}=\displaystyle\frac{2b^2}{a}\,.\fqed
\]
\end{proof}

\begin{figure}[tb]
\begin{center}
\begin{tikzpicture}[scale=.9]

% Size and center of the ellipse
\def\a{4}
\def\b{2}

% Draw an ellipse with center O
\coordinate (O) at (0,0);
%\vertexsm{O};
\node[below left] at (O) {$O$};
\draw[name path=ellipse] (0,0) ellipse[x radius=\a,y radius=\b];
\draw[name path=circle] (4,0) arc(0:180:4cm);

\coordinate (O) at (0,0);

% The Sun is at a focal point
\coordinate (F1) at ({-sqrt(\a*\a-\b*\b)},0);
\coordinate (F2) at ({+sqrt(\a*\a-\b*\b)},0);
\node[below] at (F1) {$S$};
\node[below left] at (F2) {$H$};
%\vertexsm{F1};
%\vertexsm{F2};

% Draw axes whose center is O
\coordinate (L) at +(180:{\a} and {\b});
\coordinate (R) at +(0:{\a} and {\b});
\coordinate (B) at +(-90:{\a} and {\b});
\coordinate (T) at +(90:{\a} and {\b});
\node[above] at (T) {$C$};
\node[below] at (B) {$D$};
\draw[name path=major] (L) -- (R);
\draw[name path=minor] (B) -- (O) -- node[left] {$b$} (T);
\draw (O) rectangle +(7pt,7pt);

\path (F1) -- node[below] {$c$} (O) -- node[below] {$c$} (F2);

% Draw latus rectum as intersections of perpendicular at H
%  with the ellipse
\path[name path=LR] ($(F2)+(0,-2.5)$) -- ($(F2)+(0,2.5)$);
\path [name intersections = {of = ellipse and LR, by = {LR1,LR2} }];
\path [name intersections = {of = circle and LR, by = {LR1c,LR2c} }];
\draw[very thick] (LR1) -- (LR2);

% Label endpoint of the LR
%\vertexsm{LR1};
\node[above left,xshift=2pt,yshift=6pt] at (LR1) {$L_1$};
%\vertexsm{LR2};
\node[below,yshift=-2pt] at (LR2) {$L_2$};
\draw (LR1) -- (LR1c);
\draw (F2) rectangle +(7pt,7pt);

% Length of LR
\draw[<->] ($(LR1)+(1,0)$) -- node[right] {$L$} ($(LR2)+(1,0)$);

\end{tikzpicture}
\caption{The latus rectum of an ellipse}\label{f.ellipse-latus-rectum}
\end{center}
\end{figure}

%%%%%%%%%%%%%%%%%%%%%%%%%%%%%%%%%%%%%%%%%%%%%%%%%

\newpage

\subsection{The area of an ellipse}

\begin{theorem}\label{thm.ellipse-area}
The area of an ellipse is $\pi a b$.
\end{theorem}
\begin{proof}
From Equation~\ref{eq.ye}
\[
y_e = \frac{b}{a}\sqrt{a^2-x^2}\,,
%y_c &=& \sqrt{a^2-x^2}\\
\]
so the area of an ellipse is
\[
A_e = 2\int_{-a}^{a}\frac{b}{a}\sqrt{a^2-x^2}\; dx = 2\frac{b}{a}\int_{-a}^{a}\sqrt{a^2-x^2}\; dx= \frac{b}{a}A_c\,.
\]
If we can show that the area of a circle is $\pi a^2$ the theorem follows.

The proof uses polar coordinates, where $x=a\cos \theta$ and $y=a\sin \theta$. First, we derive the formula for the integral of $\sin^2 \theta$ using the double-angle identity.
\begin{eqn}
%\cos 2\theta &=& 1-2\sin^2 \theta \\[4pt]
\sin^2 \theta &=& \frac{1-\cos 2\theta}{2}\\[6pt]
\int \sin^2 \theta \;d\theta &=& \int\frac{1-\cos 2\theta}{2} \;d\theta
=\frac{\theta}{2} - \frac{\sin 2\theta}{4}  + C\,.
\end{eqn}
\noindent{}Now we can compute the area of a circle as twice the area of a semicircle by changing from Cartesian to polar coordinates and integrating.
\begin{eqn}
A_c &=& 2\int_{-a}^{a} \sqrt{a^2-x^2} \;dx= 2\int_{-\pi}^{0} \sqrt{a^2-(a \cos \theta)^2} \;\;d(a\cos\theta)\\[4pt]
%&=& 2a^2 \int_{-\pi}^{0} \sqrt{1-\cos^2 \theta} \;\;d(\cos\theta)\\[4pt]
&=& 2\cdot a\cdot a \int_{-\pi}^{0} \sin \theta (-\sin\theta)\;d\theta
= -2a^2 \int_{-\pi}^{0} \sin^2 \theta \;d\theta\\[4pt]
&=& -2a^2 \left.\left( \frac{\theta}{2} - \frac{\sin 2\theta}{4}+C\right)\right|_{-\pi}^{0}=\pi a^2\,.\fqed
\end{eqn}
\end{proof}

%%%%%%%%%%%%%%%%%%%%%%%%%%%%%%%%%%%%%%%%%%%%%%%%%%%%%%%%%%%%%%%%%%%%%%

\subsection{The angles between a tangent and the lines to the foci}

\begin{theorem}\label{thm.tangent-angles}
Let $P$ be a point on the ellipse whose foci are $S,H$. Let $PU$ be the extension of $SP$ such that $SU=AA'=2a$ (Figure~\ref{f.tangent-angles}). Let $RQ$ be the bisector of $\angle HPU$. Then $RQ$ is the tangent to the ellipse at $P$ and $\angle RPS = \angle HPQ$.
\end{theorem}

%%%%%%%%%%%%%%%%%%%%%%%%%%%%%%%%%%%%%%%%%%%%%%%%%%%%%%%%%%%%%%%%%%%%%%

\begin{figure}[t]
\begin{center}
\begin{tikzpicture}[scale=1.1]

% Size and center of the ellipse
\def\a{4.33}
\def\b{3.5}
\coordinate (C) at (0,0);

% Draw an ellipse with center C
\draw[name path=ellipse] (\a,0) 
  arc[start angle=0,end angle=180, x radius=\a,y radius=\b];

% Locate the focal points
\coordinate (F1) at ({-sqrt(\a*\a-\b*\b)},0);
\coordinate (F2) at ({+sqrt(\a*\a-\b*\b)},0);
\node[below] at (F1) {$S$};
\node[below] at (F2) {$H$};

% Draw axes whose center is C
\coordinate (L) at +(180:{\a} and {\b});
\coordinate (R) at +(0:{\a} and {\b});
\coordinate (Top) at +(90:{\a} and {\b});
\draw[name path=major] (L) node[below] {$A'$} -- (R) node[below] {$A$};
\draw[name path=minor] (C) -- (Top);

% Select an arbitrary point P on the ellipse and
%   draw lines to focal points
% Extend line SP to U
\draw[name path=fromF1p] (F1) -- +(15:{2*\a}) coordinate (U);
\node[right] at (U) {$U$};
\path [name intersections = {of = ellipse and fromF1p, by = {P} }];
\draw[name path=ph] (P) -- (F2);
\node[above right,xshift=-4pt,yshift=2pt] at (P) {$P$};
%\vertexsm{P};

% Locate tangent at P
\tkzDefLine[bisector out](F1,P,F2) \tkzGetPoint{Tan}

% Locate another point Q on the tangent
\coordinate (Q) at ($(Tan)!.75!(P)$);
\node[right] at (Q) {$Q$};
\draw (Q) -- (F1);
%\vertexsm{Q};

% Draw the tangent from Q to R
\draw[name path=tan] ($(Tan)!.60!(P)$) coordinate (Z) --
  ($(Tan)!1.3!(P)$) coordinate (R);
\node[right] at (R) {$R$};

% Draw HQ and QU
\draw (F2) -- 
  node[below right,xshift=6pt,yshift=6pt] {$d$} (Q) -- 
  node[right] {$d$} (U);
\draw[thick,dashed,name path=F2U] (F2) -- (U);

% Draw dashed line to local W
\path [name intersections = {of = tan and F2U, by = {W} }];
\node[right,xshift=1pt,yshift=-2pt] at (W) {$W$};

% Label angles
\node[below,yshift=-4pt] at (P) {$\alpha$};
\node[right,xshift=4pt,yshift=-5pt] at (P) {$\alpha$};
\node[left,xshift=-4pt,yshift=6pt] at (P) {$\alpha$};

% Indicate lengths of SU
\draw [<->] ($(F1)+(0,3pt)$) -- node[above] {$2a$} ($(U)+(0,3pt)$);

\end{tikzpicture}
\caption{Angles at the tangent}\label{f.tangent-angles}
\end{center}
\end{figure}

%%%%%%%%%%%%%%%%%%%%%%%%%%%%%%%%%%%%%%%%%%%%%%%%%%%%%%%%%%%%%%%%

\begin{proof} 
We prove that any point $Q\neq P$ on the bisector is not on the ellipse; therefore, the bisector $RQ$ has only one point of intersection with the ellipse and must be a tangent. Since $RQ$ is the angle bisector of $\angle HPU$, $\angle HPQ =\angle UPQ=\alpha$, and by vertical angles $\angle RPS=\angle QPU=\alpha$.

Construct the line $HU$ to form the triangle $\triangle HPU$. By construction $SU=2a$ so $PU=PH$ and by the angle bisector theorem,
\[
\frac{UW}{HW}=\frac{PU}{PH} = 1\,.
\] 
$\triangle UWQ\cong\triangle HWQ$ by side-angle-side so $UQ=HQ$. Suppose that $Q$ is on the ellipse. Then $2a=SQ+HQ=SQ+QU$. By the triangle inequality $2a=SQ+QU>SU=2a$, contradicting that $Q$ is on the ellipse.\hqed
\end{proof}

%%%%%%%%%%%%%%%%%%%%%%%%%%%%%%%%%%%%%%%%%%%%%%%

\begin{figure}[t]
\begin{center}
\begin{tikzpicture}

% Size and center of the ellipse
\def\a{3}
\def\b{1.5}
\coordinate (C) at (0,0);
\node[below left,xshift=2pt] at (C) {$C$};

% Draw an ellipse and the inner and outer circles
\draw[name path=ellipse] (C) ellipse[x radius=\a,y radius= \b];
\draw[very thick,dotted,red,name path=outer] (C) circle[radius=\a];
\draw[thick,dashed,blue,name path=inner] (C) circle[radius=\b];

% Locate P on the ellipse
\path[name path=toP] (C) -- +(30:4);
\path [name intersections = {of = ellipse and toP, by = {P} }];

% Label the angle of CP
\node[above right,xshift=6pt,yshift=-2pt] at (C) {$t$};

% Draw major and minor axes
\draw[name path=yplus] (C) -- +(0,\a);
\draw (C) -- +(0,-\a);
\draw (C) -- +(\a,0);
\draw (C) -- +(-\a,0);

% Find intersection PI of line from center to P and the inner circle
\node[above right] at (P) {$P$};
\path[name path=horiz] (P) -- +(-\a,0);
\path [name intersections = {of = inner and horiz, by = {PI} }];

% Extend ray C->PI to outer circle and label intersection PO
\draw[name path=ray] (C) -- ($(C) !2! (PI)$) coordinate (PO);

% Project PO on y axis
\path[name path=horizPO] (PO) -- +(-\a,0);
\path [name intersections = {of = yplus and horizPO, by = {yint} }];
\draw[<->,style={shorten <= 2pt}] (PO) --
  node[above,xshift=-5pt] {$a\cos t$} (yint);

% Project P on y axis
\path [name intersections = {of = yplus and horiz, by = {xint} }];
\draw[<->] ($(C)+(-5pt,0)$) -- 
  node[left,xshift=1pt] {$b\sin t$} ($(xint)+(-5pt,0)$);
\draw (P) -- (xint);

% Draw blue and red lines from C to PI and PO
\draw[thick,blue] (C) -- (PI);
\draw[thick,red] (PI) -- (PO);

% Dots at vertices
\vertexsmcolor{PI}{blue};
\vertexsmcolor{PO}{red};
%\vertexsm{C};
%\vertexsm{P};

\end{tikzpicture}
\caption{Parametric representation of an ellipse}\label{f.parametric}
\end{center}
\end{figure}

%%%%%%%%%%%%%%%%%%%%%%%%%%%%%%%%%%%%%%%%%%%%%%%%%%%%%%

\subsection{The parametric representation of an ellipse}

Figure~\ref{f.parametric} shows an ellipse and two circles: one whose radius is the length of the semi-major axis (dotted red) and one whose radius is the semi-minor axis (dashed blue). The figure shows the \emph{parametric representation} of a point $P=(x,y)$ on the ellipse:
\[
(x,y)= (a\cos t,\, y = b \sin t)\,.
\]
The parameter $t$ is \emph{not} the angle of $P$ relative to the positive $x$-axis. It is constructed by projecting $P$ onto the minor axis. Then the line at angle $t$ is constructed from $C$ to the intersection of the projection with the inner circle (dashed). The line is extended to intersect the outer circle (dotted). The parametric representation of $P$ is computed by multiplying the lengths of the axes by the appropriate trigonometry function.

%%%%%%%%%%%%%%%%%%%%%%%%%%%%%%%%%%%%%%%%%%%%%%%%%%%%%%%%%%%%%

\begin{figure}[b]
\begin{center}
\begin{tikzpicture}[scale=.7]

% Size and center of the ellipse
\def\a{4.33}
\def\b{3.5}

% A bit of padding to center
\path ({-\a},0) -- +(-30pt,0);

% Draw an ellipse with center C
\coordinate (C) at (0,0);
\node[below,xshift=-1pt,yshift=-1pt] at (C) {$C$};
\draw[name path=ellipse1] (C) ellipse[x radius=\a,y radius= \b];
\draw[name path=ellipse,thick,dashed] (C) ellipse[rotate=15,x radius=\a,y radius= \b];

% The Sun is at a focal point
\coordinate (F1) at ({-sqrt(\a*\a-\b*\b)},0);
\coordinate (F2) at ({+sqrt(\a*\a-\b*\b)},0);

% Draw axes whose center is C
\coordinate (L) at +(180:{\a} and {\b});
\coordinate (R) at +(0:{\a} and {\b});
\coordinate (Top) at +(90:{\a} and {\b});
\draw[name path=major] (L) -- (R);
\draw[name path=minor] (C) -- (Top);

% Select an arbitrary point P on the ellipse and draw a line to the Sun
\path[name path=fromF1p] (F1) -- +(20:8);
\path [name intersections = {of = ellipse and fromF1p, by = {P} }];
\path[name path=ph] (P) -- (F2);
\path[name path=pc] (P) -- (C);
\node[right,yshift=3pt] at (P) {$P$};

% Select an arbitrary point Q on the ellipse and draw a line to the Sun
\path[name path=fromF1q] (F1) -- +(40:7);
\path [name intersections = {of = ellipse and fromF1q, by = {Q} }];
\node[above] at (Q) {$Q$};

% Draw tangent at P
\path (F1) -- ($(F1)!1.1!(P)$);
\tkzDefLine[bisector out](F1,P,F2) \tkzGetPoint{Tan}
\path[name path=t] ($(Tan)!.65!(P)$) 
  coordinate (Z) -- ($(Tan)!1.5!(P)$);

% Draw QR
\path[name path=qr,red] (Q) -- +(10:2);
\path [name intersections = {of = t and qr, by = {R} }];

% Draw QX
\path[name path=qx] (Q) -- +($(Z)-(R)$);
\path [name intersections = {of = qx and fromF1p, by = {X} }];
\path [name intersections = {of = qx and pc, by = {V} }];
\draw (X) -- (V) node[below right,xshift=-2pt,yshift=-4pt] {$V$};

% Conjugate diameter
\path[name path=cd] (C) -- +($(R)-(Z)$);
\path[name path=dk] (C) -- +($(Z)-(R)$);
\path [name intersections = {of = cd and ellipse, by = {D} }];
\path [name intersections = {of = dk and ellipse, by = {K} }];
\node[above left,yshift=2pt] at (D) {$D$};
\node[below right,yshift=0pt] at (K) {$K$};
\draw (D) -- (K);

% Conjugate diameter
\draw[name path=pc] (P) -- ($(P)!2!(C)$) coordinate (G) node[left,xshift=-2pt,yshift=-2pt] {$G$};

\draw[red,thick] (G) -- (V);
\draw[blue,thick] (V) -- (P);
\draw[green!80!black,thick] (Q) -- (V);

%\vertexsm{V};
\vertexsm{G};
%\vertexsm{C};
\vertexsm{P};
\vertexsm{D};
\vertexsm{Q};
\vertexsm{K};

\end{tikzpicture}
\caption{Ratios on conjugate diameters}\label{f.conj-ratios}
\end{center}
\end{figure}

%%%%%%%%%%%%%%%%%%%%%%%%%%%%%%%%%%%%%%%%%%%%%%%%%%%%%%%%%%%%%%%%%%%%%%

\subsection{Ratios}

\begin{theorem}\label{thm.ratios}
Let $P=(x,y)$ be a point on an ellipse (not on the major axis $AA'$) and construct a perpendicular $PV$ from $P$ to the major axis (Figure~\ref{f.ellipse-ratios}). Then
\[
\frac{A'V\cdot AV}{PV^2} = \frac{a^2}{b^2}\,.
\]
\end{theorem}
\begin{proof}
By Equation~\ref{eq.point-on-ellipse}),
\begin{eqn}
y^2&=&b^2\cdot \left(1-\frac{x^2}{a^2}\right)=\frac{b^2(a^2-x^2)}{a^2}\\
\frac{A'V\cdot AV}{PV^2}&=&\frac{(a+x)(a-x)}{y^2}=\frac{a^2(a^2-x^2)}{b^2(a^2-x^2)}= \frac{a^2}{b^2}\,.\fqed
\end{eqn}
\end{proof}

%%%%%%%%%%%%%%%%%%%%%%%%%%%%%%%%%%%%%%%%%%%%%%%%%%%%%%%%%%%%%%%%%%%%%%

\begin{theorem}\label{thm.conj-diag}
Let $PG,DK$ be conjugate diameters of an ellipse and let $Q$ be a point on the ellipse (Figure~\ref{f.conj-ratios}). Drop a perpendicular $QV$ from $Q$ to the major axis, then
\[
PV = \frac{QV^2\cdot CP^2}{GV \cdot CD^2}\,.
\]
\end{theorem}

\begin{proof}
Figure~\ref{f.conj-ratios} shows a dashed ellipse which is the original ellipse rotated about the same center $C$, so that $CP$ is the semi-major axis and $CD$ is the semi-minor axis. By Theorem~\ref{thm.ratios}
\[
\frac{GV\cdot PV}{QV^2}=\frac{a'^2}{b'^2}\,,
\]
where $a',b'$ are the lengths of the semi-major and semi-minor axes of the rotated ellipse. By construction $a'=CP$ and $b'=CD$ so
\begin{eqn}
\frac{GV\cdot PV}{QV^2} &=& \frac{CP^2}{CD^2}\\
PV &=& \frac{QV^2\cdot CP^2}{GV \cdot CD^2}\,.\fqed
\end{eqn}
\end{proof}

\vspace*{-5ex}

%%%%%%%%%%%%%%%%%%%%%%%%%%%%%%%%%%%%%%%%%%%%%%%%%%%%%%%%%%%55

\subsection{Areas of parallelograms}

There are two equivalent definitions of conjugate diameters.

\begin{definition}\label{def.conjugate}\mbox{}
Let $P$ be a point on an ellipse, $PG$ a diameter and let $t$ be the tangent to the ellipse at $P$. Diameter $DK$ is a \emph{conjugate diameter} if it is parallel to $t$ (Figure~\ref{f.ellipse-conj}). Two diameters $PG$ and $DK$ are \emph{conjugate diameters} if the midpoints of chords ($D'K', D''K''$) parallel to one diameter ($DK$) lie on another diameter ($PG$).
\end{definition}

\vspace*{-3ex}

\begin{theorem}\label{thm.conj-diam-para}
The areas of parallelograms formed by tangents to the intersections of conjugate diameters with the ellipse are equal (Figure~\ref{f.ellipse-conj-diam-proof}).
\end{theorem}

%%%%%%%%%%%%%%%%%%%%%%%%%%%%%%%%%%%%%%%%%%%%%%%%%%%%%%%%%%%%%%%%%%%%%%

\begin{figure}[t]
\begin{center}
\begin{tikzpicture}[scale=.6]

% Size and center of the ellipse
\def\a{4.33}
\def\b{3.5}
\coordinate (C) at (0,0);
\node[below right,xshift=3pt,yshift=2pt] at (C) {$C$};

% Draw the ellipse
\draw[name path=ellipse] (C) ellipse[x radius=\a,y radius= \b];

% Locate the focal points
\coordinate (F1) at ({-sqrt(\a*\a-\b*\b)},0);
\coordinate (F2) at ({+sqrt(\a*\a-\b*\b)},0);

% Draw axes AA' and BB'
\coordinate (L) at +(180:{\a} and {\b});
\coordinate (R) at +(0:{\a} and {\b});
\coordinate (Top) at +(90:{\a} and {\b});
\coordinate (Bot) at +(-90:{\a} and {\b});
\draw[name path=major] (L) node[left] {$A'$} --
  (R) node[right,xshift=-2pt] {$A$};
\draw[name path=minor] (Bot) node[below,yshift=2pt] {$B'$} --
  (Top) node[above,yshift=-1pt] {$B$};

% Select an arbitrary point P on the ellipse
\path[name path=fromF1p] (F1) -- +(15:8);
\path [name intersections = {of = ellipse and fromF1p, by = {P} }];
\path[name path=ph] (P) -- (F2);
\path[name path=pc] (P) -- (C);
\node[right,xshift=-2pt,yshift=3pt] at (P) {$P$};

% Draw tangent at P
\tkzDefLine[bisector out](F1,P,F2) \tkzGetPoint{Tan1}
\draw[name path=t1] ($(Tan1)!.25!(P)$) -- 
  node[right,near end] {$t$} ($(Tan1)!1.75!(P)$);

% Conjugate diameter from P through C
\draw[name path=pc] (P) -- ($(P)!2!(C)$) coordinate (G);
\node[left,yshift=-2pt] at (G) {$G$};

% Find a point Z on tangent for drawing the tangent
\path[name path=f1q] (F1) -- +(55:8);
\path [name intersections = {of = f1q and t1, by = {Z} }];

% Conjugate diameters from D through C
\path[name path=cd] (C) -- +($(Z)-(P)$);
\path[name path=dk] (C) -- +($(P)-(Z)$);
\path [name intersections = {of = cd and ellipse, by = {D} }];
\path [name intersections = {of = dk and ellipse, by = {K} }];
\node[above left,xshift=3pt,yshift=-2pt] at (D) {$D$};
\node[below right,xshift=-1pt,yshift=2pt] at (K) {$K$};
\draw (D) -- (K);

% Draw parallel to the tangent
\path (C) -- +(24:-1.6) coordinate(PP);
%\vertexsm{PP};
\path[name path=pp1] (PP) -- +($(Z)-(P)$);
\path[name path=pp2] (PP) -- +($(P)-(Z)$);
\path [name intersections = {of = pp1 and ellipse, by = {DP} }];
\path [name intersections = {of = pp2 and ellipse, by = {KP} }];
\draw[thick,dashed] (DP) node[above,xshift=-4pt,yshift=-1pt] {$D'$} -- 
  (KP) node[below,xshift=5pt,yshift=2pt] {$K'$};

\path (C) -- +(24:2.9) coordinate(PPP);
%\vertexsm{PPP};
\path[name path=ppp1] (PPP) -- +($(Z)-(P)$);
\path[name path=ppp2] (PPP) -- +($(P)-(Z)$);
\path [name intersections = {of = ppp1 and ellipse, by = {DPP} }];
\path [name intersections = {of = ppp2 and ellipse, by = {KPP} }];
\draw[thick,dashed] (DPP) node[above,xshift=0pt,yshift=-1pt] {$D''$} -- 
  (KPP) node[below,xshift=6pt,yshift=2pt] {$K''$};

\end{tikzpicture}
\caption{A conjugate diameter}\label{f.ellipse-conj}
\end{center}
\end{figure}


%%%%%%%%%%%%%%%%%%%%%%%%%%%%%%%%%%%%%%%%%%%%%%%%%%%%%%%%%%%%%%%%%%%%%%

\begin{figure}
\begin{center}
\begin{tikzpicture}[scale=.67]

% Size and center of the ellipse
\def\a{4.33}
\def\b{3.5}
\coordinate (C) at (0,0);
\node[below left,xshift=2pt] at (C) {$C$};

% Draw the ellipse
\draw[name path=ellipse] (C) ellipse[x radius=\a,y radius= \b];

% Locate the focal points
\coordinate (F1) at ({-sqrt(\a*\a-\b*\b)},0);
\coordinate (F2) at ({+sqrt(\a*\a-\b*\b)},0);

% Draw axes AA' and BB'
\coordinate (L) at +(180:{\a} and {\b});
\coordinate (R) at +(0:{\a} and {\b});
\coordinate (Top) at +(90:{\a} and {\b});
\coordinate (Bot) at +(-90:{\a} and {\b});
\draw[name path=major] (L) node[left,xshift=3pt,yshift=6pt] {$A'$} --
  (R) node[right,xshift=-2pt] {$A$};
\draw[name path=minor] (Bot) node[below,xshift=-3pt,yshift=2pt] {$B'$} --
  (Top) node[above,xshift=5pt,yshift=-1pt] {$B$};

% Select an arbitrary point P on the ellipse
\path[name path=fromF1p] (F1) -- +(15:8);
\path [name intersections = {of = ellipse and fromF1p, by = {P} }];
\path[name path=ph] (P) -- (F2);
\path[name path=pc] (P) -- (C);
\node[right,xshift=-2pt,yshift=3pt] at (P) {$P$};

% Draw tangent at P
\tkzDefLine[bisector out](F1,P,F2) \tkzGetPoint{Tan1}
\path[name path=t1] ($(Tan1)!.25!(P)$) -- ($(Tan1)!1.75!(P)$);

% Conjugate diameter from P through C
\draw[name path=pc] (P) -- ($(P)!2!(C)$) coordinate (G);

% Find a point Z on tangent for drawing the tangent
\path[name path=f1q] (F1) -- +(55:8);
\path [name intersections = {of = f1q and t1, by = {Z} }];

% Conjugate diameters from D through C
\path[name path=cd] (C) -- +($(Z)-(P)$);
\path[name path=dk] (C) -- +($(P)-(Z)$);
\path [name intersections = {of = cd and ellipse, by = {D} }];
\path [name intersections = {of = dk and ellipse, by = {K} }];
\node[above left,xshift=1pt,yshift=-5pt] at (D) {$D$};
\draw (D) -- (K);

%% Draw tangent at G (continuation of PC)
\tkzDefLine[bisector out](F2,G,F1) \tkzGetPoint{Tan2}
\path[name path=t2] ($(Tan2)!.2!(G)$) -- ($(Tan2)!1.7!(G)$);

%% Draw tangent at D
\tkzDefLine[bisector out](F1,D,F2) \tkzGetPoint{Tan3}
\path[name path=t3] ($(Tan3)!-1!(D)$) -- ($(Tan3)!2.8!(D)$);

% Draw tangent at K (continuation of DC)
\tkzDefLine[bisector out](F1,K,F2) \tkzGetPoint{Tan4}
\path[name path=t4] ($(Tan4)!-.1!(K)$)  -- ($(Tan4)!2!(K)$);

% Draw large dashed rectangle
\draw[thick,dashed] (\a,\b) node[above right] {$C'$} -- (-\a,\b) -- 
  (-\a,-\b) -- (\a,-\b) -- cycle;

% Get the intersections of the tangents
\path [name intersections = {of = t1 and t3, by = {J} }];
\path [name intersections = {of = t2 and t3, by = {KK} }];
\path [name intersections = {of = t1 and t4, by = {M} }];
\path [name intersections = {of = t2 and t4, by = {LL} }];

% Draw the parallelogram
\draw (J) node[above] {$J$} -- (KK) node[left] {$K$} -- 
  (LL) node[below] {$L$} -- (M) node[right] {$M$} -- cycle;

% Draw triangles
\draw[very thick,blue] (C) -- (D) -- (P) -- cycle;
\draw[very thick,red] (C) -- (Top) -- (R) -- cycle;

% Draw perpeciculars to major axis
\draw[very thick,dashed,blue] (D) -- ($(L)!(D)!(R)$) node[below] {$D'$};
\draw[very thick,dashed,blue] (P) -- ($(L)!(P)!(R)$) node[below] {$P'$};

\end{tikzpicture}
\caption{Parallelograms formed by conjugate diameters}\label{f.ellipse-conj-diam-proof}
\end{center}
\end{figure}

\begin{proof}
We show that the area of the parallelogram $JKLM$ is equal to the area of the parallelogram formed by tangents to the major and minor axes (dashed). By symmetry it suffices to prove that the areas of one of the quadrants of those parallelograms are equal: $A_{ACBC'}=A_{PCDJ}$. Since the diagonals bisect a parallelogram, it suffices to prove that that the area of $\triangle ABC$ (red) equals the area of $\triangle PCD$ (blue).

Let $P=(x_p,y_p)=(a\cos t, b\sin t), D=(x_d,y_d)$ be the parametric representations of the points on the ellipse. Conjugate diameters are perpendicular so $\angle DCP$ is a right angle and
\[
D=(x_d,y_d)=(a\cos (t+\pi/2), b\sin (t+\pi/2))=(-a\sin t, b\cos t)\,.
\]
Construct $DD'=(x_d,0)$ and $PP'=(x_p,0)$ perpendicular to the major axis. The area of $\triangle PCD$ can be computed as the area of the trapezoid $P'PDD'$ minus the areas of the triangles $\triangle D'DC, \triangle P'PC$. Therefore,
\begin{eqn}
\triangle PCD &=& \frac{y_p+y_d}{2} (x_p-x_d) - \frac{1}{2}x_dy_d - \frac{1}{2}x_py_p=\frac{1}{2}\left(x_py_d -x_dy_p\right)\\[6pt]
&=&\frac{1}{2} \left( a\cos t \cdot b \cos t - (-a)\sin t\cdot b\sin t\right)=\frac{1}{2}ab=\triangle ABC\,.\fqed
\end{eqn}
\end{proof}

%%%%%%%%%%%%%%%%%%%%%%%%%%%%%%%%%%%%%%%%%%%%%%

\begin{figure}
\begin{center}
\begin{tikzpicture}[scale=.7]

% Size and center of the ellipse
\def\a{4.33}
\def\b{3}
\coordinate (C) at (0,0);
\node[below] at (C) {$C$};

% Draw an ellipse with center C
\draw[name path=ellipse] (\a,0) 
  arc[start angle=0,end angle=180, x radius=\a,y radius=\b];

% Locate the focal points
\coordinate (F1) at ({-sqrt(\a*\a-\b*\b)},0);
\coordinate (F2) at ({+sqrt(\a*\a-\b*\b)},0);

% Draw axes whose center is C
\coordinate (L) at +(180:{\a} and {\b});
\coordinate (R) at +(0:{\a} and {\b});
\node[below right] at (R) {$A$};
\node[below left] at (L) {$A'$};
\coordinate (Top) at +(90:{\a} and {\b});
\draw[name path=major] (L) -- (R);
\draw[name path=minor] (C) -- (Top);

% Select an arbitrary point P on the ellipse and draw lines to the major axes
\path[name path=fromF1p] (F1) -- +(30:6);
\path [name intersections = {of = ellipse and fromF1p, by = {P} }];
\node[above right,yshift=-4pt] at (P) {$P=(x,y)$};
\path[name path=pv] (P) -- +(0,-\b);
\path [name intersections = {of = major and pv, by = {V} }];
\draw (P) -- node[left] {$y$} (V) node[below] {$V$};

% Draw length arrows
\draw[<->] ($(L)+(0,-24pt)$) -- 
  node[fill=white] {$a\!+\!x$} ($(V)+(0,-24pt)$);
\draw[<->] ($(V)+(0,-24pt)$) -- 
  node[fill=white] {$a\!-\!x$} ($(R)+(0,-24pt)$);

% Dots at vertices
%\vertexsm{P};
%\vertexsm{V};

\end{tikzpicture}
\caption{Ratios on conjugate diameters}\label{f.ellipse-ratios}
\end{center}
\end{figure}
