% !TeX root = orbits.tex
% !TeX Program=pdfLaTeX

\chapter{Ellipses in Euclidean geometry}\label{s.geometry}

The proofs of theorems about planetary orbits used analytic geometry and trigonometry, but for many years after the invention of analytic geometry, mathematicians continued to limit themselves to Euclidean geometry. This section contains proofs in Euclidean geometry of theorems that appeared in Chapter~\ref{s.ellipse}.\footnote{Except for Theorems~\ref{thm.tangent-angles}, \ref{thm.conj-diag} which were proved using Euclidean geometry and Theorem~\ref{thm.ellipse-area} which requires taking limits.} The goal is to prove Theorem~\ref{thm.conj-diam-para} in Euclidean geometry.

\begin{center}
\fbox{\parbox{.8\textwidth}{To remain consistency with Besant, in this chapter the center of the ellipse will be denoted $C$ instead of $O$.}}
\end{center}

%%%%%%%%%%%%%%%%%%%%%%%%%%%%%%%%%%%%%%%%%%%%%%%%%%%%%%%%%%%%%%%%%%%%%

\section{A right angle at the focus of an ellipse}

\begin{theorem}\label{thm.bisect}
Let $P,P'$ be points on the ellipse and let $F$ be the intersection of $PP'$ with the directrix. Then $FS$ bisects the exterior angle of $\angle P'SP$(Figure~\ref{f.bisect-angle}).
\end{theorem}

%%%%%%%%%%%%%%%%%%%%%%%%%%%%%%%%%%%%%%%%%%%%%%%%%%%%%%%%%%

\begin{figure}[b]
\begin{center}
\begin{tikzpicture}[scale=1.1]

\clip (-4.1,-3) rectangle +(7.2,5.5);

\def\a{3}
\def\b{2}

\coordinate (O) at (0,0);
% Draw the ellipse
\draw[name path global=ellipse,thick,dashed] (O) 
  ellipse[x radius={\a},y radius= {\b}];

% Locate the foci
\coordinate (F1) at ({-sqrt(\a*\a-\b*\b)},0);
\node[right,yshift=-2pt] at (F1) {$S$};

\def\x{.7}
\def\y{(\b/\a)*sqrt(\a*\a-\x*\x)}
\coordinate (P) at ({\x},{\y});
\node[above] at (P) {$P$};
\def\xp{-2.2}
\def\yp{-(\b/\a)*sqrt(\a*\a-\xp*\xp)}
\coordinate (PP) at ({\xp},{\yp});
\node[right,xshift=2pt,yshift=2pt] at (PP) {$P'$};

% Construct focus and directrix
\coordinate (X) at (-3.5,0);
\node[left] at (X) {$X$};
\path[name path=directrix] ($(X)+(0,4)$) -- ($(X)+(0,-5)$);

% Locate F as extension of PP' to the directrix
\path[name path=PPP] (P) -- ($(P)!1.6!(PP)$);
\path[name intersections = {of = PPP and directrix, by = {F} }];
\node[left] at (F) {$F$};
\draw (PP) -- (F);

% Construct perpendiculars through K, K' to the directrix
\draw (P) -- (P-|X) coordinate (K);
\node[left] at (K) {$K$};
\draw[rotate=-90] (K) rectangle +(6pt,6pt);
\draw (PP) -- (PP-|X) coordinate (KP);
\node[left] at (KP) {$K'$};
\draw[rotate=-90] (KP) rectangle +(6pt,6pt);

% Draw triangles
\draw (F1) -- (X) -- (F) -- (K);
\draw[thick,blue] (F) -- (F1);
\draw[thick,red] (P) -- (F1) -- (PP) -- cycle;
\draw[thick,red,dashed] (P) -- ($(P)!2!(F1)$);

% Label angles
\node[below,xshift=-3pt,yshift=-10pt] at (F1) {\sm{\beta}};
\node[below left,xshift=-6pt,yshift=-8pt] at (F1) {\sm{\beta}};

\end{tikzpicture}
\end{center}
\caption{Bisecting the angle at the focus}\label{f.bisect-angle}
\end{figure}

%%%%%%%%%%%%%%%%%%%%%%%%%%%%%%%%%%%%%%%%%%%%%%%%%%%%%%%%%%

\begin{proof}
Since $P,P'$ are on the ellipse 
\[
\frac{SP}{PK}=\frac{SP'}{P'K'}=e\,,
\]
and since $\triangle PFK\sim P'FK'$,
\[
\frac{SP}{SP'}=\frac{PK}{P'K'}=\frac{PF}{P'F}\,.
\]
By the exterior angle bisector theorem (Theorem~\ref{thm.exterior-angle-bisector}), $FS$ bisects the exterior angle of $\angle P'SP$.\hqed
\end{proof}

\begin{theorem}\label{thm.right-angle}
Let $P$ be a point on the ellipse and construct lines $PA,PA'$. Label their intersections with the directrix by $E$ and $F$, respectively. Then $\angle FSE$ is a right angle (Figure~\ref{f.right-angle}).
\end{theorem}

\begin{proof}
$P,A,A'$ are all points on the ellipse so Theorem~\ref{thm.bisect} applies. $FS$ bisects $\angle PSX=2\gamma$ and $ES$ bisects $\angle P'SX=2\delta$, so $2\gamma + 2\delta= 180^\circ$ and $\angle FSE=\gamma + \delta= 90^\circ$.\hqed
\end{proof}

%%%%%%%%%%%%%%%%%%%%%%%%%%%%%%%%%%%%%%%%%%%%%%%%%%%%%

\begin{figure}[t]
\begin{center}
\begin{tikzpicture}

\clip (-.6,-2.1) rectangle +(10.2,6.2);

% Construct the directrix and the focus
\coordinate (X) at (0,0);
\node[left] at (X) {$X$};
\path[name path=directrix] ($(X)+(0,4)$) -- ($(X)+(0,-2.2)$);
\coordinate (S) at (3,0);
\node[below right] at (S) {$S$};

% Locate the vertices
\coordinate (A) at (2,0);
\node[above left] at (A) {$A$};
\coordinate (AP) at (9,0);
\node[right] at (AP) {$A'$};

% Find a convenient E and construct paths through A, A', S
\path[name path=findE] (S) -- +(212:5);
\path[name intersections = {of = findE and directrix, by = {E} }];
\node[left] at (E) {$E$};
\path[name path=EA] (E) -- ($(E)!2.3!(A)$);
\path[name path=EAP] (E) -- ($(E)!1.1!(AP)$);
\path[name path=ES] (E) -- ($(E)!2!(S)$);

% Locate P, P' and draw EP
\path[name path=SP] (S) -- +(60:2.7) -- +(240:2.2);
\path[name intersections = {of = EA and SP, by = {P} }];
\node[above right] at (P) {$P$};
\path[name intersections = {of = EAP and SP, by = {PP} }];
\node[below] at (PP) {$P'$};
\vertexsmcolor{PP}{red};
\draw (E) -- (P);

% Locate F and draw lines
\path[name path=APF] (AP) -- ($(AP)!2.1!(P)$);
\path[name intersections = {of = APF and directrix, by = {F} }];
\draw (AP) -- (F) -- (E);
\node[left] at (F) {$F$};

% Draw blue angle and red triangle
\draw[name path=SF,thick,blue] (E) -- (S) -- (F);
\draw[red,thick] (P) -- (S) -- (AP) -- cycle;
\draw[red,thick,dashed] (PP) -- (S) -- (X);

% Label angles
\node[above,xshift=0pt,yshift=3pt] at (S) {$\gamma$};
\node[above left,xshift=-6pt,yshift=-1pt] at (S) {$\gamma$};
\node[below left,xshift=-16pt,yshift=1pt] at (S) {$\delta$};
\node[below,xshift=-16pt,yshift=-10pt] at (S) {$\delta$};

\draw[thick,rotate=127] (S) rectangle +(6pt,6pt);

\end{tikzpicture}
\end{center}
\caption{The right angle at the focus}\label{f.right-angle}
\end{figure}

%%%%%%%%%%%%%%%%%%%%%%%%%%%%%%%%%%%%%%%%%%%%%%%%%%%%%%%%%%%%%%%%

\section{Ratios of perpendiculars to the axes}

We start with a preliminary theorem.
\begin{theorem}\label{thm.dividing}
Let $AA'$ be a line segment whose midpoint is $C$. Then 
\[
AC^2-CN^2=AN\cdot NA'\,.
\]
\end{theorem}

\begin{center}
\begin{tikzpicture}[scale=1.2]

\draw (0,0) coordinate (A) node[below] {$A$} -- 
  (2.5,0) coordinate (N) node[below] {$N$} --
  (4,0) coordinate (C) node[below] {$C$} -- 
  (8,0) coordinate (AP) node[below] {$A'$};
\vertexsm{A};
\vertexsm{N};
\vertexsm{C};
\vertexsm{AP};

\end{tikzpicture}
\end{center}
\begin{proof}
$AN=AC-CN$ and $NA'=A'C+CN=AC+CN$ since $C$ is the midpoint of $AA'$. The result is obtaining by multiplying the two equations.\hqed
\end{proof}

\begin{theorem}(Theorem~\ref{thm.ratios})\label{thm.ratios-besant}
Let $P$ be a point on an ellipse not on the major axis and construct perpendiculars $PN,PM$ from $P$ to the major and minor axes, respectively (Figure~\ref{f.besant-ratios}). Then
\begin{eqnlabels}
\frac{PN^2}{A'N\cdot NA}&=&\frac{BC^2}{AC^2} \label{eqn.pnan}\\[4pt]
\frac{PM^2}{B'N\cdot NA}&=&\frac{AC^2}{BC^2}\label{eqn.pmbn}\,.
\end{eqnlabels}
\end{theorem}

%%%%%%%%%%%%%%%%%%%%%%%%%%%%%%%%%%%%%%%%%%%%%%%%%%%%%%%%%%%%%%%%

\begin{figure}[t]
\begin{center}
\begin{tikzpicture}

\clip (-6.8,-2) rectangle +(11.6,6.2);
\def\a{4.5}
\def\b{3}

\pic{semi-ellipse={{\a}/{\b}}};
\draw[thick,dashed] (O) -- +(0,{-\b/3}) node[below] {$B'$};
\node[below] at (F1) {$S$};

% Construct directrix and focus
\coordinate (X) at (-6,0);
\node[left] at (X) {$X$};
\path[name path=directrix] ($(X)+(0,4.5)$) -- ($(X)+(0,-2.5)$);

\node[above left] at (Left) {$A$};
\node[below] at (Right) {$A'$};
\node[below left] at (O) {$C$};
\node[above] at (Top) {$B$};

% Locate P and E and draw lines
\path[name path=fromL,red] (Left) -- +(50:5) -- +(230:4);
\path [name intersections = {of = ellipse and fromL, by = {dummy,P} }];
\node[above] at (P) {$P$};
\path [name intersections = {of = directrix and fromL, by = {E} }];
\node[left] at (E) {$E$};

% Locate F and draw lines
\path[name path=fromR] (Right) -- ($(Right)!1.7!(P)$);
\path [name intersections = {of = directrix and fromR, by = {F} }];
\node[left] at (F) {$F$};
\draw (Right) -- (F);
\draw (F) -- (F1) -- (E) -- cycle;
\draw[rotate=125] (F1) rectangle +(7pt,7pt);

% Construct perpendiculars from P
\path (P) -- ($(Left)!(P)!(O)$) coordinate (N) node[below] {$N$};
\draw (P) -- ($(Top)!(P)!(O)$) coordinate (M) node[right] {$M$};
\draw[rotate=180] (M) rectangle +(6pt,6pt);
\draw[rotate=90] (N) rectangle +(6pt,6pt);

% Draw red and blue similar triangles
\draw[thick,red] (X) -- (E) -- (P) -- (N) -- cycle;
\draw[thick,blue] ($(Right)+(-4pt,2pt)$) -- ($(X)+(0,2pt)$) --
  (F) -- ($(F)!.99!(Right)$);
\draw[thick,blue] ($(P)+(2pt,0)$) -- ($(P)!.98!($(N)+(2pt,0)$)$);

\end{tikzpicture}
\end{center}
\caption{Ratio of an ordinate}\label{f.besant-ratios}
\end{figure}

%%%%%%%%%%%%%%%%%%%%%%%%%%%%%%%%%%%%%%%%%%%%%%%%%%%%%%%%%%%%%%%%

\begin{proof} (Equation \ref{eqn.pnan})
$\triangle AXE\sim \triangle ANP$ (red) since they are right triangles and the vertical angles at $A$ are equal. Therefore,
\begin{equation}
\frac{PN}{AN}=\frac{EX}{AX}\,.\label{eqn.triangle1}
\end{equation}%
$\triangle PA'N\sim \triangle FA'X$ (blue) so
\begin{equation}
\frac{PN}{A'N}=\frac{FX}{A'X}\,.\label{eqn.triangle2}
\end{equation}%
Multiplying Equations~\ref{eqn.triangle1} and \ref{eqn.triangle2} gives
\[
\frac{PN^2}{AN\cdot A'N}=\frac{EX\cdot FX}{AX\cdot A'X}\,.
\]
By Theorem~\ref{thm.right-angle} $\triangle FSE$ is a right triangle so by Theorem~\ref{thm.alt-hypo},
\[
\frac{PN^2}{AN\cdot A'N}=\frac{SX^2}{AX\cdot A'X}\,.
\]
Since $P$ was arbitrary this holds for any point on the ellipse, in particular, for $B$ on the minor axis, where $PN=BC$ and $AN=AN'=AC$. Therefore,
\begin{eqn}
\frac{BC^2}{AC^2}&=&\frac{SX^2}{AX\cdot A'X}\\[4pt]
\frac{PN^2}{AN\cdot A'N}&=&\frac{SX^2}{AX\cdot A'X}=\frac{BC^2}{AC^2}\,.\fqed
\end{eqn}%
\end{proof}

%%%%%%%%%%%%%%%%%%%%%%%%%%%%%%%%%%%%%%%%%%%%%%%%%%%%%%%%%%%%%%%%

\begin{proof} (Equation \ref{eqn.pmbn})
Since $CM=PN, PM=CN$, by Theorem~\ref{thm.dividing}, Equation~\ref{eqn.pnan} becomes 
\begin{eqnlabels}
\frac{CM^2}{AC^2-PM^2}&=&\frac{BC^2}{AC^2}\nonumber\\[4pt]
\frac{AC^2}{AC^2-PM^2}&=&\frac{BC^2}{CM^2}\label{eqn.acbc1}\,.
\end{eqnlabels}%
By inverting the ratios it can be shown that
\begin{equation}
\frac{AC^2}{PM^2}=\frac{BC^2}{BC^2-CM^2},\label{eqn.acbc2}
\end{equation}
and then by Theorem~\ref{thm.dividing} we have
\[
\frac{PM^2}{BM\cdot MB'}=\frac{AC^2}{BC^2}\,.\fqed
\]%
\end{proof}

%%%%%%%%%%%%%%%%%%%%%%%%%%%%%%%%%%%%%%%%%%%%%%%%%%%%%%%%%%%%%%%%

\section{The circle circumscribing an ellipse}

\begin{theorem}(Theorem~\ref{thm.ellipse-b-over-a})\label{thm.ellipse-b-over-a-besant}
Consider circle circumscribed about an ellipse (Figure~\ref{f.ellipse-latus-rectum-geometry}). Choose a point $N$ on the major axis and construct a perpendicular through $N$. Let its intersections with the ellipse and the circle be $P$ and $Q$, respectively. Then
\[
\frac{PN}{QN}=\frac{BC}{AC}\,.
\]
\end{theorem}
\begin{proof} 
From Theorem~\ref{thm.ratios-besant},
\[
\frac{PN^2}{AN\cdot NA'} = \frac{BC^2}{AC^2}\,,
\]
and by Theorem~\ref{thm.alt-hypo}, $AN\cdot NA=QN^2$.\hqed
\end{proof}

%%%%%%%%%%%%%%%%%%%%%%%%%%%%%%%%%%%%%%%%%%%%%%%%%%%%%%%%%%%%%%%%

\begin{figure}
\begin{center}
\begin{tikzpicture}[scale=.9]
\clip (-4.2,-3.5) rectangle +(8.4,7.2);

\def\a{3.2}
\def\b{2}
\pic{ellipse={{\a}/{\b}}};

\draw[name path=circle] circle[radius={\a}];
\node[below] at (F1) {$S$};
\node[below right,xshift=-3pt] at (F2) {$H$};
\node[above] at (Top) {$B$};
\node[below] at (Bot) {$B'$};
\node[right] at (Right) {$A'$};
\node[left] at (Left) {$A$};

% Draw axes through O
\draw (F1) -- (O) -- (F2);
\draw[name path=minor] (O) -- (Top);

\coordinate (X) at (-1,0);
\path[name path=atX] (X) -- ($(X)+(0,4.5)$);
\path [name intersections = {of = atX and ellipse, by = {E1} }];
\path [name intersections = {of = atX and circle, by = {C1} }];
\draw (X) -- (C1);

% Label X and the intersections
\node[below] at (X) {$N$};
\node[above] at (C1) {$Q$};
\node[above left,xshift=2pt,yshift=-2pt] at (E1) {$P$};

\draw (O) rectangle +(6pt,6pt);
\draw (X) rectangle +(6pt,6pt);

% Draw latus rectum as intersections of perpendicular at H
%  with the ellipse
\path[name path=LR] ($(F2)+(0,-2.5)$) -- ($(F2)+(0,2.5)$);
\path [name intersections = {of = ellipse and LR, by = {LR1,LR2} }];
\path [name intersections = {of = circle and LR, by = {LR1c,LR2c} }];
\draw[very thick,red] (LR1) -- (LR2);

% Label endpoint of the LR
\node[above right] at (LR1) {$L_1$};
\node[below right] at (LR2) {$L_2$};
\draw (F2) rectangle +(6pt,6pt);
\draw (F1) -- (Top) -- (F2);

% Length of LR
\draw[<->,dashed,thick] ($(LR1)+(-.3,0)$) --
  node[left,xshift=2pt,yshift=-8pt] {$L$} ($(LR2)+(-.3,0)$);

\end{tikzpicture}
\caption{The circumscribed circle and the latus rectum of an ellipse}\label{f.ellipse-latus-rectum-geometry}
\end{center}
\end{figure}

%%%%%%%%%%%%%%%%%%%%%%%%%%%%%%%%%%%%%%%%%%%%%%%%%%%%%%%%%%%%%%%%

\section{The latus rectum of an ellipse}

The following theorem proves Theorem~\ref{thm.ellipse-lr} in Euclidean geometry.
\begin{theorem}\label{thm.ellipse-lr-besant}
$L$, the length of the latus rectum of an ellipse, is 
$\displaystyle\frac{2BC^2}{AC}$ (Figure~\ref{f.ellipse-latus-rectum-geometry}).
\end{theorem}
\begin{proof}
By Theorem~\ref{thm.ratios-besant},
\[
\frac{HL_1^2}{AH\cdot HA'}=\frac{BC^2}{AC^2}\,.
\]
By Theorem 7.2, $BH=AC$, so by Pythagoras's theorem,
\[
BC^2=BH^2-HC^2=AC^2-HC^2=(AC-HC)(AC+HC)=AS\cdot HA'\,.
\]
Therefore, the length of one-half the latus rectum is
\[
HL_1^2=\frac{BC^4}{AC^2}\,.\fqed
\]%
\end{proof}

%%%%%%%%%%%%%%%%%%%%%%%%%%%%%%%%%%%%%%%%%%%%%%%%%%%%%%%%%%%%%%%%

\section{Areas of parallelograms}

\begin{theorem}\label{thm.perp-tangent}
Let $Y$ be the intersection the perpendicular from the focus $S$ to the tangent $TT'$ at $P$, and let $L$ be the intersection of $S'P$ and $SY$ (Figure~\ref{f.perp-focus-tangent}). Then $Y$ is on the circumscribing circle and $CY\parallel S'L$.
\end{theorem}

%%%%%%%%%%%%%%%%%%%%%%%%%%%%%%%%%%%%%%%%%%%%%%%%%%%%%%

\begin{figure}[b]
\begin{center}
\begin{tikzpicture}

\clip (-5,-.5) rectangle +(13,5.3);

% Size and center of the ellipse
\def\a{4.5}
\def\b{3}
\def\angle{55}

\pic{semi-ellipse={{\a}/{\b}}};
\draw[thick,dotted] (\a,0) 
  arc[start angle=0,end angle=180,radius=\a];

\pic{point-on-ellipse={{\a}/{\b}/{\angle}}};
\node[fill=white,above right,white] at (P) {$P$};
\node[above,xshift=2pt] at (P) {$P$};

\pic{tangent-path={1.5*\a}/{1.5*\b}/{\angle}};
\path[name path=ct] (O) -- ($(O)!2!(Right)$);
\path[name path=bt] (O) -- ($(O)!1.5!(Top)$);
\path [name intersections = {of = t and bt, by = {TP} }];
\path [name intersections = {of = t and ct, by = {T} }];
\node[below] at (T) {$T$};
\node[above,yshift=2pt] at (TP) {$T'$};
\draw (O) -- (T) -- (TP) -- cycle;

% Construct the perpendicular through S to the tangent at Y
\draw (F2) -- ($(T)!(F2)!(TP)$) coordinate (Y);
\node[right,xshift=3pt,yshift=4pt] at (Y) {$Y$};
\draw[thick,dashed] (O) -- (Y);

% Construct lines through P and Y that meet in L
\path[name path=spp] (F1) -- ($(F1)!1.5!(P)$);
\path[name path=sy] (F2) -- ($(F2)!2.3!(Y)$);
\path [name intersections = {of = spp and sy, by = {LL} }];
\node[above right] at (LL) {$L$};
\draw (F1) -- (P) -- node[above] {$d$} (LL) -- 
  (F2) -- node[near start,left] {$d$} (P);

% Lable angles
\node[left,xshift=-3pt] at (P) {\sm{\alpha}};
\node[right,xshift=3pt] at (P) {\sm{\alpha}};
\node[below right,xshift=1pt,yshift=-3pt] at (P) {\sm{\alpha}};
\node[below right,xshift=-1pt,yshift=-2pt] at (TP) {\sm{\beta}};
\node[above right,xshift=2pt,yshift=0pt] at (F2) {\sm{\beta}};

\draw[rotate=-115]  (Y) rectangle +(6pt,6pt);
\draw  (O) rectangle +(6pt,6pt);

\draw[thick,red] ($(F2)+(0,2pt)$) -- ($(T)+(-10pt,2pt)$);
\draw[thick,red] ($(Y)+(0,-2pt)$) -- ($(T)+(-9pt,2pt)$);
\draw[thick,red] (Y) -- (F2);
\draw[thick,blue] (O) -- (TP) -- (T) -- cycle;

\node[below] at (Left) {$A$};
\node[below] at (Right) {$A'$};
\node[below] at (O) {$C$};
\node[below] at (F1) {$S'$};
\node[below] at (F2) {$S$};

\end{tikzpicture}
\caption{The perpendicular from a focus to a tangent}\label{f.perp-focus-tangent}
\end{center}
\end{figure}

%%%%%%%%%%%%%%%%%%%%%%%%%%%%%%%%%%%%%%%%%%%%%%%%%%%%%%%%%%%%%%

\begin{proof}

$\triangle STY\sim\triangle T'TC$ since they are right triangles that share an acute angle, so $\angle CT'T=\angle YST=\beta$. By Theorem~\ref{thm.tangent-angles}, $\angle SPY=\angle S'PT'=\alpha$ since they are the angles to the foci at the tangent. $\angle S'PT'=\angle YPL=\alpha$ are vertical angles, so $\angle SPY=\angle LPY=\alpha$. 

Then $\triangle SPY\cong\triangle LPY$ since they are right triangles with an equal acute angle and a common side $PY$. Therefore, $PL=PS$ and $S'L=S'P+PL=S'P_PS=AA'$. Since $\triangle SPY\cong\triangle LPY$, $SY=YL$, and since $S,S'$ are foci, $S'C=SC$. It follows that $\triangle CSY\sim \triangle S'SL$ and $CY\parallel S'L$. By similarity,
\[
\frac{CY}{S'L}=\frac{CS}{S'S}=\frac{1}{2}\cdot\frac{CS}{CS}=\frac{1}{2}\,.
\]
Therefore, $2CY=S'L=AA'$ so $CY=CA$ and $Y$ is on the circumscribing circle.\hqed
\end{proof}

%%%%%%%%%%%%%%%%%%%%%%%%%%%%%%%%%%%%%%%%%%%%%%%%%%%%%%%%%%%%%%

Figure~\ref{f.perp-perp-tangent} is based on Figure~\ref{f.perp-focus-tangent} with the addition the perpendicular from focus $S'$ intersecting the tangent at $Y'$ and intersecting the extension of $SP$ at $L'$ (not shown).

\begin{theorem}\label{thm.perp-perp-tangent}
$SY\cdot S'Y' = BC^2$.
\end{theorem}

\begin{proof}
Theorem~\ref{thm.perp-tangent} showed that $Y$ is on the circumscribing circle and the same proof shows that $Y'$ is also on the circle. Extend $YS$ to intersect the circle at $Z$ and connect $Y'Z$. Since $\angle Y'YZ$ is a right angle, $Y'Z$ is a diameter and $C$ is on the line. $\triangle SCZ \cong S'C'Y'$ by side-angle-side so $SY\cdot S'Y' = SY \cdot SZ$. But $YZ$ and $AA'$ are secants intersecting at $S$, so
\[
SY\cdot SZ = AS\cdot SA' = AC^2-CS^ 2 = BC^2\,,
\]
by Theorems~\ref{thm.dividing} and \ref{thm.ellipses-features}.\hqed
\end{proof}

%%%%%%%%%%%%%%%%%%%%%%%%%%%%%%%%%%%%%%%%%%%%%%%%%%%%%%

\begin{figure}
\begin{center}
\begin{tikzpicture}

\clip (-5,-5) rectangle +(10.5,11);

% Size and center of the ellipse
\def\a{4.5}
\def\b{3}
\def\angle{55}

\pic{ellipse={{\a}/{\b}}};
\draw[very thick,dotted,name path=circle] (\a,0) 
  arc[start angle=0,end angle=360,radius=\a];

\pic{point-on-ellipse={{\a}/{\b}/{\angle}}};
\node[fill=white,above right,white] at (P) {$P$};
\node[above,xshift=2pt] at (P) {$P$};

\pic{tangent={1.5*\a}/{1.5*\b}/{\angle}};
\path[name path=ct] (O) -- ($(O)!2!(Right)$);
\path[name path=bt] (O) -- ($(O)!1.5!(Top)$);
\path [name intersections = {of = t and bt, by = {TP} }];
\path [name intersections = {of = t and ct, by = {T} }];

% Construct the perpendicular through S to the tangent at Y
\path (F2) -- ($(T)!(F2)!(TP)$) coordinate (Y);
\node[right,xshift=3pt,yshift=4pt] at (Y) {$Y$};
\path (F1) -- ($(T)!(F1)!(TP)$) coordinate (YP);
\node[right,xshift=-12pt,yshift=10pt] at (YP) {$Y'$};

% Construct lines through P and Y that meet in L
\path[name path=spp] (F1) -- ($(F1)!1.5!(P)$);
\path[name path=sy] (F2) -- ($(F2)!2.3!(Y)$);
\path [name intersections = {of = spp and sy, by = {LL} }];
\node[above right] at (LL) {$L$};
\draw (P) -- (F2) -- (LL) -- (F1) -- (YP) -- (O) -- (Y);

\draw[rotate=-115]  (Y) rectangle +(6pt,6pt);
\draw[rotate=-115]  (YP) rectangle +(6pt,6pt);

\node[below left] at (Left) {$A$};
\node[below right] at (Right) {$A'$};
\node[above] at (Top) {$B$};
\node[below,xshift=10pt] at (O) {$C$};
\node[below] at (F1) {$S'$};
\node[below,xshift=4pt] at (F2) {$S$};

\draw[red,thick] (F2) -- ($(F2)!2.3!(P)$);
\draw[red,thick] (F1) -- ($(F1)!1.3!(YP)$);
\node[above,xshift=8pt,yshift=65pt] at (Top) {$L'$};
\node[above,xshift=8pt,yshift=50pt] at (Top) {$\vdots$};

\path[name path=ys] (Y) -- ($(Y)!4!(F2)$);
\path [name intersections = {of = ys and circle, by = {Z} }];
\draw (F2) -- (Z) node[below] {$Z$} -- (O);

\draw[blue,thick] (F2) -- (O) -- (Z) -- cycle;
\draw[blue,thick] (F1) -- (O) -- (YP);
\draw[blue,thick] ($(F1)+(-2pt,0)$) -- ($(YP)+(-2pt,0)$);

\end{tikzpicture}
\caption{Perpendiculars from the foci to the tangent}\label{f.perp-perp-tangent}
\end{center}
\end{figure}

%%%%%%%%%%%%%%%%%%%%%%%%%%%%%%%%%%%%%%%%%%%%%%%%%%%%%%%%%%%%%%

\begin{theorem}\label{thm.cnntacac}
Let $N$ be the intersection of the perpendicular from $P$ to the major axis (Figure~\ref{f.segments-major}). Then $CN\cdot NT = AC^2=AN\cdot NA'$.
\end{theorem}

\begin{figure}[t]
\begin{center}
\begin{tikzpicture}

\clip (-4.8,-.5) rectangle +(14.2,5.1);

% Size and center of the ellipse
\def\a{4.5}
\def\b{3.5}
\def\angle{60}

\pic{semi-ellipse={{\a}/{\b}}};
\pic{point-on-ellipse={{\a}/{\b}/{\angle}}};

\node[below] at (F1) {$S'$};
\node[below] at (F2) {$S$};

\pic{tangent-path={3*\a}/{3*\b}/{\angle}};
\path[name path=ct] (O) -- ($(O)!2!(Right)$);
\path[name path=bt] (O) -- ($(O)!1.5!(Top)$);
\path [name intersections = {of = t and bt, by = {TP} }];
\path [name intersections = {of = t and ct, by = {T} }];
\node[below] at (T) {$T$};
\node[above] at (TP) {$T'$};
\draw (O) -- (T) -- (TP) -- cycle;

\draw (F1) -- (P) -- (F2);

\path[name path=ct] (O) -- ($(O)!2!(Right)$);
\path[name path=bt] (O) -- ($(O)!1.5!(Top)$);
\path [name intersections = {of = t and bt, by = {TP} }];
\path [name intersections = {of = t and ct, by = {T} }];
\node[below] at (T) {$T$};
\node[above] at (TP) {$T'$};

% Locate perpendicular through focus to the tangent at Y
\draw (F2) -- ($(T)!(F2)!(TP)$) coordinate (Y);
\node[above] at (Y) {$Y$};

% Locate perpendicular through P to the major axis
\draw (P) --  ($(Left)!(P)!(Right)$) coordinate (N);
\node[below] at (N) {$N$};
\draw (N) -- (Y) -- (O);

% Label the angles
\node[left,xshift=-3pt,yshift=1pt] at (Y) {\sm{\alpha}};
\node[left,xshift=-3pt] at (P) {\sm{\alpha}};
\node[below right,xshift=0pt,yshift=-2pt] at (P) {\sm{\alpha}};
\node[above right,xshift=2pt,yshift=0pt] at (N) {\sm{\alpha}};

% Draw the triangles
\draw[thick,red] ($(Y)+(-1pt,-2pt)$) -- 
  ($(O)+(8pt,2pt)$) -- ($(T)+(-4pt,2pt)$) -- (Y);
\draw[thick,blue] (O) -- (N) -- (Y) -- cycle;

\draw[rotate=-115]  (Y) rectangle +(6pt,6pt);
\draw[rotate=90]  (N) rectangle +(6pt,6pt);

\node[below] at (Left) {$A$};
\node[below] at (Right) {$A'$};
\node[below] at (O) {$C$};

\end{tikzpicture}
\caption{Ratios of segments of the major axis}\label{f.segments-major}
\end{center}
\end{figure}

%%%%%%%%%%%%%%%%%%%%%%%%%%%%%%%%%%%%%%%%%%%%%%%%%%

\begin{proof}
Continuing with the construction from Figure~\ref{f.perp-focus-tangent}, we focus on the segments $AN,NA'$ (Figure~\ref{f.segments-major}).  

$CY\parallel S'P$ (Theorem~\ref{thm.perp-tangent}) so $\angle CYP = \angle S'PT'=\angle SPY$ by corresponding angles and Theorem~\ref{thm.tangent-angles}. $\angle SYP$ and $\angle SNP$ are right angles and therefore $SYPN$ is quadrilateral that can be circumscribed by a circle whose diameter is $PS$.\footnote{A quadrilateral whose opposite angle are supplementary can be circumscribed by a circle. If two opposite angles are right angles, they sum to $180^\circ$ so the other two angles must also sum to $180^\circ$.} Therefore, $\angle SPY = \angle SNY$ since they are subtended by the same chord $YS$.

Since $\angle CYT\sim \angle CNY$ and $\angle 	YCT$ is a common angle,
$\triangle CYT\sim \triangle CNY$ and
\begin{eqnlabels}
\frac{CN}{CY}&=&\frac{CY}{CT}\nonumber\\[6pt]
CN\cdot CT&=&CY^2=AC^2\,,\label{eqn.cnctacac}
\end{eqnlabels}%
since the perpendicular to the tangent from a focus is on the circumscribing circle (Theorem~\ref{thm.perp-tangent}). Since $NT=CT-CN$, we have $CN\cdot NT = CN\cdot CT - CN^2$ which equals $AC^2-CN^2$ by Equation~\ref{eqn.cnctacac}. This in turn equals $AN\cdot NA'$ by Theorem~\ref{thm.dividing}.\hqed
\end{proof}

%%%%%%%%%%%%%%%%%%%%%%%%%%%%%%%%%%%%%%%%%%%%%%%%%%%%%%%%%%%%%%

\begin{figure}[t]
\begin{center}
\begin{tikzpicture}

\clip (-5,-3.2) rectangle +(12.4,7.4);

% Size and center of the ellipse
\def\a{4.5}
\def\b{3}
\def\angle{50}

\pic{ellipse={{\a}/{\b}}};
\pic{point-on-ellipse={{\a}/{\b}/{\angle}}};
\pic{tangent-path={{1.5*\a}/{1.5*\b}/{\angle}}};

\path[name path=bt] (O) -- ($(O)!1.7!(Top)$);
\path [name intersections = {of = t and bt, by = {T} }];
\node[left] at (T) {$T$};

% Conjugate diameters from D through C
\path[name path=cd] (O) -- +({\angle+90}:{1.5*\a} and {1.5*\b});
\path[name path=dk] (O) -- +({\angle+90}:{-1.5*\a} and {-1.5*\b});
\path [name intersections = {of = cd and ellipse, by = {D} }];
\path [name intersections = {of = dk and ellipse, by = {K} }];
\node[above left,xshift=3pt,yshift=-2pt] at (D) {$D$};
\node[below right,xshift=0pt,yshift=0pt] at (K) {$K$};
\draw (D) -- (K);

% Perpendicular through P on major axis and 
%   intersection with the conjugate diameter
\draw (P) -- ($(O)!(P)!(Right)$) coordinate (N);
\node[below right] at (N) {$N$};
\path[name path=pk] (P) -- ($(P)!2!(N)$);
\path [name intersections = {of = pk and dk, by = {J} }];
\node[below left,yshift=4pt] at (J) {$J$};
\draw (N) -- (J);

% Locate normal to the tangent and its 
%   intersection with the conjugate diameter and the major axis
\draw[name path=pf] (P) -- ($(D)!(P)!(K)$) coordinate (F);
\node[below left] at (F) {$F$};
\path [name intersections = {of = pf and major, by = {G} }];
\node[above,xshift=-3pt,yshift=2pt] at (G) {$G$};

% Find intersection of tangent with the major axis
\path[name path=semimajor] (O) -- ($(O)!1.6!(Right)$);
\path [name intersections = {of = t and semimajor, by = {TP} }];
\node[below] at (TP) {$T'$};
\draw (Top) -- (T) -- (TP) -- (Right);

% Draw colored triangles and quadrilateral
\path[fill,black!10!white] (N) -- (G) -- (F) -- (J) -- cycle;
\draw[thick,red] (P) -- (N) -- (G) -- cycle;
\draw[thick,red] (O) -- (G) -- (F) -- cycle;
\draw[thick,blue] (O) -- (T) -- ($(TP)+(-3pt,2pt)$);
\draw[thick,blue] ($(TP)+(-3pt,2pt)$) -- ($(O)+(0,2pt)$);
\draw[thick,blue] ($(P)+(2pt,0)$) -- ($(N)+(2pt,0)$);

% Label angles
\node[above right,xshift=2pt,yshift=1pt] at (G) {\sm{\alpha}};
\node[below left,xshift=-1pt,yshift=1pt] at (G) {\sm{\alpha}};
\node[below right,xshift=-4pt,yshift=0pt] at (G) {\sm{180^\circ\!-\!\alpha}};
\node[below right,xshift=12pt,yshift=2pt] at (O) {\sm{\beta}};
\node[above left,xshift=0pt,yshift=4pt] at (O) {\sm{\alpha}};
\node[below,xshift=-4pt,yshift=-8pt] at (P) {\sm{\beta}};
\node[below right,xshift=-2pt,yshift=-4pt] at (T) {\sm{\alpha}};
\node[above left,xshift=2pt,yshift=2pt] at (J) {\sm{\alpha}};

\draw[rotate=-32]  (F) rectangle +(6pt,6pt);
\draw[rotate=180]   (N) rectangle +(6pt,6pt);
\draw[rotate=-212] (P) rectangle +(6pt,6pt);
\draw[rotate=0]    (O) rectangle +(6pt,6pt);

\node[above left] at (Top) {$B$};
\node[below right] at (Right) {$A'$};
\node[below left] at (Left) {$A$};
\node[below left] at (O) {$C$};

\end{tikzpicture}
\caption{Parallelograms formed by conjugate diameters}\label{f.parallelogram-conjugate-diameters}
\end{center}
\end{figure}

%%%%%%%%%%%%%%%%%%%%%%%%%%%%%%%%%%%%%%%%%%%%%%%%%%%%%%%%%%%%%%

Construct the normal to the tangent at $P$ and let its intersection with the conjugate diameter $DK$ be $F$ and its intersection with the major axis be $G$. Construct a perpendicular from $P$ to the major axis and let its intersection be $N$. Let the intersection of the tangent with the minor axis be $T$ and its intersection with the major axis be $T'$ (Figure~\ref{f.parallelogram-conjugate-diameters}).
\begin{theorem}\label{thm.parallelogram1}
$PF\cdot PG = BC^2$.
\end{theorem}

%%%%%%%%%%%%%%%%%%%%%%%%%%%%%%%%%%%%%%%%%%%%%%%%%%%%%%%%%%%%%%

\begin{proof}
$\triangle NPG \sim \triangle FPJ$ so $\angle PGN=\angle PJF = \alpha$ and
\begin{eqnlabels}
\frac{PF}{PN}&=&\frac{PJ}{PG}\nonumber\\[6pt]
PF\cdot PG &=& PJ\cdot PN\,.\label{eqn.pnpj}
\end{eqnlabels}%
By vertical angles $\angle PGN = \angle CGF = \alpha$ so 
$\triangle NPG\sim \triangle FCG$ and $\angle NPG = \angle FCG = \beta =90^\circ-\alpha$. By adding $\beta$ to the right angles $\angle BCN$ and $\angle TPF$, we get that $\angle TCJ = \angle TPJ$ and therefore $TPJC$ is a parallelogram, so $CT=PJ$ and $PF\cdot PG =PJ\cdot PN= CT\cdot PN$. By Equation~\ref{eqn.pnpj}, the theorem will be proven if we can show that $CT\cdot PN=BC^2$.

$\triangle TT'C\sim \triangle PT'N$ so 
\begin{eqn}
\frac{CT}{CT'}&=&\frac{PN}{NT'}\\[6pt]
\frac{CT}{PN}&=&\frac{CT'}{NT'}\,.
\end{eqn}

Multiplying each side by fractions equal to $1$ gives
\[
\frac{CT\cdot PN}{PN^2}=\frac{CT'\cdot CN}{CN \cdot NT'}\,.
\]
By Equation~\ref{eqn.cnctacac}, $CN\cdot CT' = AC^2$, and by Theorem~\ref{thm.cnntacac}, $CN\cdot NT'=AN\cdot NA'$, so
\[
\frac{CT\cdot PN}{PN^2}=\frac{AC^2}{AN\cdot NA'}\,.
\]
Multiplying the right-hand side by $PN^2/PN^2$ and use Theorem~\ref{thm.ratios-besant} to get
\begin{eqnlabels}
\frac{CT\cdot PN}{PN^2}&=&\frac{PN^2}{AN\cdot NA'}\cdot \frac{AC^2}{PN^2}=\frac{BC^2}{AC^2}\cdot \frac{AC^2}{PN^2}\nonumber\\[6pt]
CT\cdot PN&=&BC^2\,.\label{eqn.ctpn}\fqed
\end{eqnlabels}
\end{proof}

%%%%%%%%%%%%%%%%%%%%%%%%%%%%%%%%%%%%%%%%%%%%%%%%%%%%%%

Let $PP',DK$ be conjugate diameters, let $DT',PT$ be tangents to $D,P$, respectively, where $T',T$ are their intersections with the major axis. Let $DM, PN$ be perpendiculars to the major axis (Figure~\ref{f.parallelogram3}).
\begin{theorem}\label{thm.cmpnacbc}
\[
\begin{array}{l@{\hspace{3em}}l}
CN^2=AM\cdot MA'& CM^2=AN\cdot NA'\\[6pt]
\displaystyle\frac{DM}{CN}=\frac{BC}{AC}& \displaystyle\frac{CM}{PN}=\frac{AC}{BC}\,.
\end{array}
\]
\end{theorem}

%%%%%%%%%%%%%%%%%%%%%%%%%%%%%%%%%%%%%%%%%%%%%%%%%%%%%%%%%%%%%%%%%%%%%%

\begin{figure}[t]
\begin{center}
\begin{tikzpicture}

\clip (-5,-2.6) rectangle +(9.5,5.3);

% Size and center of the ellipse
\def\a{3}
\def\b{2}
\def\angle{40}

\pic{ellipse={\a}/{\b}};
\pic{point-on-ellipse={\a}/{\b}/{\angle}};
\path[name path global=t] 
  (P) -- ++ ({-2*\a*sin(\angle)},{2*\b*cos(\angle)})
  (P) -- ++ ({\a*sin(\angle)},{-\b*cos(\angle)});

% Draw tangent at P and locate intersection
%  with the extension of the major axis
\path[name path=tans] ($(Left)+(-4,0)$) -- ($(Right)+(2,0)$);
\path [name intersections = {of = t and tans, by = {T} }];
\node[right] at (T) {$T$};
\draw (P) -- (T);

% Diameter from P through C
\draw[name path=pc] (P) -- ($(P)!2!(O)$) coordinate (G);
\node[below left] at (G) {$P'$};

% Conjugate diameter from D through C
\path[name path=cd] (O) -- +({\angle+90}:{1.5*\a} and {1.5*\b});
\path[name path=dk] (O) -- +({\angle+90}:{-1.5*\a} and {-1.5*\b});
\path [name intersections = {of = cd and ellipse, by = {D} }];
\path [name intersections = {of = dk and ellipse, by = {K} }];
\node[above left] at (D) {$D$};
\node[below right] at (K) {$K$};
\draw (D) -- (K);

\path[name path global=t3] 
  (D) -- ++ ({3*\a*sin(\angle+90)},{-3*\b*cos(\angle+90)})
  (D) -- ++ ({-2*\a*sin(\angle+90)},{2*\b*cos(\angle+90)});
\path [name intersections = {of = tans and t3, by = {TP} }];
\draw (D) -- (TP) node[below] {$T'$};
\draw (T) -- (TP);

% Construct perpendiculars from P, D to the major axis
\draw (P) -- ($(Left)!(P)!(Right)$) coordinate (N) node[below] {$N$};
\draw (D) -- ($(Left)!(D)!(Right)$) coordinate (MM) node[below] {$M$};

\draw[rotate=90]   (N) rectangle +(5pt,5pt);
\draw[rotate=0]    (MM) rectangle +(5pt,5pt);

\node[above] at (Top) {$B$};
\node[below] at (Bot) {$B'$};
\node[below left] at (Left) {$A'$};
\node[below right] at (Right) {$A$};
\node[below left,xshift=3pt,yshift=-2pt] at (O) {$O$};

\end{tikzpicture}
\caption{Ratios of perpendiculars to the major axis}\label{f.parallelogram3}
\end{center}
\end{figure}

%%%%%%%%%%%%%%%%%%%%%%%%%%%%%%%%%%%%%%%%%%%%%%%%%%%%%%

\begin{proof}
By Theorem~\ref{thm.cnntacac},
\begin{eqn}
CN\cdot CT &=& AC^2=CM\cdot CT'\\[6pt]
\frac{CM}{CN} &=& \frac{CT}{CT'}\,.
\end{eqn}%
Since $DK$ and $PP'$ are conjugate diameters, $DT'\parallel PP'$ and $\triangle T'DC \sim \triangle CPT$, so
\begin{eqn}
\frac{CM}{CN} &=& \frac{CT}{CT'} =  \frac{CN}{MT'}\\[6pt]
CN^2 &=& CM\cdot MT'\,,
\end{eqn}%
Therefore,
\begin{equation}
CN^2=CM\cdot MT'=AC^2 = AM\cdot MA'\label{eqn.cnannap}
\end{equation}%
by Theorem~\ref{thm.cnntacac}. By Theorem~\ref{thm.ratios-besant},
\[
\frac{DM^2}{AM\cdot MA'}=\frac{BC^2}{AC^2}\,,
\]
and by Equation~\ref{eqn.cnannap},
\begin{eqn}
\frac{DM^2}{CN^2}&=&\frac{BC^2}{AC^2}\\[6pt]
\frac{DM}{CN}&=&\frac{BC}{AC}\,,
\end{eqn}%
A symmetric argument shows that $CM^2= AN\cdot NA'$ and $CM/PN=BC/AC$.\hqed
\end{proof}

%%%%%%%%%%%%%%%%%%%%%%%%%%%%%%%%%%%%%%%%%%%%%%%%%%%%%%%%%%%%%%%%%%%%%%

\begin{theorem}(Theorem~\ref{thm.conj-diam-para})\label{thm.area-parallelogram}
The area of the parallelogram formed by the tangents at the ends of the conjugate diameters $PP',DK$ is equal to the area of the rectangle enclosing the ellipse at the ends of the axes (Figure~\ref{f.parallelogram2}).
\end{theorem}

%%%%%%%%%%%%%%%%%%%%%%%%%%%%%%%%%%%%%%%%%%%%%%%%%%%%%%%%%%%%%%%%%%%%%%

\begin{figure}[t]
\begin{center}
\begin{tikzpicture}
\clip (-5.2,-3.5) rectangle +(10.4,7.5);

% Size and center of the ellipse
\def\a{3.75}
\def\b{2.5}
\def\angle{30}

\pic{ellipse={\a}/{\b}};
\pic{point-on-ellipse={\a}/{\b}/{\angle}};
\pic{tangent={\a}/{\b}/{\angle}};

% Conjugate diameter from P through C
\draw[name path=pc] (P) -- ($(P)!2!(O)$) coordinate (G);
\node[below,xshift=-4pt,yshift=2pt] at (G) {$P'$};

% Conjugate diameters from D through C
\path[name path=cd] (O) -- +({\angle+90}:{1.5*\a} and {1.5*\b});
\path[name path=dk] (O) -- +({\angle+90}:{-1.5*\a} and {-1.5*\b});
\path [name intersections = {of = cd and ellipse, by = {D} }];
\path [name intersections = {of = dk and ellipse, by = {K} }];
\node[above left,xshift=1pt,yshift=-5pt] at (D) {$D$};
\node[below right,xshift=1pt,yshift=3pt] at (K) {$K$};
\draw (D) -- (K);

% Draw tangent at G (continuation of PC)
\draw[name path global=t2] 
  (G) -- ++ ({-\a*sin(\angle)},{\b*cos(\angle)})
  (G) -- ++ ({\a*sin(\angle)},{-\b*cos(\angle)});
% Draw tangent at D
\draw[name path global=t3] 
  (D) -- ++ ({-\a*sin(90+\angle)},{\b*cos(90+\angle)})
  (D) -- ++ ({\a*sin(90+\angle)},{-\b*cos(90+\angle)});
% Draw tangent at K (continuation of DC)
\draw[name path global=t4] 
  (K) -- ++ ({-\a*sin(90+\angle)},{\b*cos(90+\angle)})
  (K) -- ++ ({\a*sin(90+\angle)},{-\b*cos(90+\angle)});

% Draw large dashed rectangle
\draw[thick,dashed] (\a,\b) -- (-\a,\b) -- (-\a,-\b) -- (\a,-\b) -- cycle;

% Get the intersections of the tangents
\path [name intersections = {of = t  and t3, by = {J} }];
\path [name intersections = {of = t2 and t3, by = {KK} }];
\path [name intersections = {of = t  and t4, by = {M} }];
\path [name intersections = {of = t2 and t4, by = {LL} }];

% Draw the parallelogram
\draw (J) -- (KK) -- (LL) -- (M) -- cycle;
\node[above right] at (J) {$L$};

% Locate perpendiculars of P, D with major axes
\draw (P) -- ($(Left)!(P)!(Right)$) coordinate (N) node[below] {$N$};
\draw (D) -- ($(Left)!(D)!(Right)$) coordinate (M) node[below] {$M$};

% Locate normal from P to the conjugate diameter
%   and its intersection with the major axes
\draw[name path=pf] (P) -- ($(D)!(P)!(K)$) coordinate (F);
\node[below left] at (F) {$F$};
\path [name intersections = {of = pf and major, by = {GG} }];
\node[above left] at (GG) {$G$};

% Draw colored triangles
\draw[thick,red]  (O) -- (M) --  (D) -- cycle;
\draw[thick,red] (GG) -- (N) --  (P) -- cycle;
\draw[thick,red] (O) -- (F) --  (GG) -- cycle;

\draw[rotate=-49]  (F) rectangle +(6pt,6pt);
\draw[rotate=90]   (N) rectangle +(6pt,6pt);
\draw[rotate=128] (P) rectangle +(6pt,6pt);
\draw[rotate=0]    (M) rectangle +(6pt,6pt);

% Draw angles
\node[below right,xshift=-2pt,yshift=-8pt] at (D) {\sm{\alpha'}};
\node[above left,xshift=-4pt,yshift=-1pt] at (O)  {\sm{\alpha}};

\node[below right,xshift=10pt,yshift=0pt] at (O) {\sm{\alpha}};
\node[below left,xshift=-4pt,yshift=2pt] at (GG) {\sm{\alpha'}};

\node[above right,xshift=5pt,yshift=-2pt] at (GG) {\sm{\alpha'}};
\node[below left,xshift=0pt,yshift=-9pt] at (P) {\sm{\alpha}};

\node[above right] at (Top) {$B$};
\node[below left] at (Bot) {$B'$};
\node[above left] at (Left) {$A'$};
\node[below right] at (Right) {$A$};
\node[below left] at (O) {$O$};

\end{tikzpicture}
\caption{Areas of parallelograms ($\alpha'=90^\circ\!-\!\alpha$)}\label{f.parallelogram2}
\end{center}
\end{figure}

%%%%%%%%%%%%%%%%%%%%%%%%%%%%%%%%%%%%%%%%%%%%%%%%%%%%%%%

\begin{proof}
By the definition of conjugate diameters, it is sufficient to show that the area of $PCDL$ is $AC\cdot BC$. The area of a parallelogram is width times height so we need to show that $CD\cdot PF=AC\cdot BC$. 

By vertical angles $\angle DCM = \angle GCF$ and $\angle CGF = \angle PGN$, so $\triangle DCM\sim \triangle PGN$ 
\[
\frac{PG}{CD}=\frac{PN}{CM}=\frac{BC}{AC}\,,
\]
by Theorem~\ref{thm.cmpnacbc}. From Theorem~\ref{thm.parallelogram1}, $PF\cdot PG = BC^2$, giving
\begin{equation}
\frac{BC}{PF}=\frac{PG}{BC}=\frac{CD}{AC}\,.\fqed
\end{equation}%
\end{proof}

