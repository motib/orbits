% !TeX root = orbits.tex

\section{Ellipses in Euclidean geometry}\label{s.geometry}

The proofs of theorems about planetary freely used analytic geometry and trigonometry, but for many years after the invention of analytic geometry, mathematicians continued to limit themselves to Euclidean geometry. In this section, I present proofs in Euclidean geometry of theorems that appeared in Section~\ref{s.ellipse}.\footnote{Theorems~\ref{thm.tangent-angles}, \ref{thm.conj-diag} were already proved using Euclidean geometry and Theorem~\ref{thm.ellipse-area} requires taking limits.} The proofs are based on the definition of ellipses in terms of the geometric concepts of focus and directrix instead of the familiar analytic definition (Equation~\ref{eqn.ellipse-formula}).

%%%%%%%%%%%%%%%%%%%%%%%%%%%%%%%%%%%%%%%%%%%%%%%%%%%%%%%%%%%%%%%%%%%%%

\subsection{The definition of an ellipse using the focus and the directrix}

\begin{definition}\label{def.ellipse}
Let $d$ be a line (the \emph{directrix}) and $S$ be a point (the \emph{focus}) not on the directrix. Let $0<e<1$ be a number (the \emph{eccentricity}). An \emph{ellipse} is the locus of points $P$ such that the ratio of  $PS$ to the distance of $P$ to the directrix is $e$.
\end{definition}

All the conic sections (parabolas, ellipses and hyperbolas) are defined the same way and are distinguished by their eccentricity.
\begin{definition}
Let $X$ be the intersection of the perpendicular to the directrix from $S$. $A$ on $SX$ is a \emph{vertex} of the ellipse if $SA/AX=e$ (in Figure~\ref{f.def-ellipse}, $e=1/2$).
\end{definition}

%%%%%%%%%%%%%%%%%%%%%%%%%%%%%%%%%%%%%%%%%%%%%%%%%%%%%%%%%%

\begin{figure}[b]
\begin{center}
\begin{tikzpicture}

% Construct the directrix and the focus
\coordinate (X) at (0,0);
\node[left] at (X) {$X$};
\draw[name path=directrix] ($(X)+(0,2)$) --
  node[left,near start] {$d$} ($(X)+(0,-2)$);
\coordinate (S) at (3,0);
\node[below] at (S) {$S$};

% Locate the vertices with eccentricity 1/2
\coordinate (A) at (2,0);
\node[below] at (A) {$A$};
\coordinate (AP) at (6,0);
\node[below] at (AP) {$A'$};

% draw the major axis
\draw  (AP) -- node[above] {$3$} (S) -- 
  node[above left] {$1$} (A) -- node[above] {$2$} (X);

%\vertexsm{X};
\vertexsm{S};
\vertexsm{A};
\vertexsm{AP};
\end{tikzpicture}
\end{center}
\caption{The elements of the definition of an ellipse}\label{f.def-ellipse}
\end{figure}

%%%%%%%%%%%%%%%%%%%%%%%%%%%%%%%%%%%%%%%%%%%%%%%%%%%%%%%%%%

Given $e=SA/AX$ we can locate $A$ as follows:
\begin{eqn}
SA+AX&=&SX\\[4pt]
SA&=&SX-\frac{SA}{e}=SX\cdot \frac{e}{1+e}\\[4pt]
AX&=&SX\cdot \frac{1}{1+e}\,.
\end{eqn}

\pagebreak[3]

Similarly, there is a second vertex $A'$ where
\begin{eqn}
A'X-SA'&=&SX\\[4pt]
SA'&=&SX+\frac{SA'}{e}=SX\cdot \frac{e}{1-e}\\[4pt]
A'X&=&SX\cdot \frac{1}{1-e}\,.
\end{eqn}%

%%%%%%%%%%%%%%%%%%%%%%%%%%%%%%%%%%%%%%%%%%%%%%%%%%%%%%%%%%

Definition~\ref{def.ellipse} is non-constructive. It states that the ellipse is the locus of points satisfying a certain property, but aside from the vertices we have not constructed any such points. Here we show how to construct any of the points on the ellipse.

Select an \emph{arbitrary} point $E$ on the directrix and construct lines from $E$ through $A$ and $S$. The line through $S$ will make an angle $\alpha$ with $SX$. Construct a line from $S$ at the \emph{same angle} $\alpha$ from $ES$ and let its intersection with $EA$ be $P$. Construct the perpendicular from $P$ to the directrix and let $K$ be its intersection with the directrix. Let $L$ be the intersection of $KP$ with $ES$ (Figure~\ref{f.construct}). 

%%%%%%%%%%%%%%%%%%%%%%%%%%%%%%%%%%%%%%%%%%%%%%%%%%%%%%%%%%

\begin{figure}[t]
\begin{center}
\begin{tikzpicture}[scale=1.2]

\clip (-1,-2) rectangle +(9,4.5);

% Construct the directrix, the focus and the major axis
\coordinate (X) at (0,0);
\node[left] at (X) {$X$};
\draw[name path=directrix] ($(X)+(0,2.2)$) --
  node[left,near start] {$d$} ($(X)+(0,-2.2)$);
\coordinate (S) at (3,0);
\node[below right] at (S) {$S$};
\draw (X) -| ($(X)!2.4!(S)$) node[right] {$N$};

% Locate a vertex
\coordinate (A) at (2,0);
\node[above left] at (A) {$A$};

% Find a convenient E and draw _paths_ to A and S
\path[name path=findE] (S) -- +(210:4);
\path [name intersections = {of = findE and directrix, by = {E} }];
\node[left] at (E) {$E$};
\path[name path=EA] (E) -- ($(E)!2.3!(A)$);
\path[name path=ES] (E) -- ($(E)!2.3!(S)$);

% Locate P by drawing a path with the same angle
\path[name path=SP] (S) -- +(60:3);
\path [name intersections = {of = EA and SP, by = {P} }];
\node[above] at (P) {$P$};
\draw (S) -- (P);

% Construct the perpendicular through P to the directrix
\path[name path=PK] (P) -- +(180:4.5);
\path [name intersections = {of = PK and directrix, by = {K} }];
\node[left] at (K) {$K$};
\draw[name path=PL] (K) -- ($(K)!2!(P)$);

% Locate L and label the segments
\path [name intersections = {of = PL and ES, by = {L} }];
\node[above] at (L) {$L$};
\path (S) -- node[left] {$b$} (P) -- node[above] {$b$} (L);

% Label the angles
\node[right,xshift=12pt,yshift=5pt] at (S) {$\alpha$};
\node[above right,xshift=6pt,yshift=6pt] at (S) {$\alpha$};
\node[below left,xshift=-12pt,yshift=2pt] at (L) {$\alpha$};

% Draw the colored triangles
\draw[red,very thick] (P) -- (L) -- (E) -- cycle;
\draw[blue,very thick] (E) -- (K) -- ($(P)+(-3pt,0)$) -- ($(E)+(0,3pt)$);

\end{tikzpicture}
\end{center}
\caption{Constructing points on the ellipse}\label{f.construct}
\end{figure}

%%%%%%%%%%%%%%%%%%%%%%%%%%%%%%%%%%%%%%%%%%%%%%%%%%%%%%%%%%

\begin{theorem}\label{thm.point-on-an-ellipse}
The point $P$ is on the ellipse.
\end{theorem}
\begin{proof}
$\angle PLS = \angle LSN=\alpha$ by alternate interior angles, so $\triangle LPS$ is isosceles and $PL=SP$. Since $PK\parallel SX$, $\triangle XEA\sim \triangle KEP$ and $\triangle AES\sim \triangle PEL$ are adjacent pairs of similar triangles, so
\[
\frac{PS}{PK}=\frac{PL}{PK}=\frac{SA}{AX}=e\,.
\]
Therefore, $P$ is on the ellipse.\hqed
\end{proof}

By choosing different points $E$ on the directrix, any point on the ellipse can be constructed.

%%%%%%%%%%%%%%%%%%%%%%%%%%%%%%%%%%%%%%%%%%%%%%%%%%%%%%%%%%

%\begin{figure}[b]
%\begin{center}
%\begin{tikzpicture}[scale=1.1]
%
%\clip (-1,-2) rectangle +(9,4.5);
%
%% Construct directrix, focus and major axis
%\coordinate (X) at (0,0);
%\node[left] at (X) {$X$};
%\draw[name path=directrix] ($(X)+(0,2.2)$) -- ($(X)+(0,-2.2)$);
%\coordinate (S) at (3,0);
%\node[below right] at (S) {$S$};
%\draw (X) -| ($(X)!2.3!(S)$) node[right] {$N$};
%
%% Locate O
%\draw[name path=SP] (S) -- +(60:2) coordinate (P);
%\node[above left] at (P) {$P$};
%
%% Find a convenient F (but called E)
%\path[name path=findE] (S) -- +(210:4);
%\path [name intersections = {of = findE and directrix, by = {E} }];
%\node[left] at (E) {$F$};
%
%% Construct perpendicular through P from the directrix
%\path[name path=PK] (P) -- +(180:4.5);
%\path [name intersections = {of = PK and directrix, by = {K} }];
%\node[left] at (K) {$K$};
%\path[name path=PL] (K) -- ($(K)!2!(P)$);
%\draw[name path=ES] (E) -- ($(E)!2!(S)$);
%\path [name intersections = {of = PL and ES, by = {PP} }];
%\node[above] at (PP) {$P'$};
%
%% Draw triangles
%\draw[blue,thick] (K) -- (S);
%\draw[red,thick] (PP) -- (S) -- (P) -- cycle;
%\draw[thick,dashed,red] (K) -- (P);
%
%% Label angles
%\draw[thick] ($(S)+(60:10pt)$)
%  arc[start angle=60,end angle=210,radius=10pt];
%\node[above,xshift=-4pt,yshift=8pt] at (S) {$\beta$};
%\node[left,xshift=-18pt,yshift=6pt] at (S) {$\beta$};
%
%\end{tikzpicture}
%\end{center}
%\caption{Constructing another vertex on $KP$}\label{f.other-vertex}
%\end{figure}

%%%%%%%%%%%%%%%%%%%%%%%%%%%%%%%%%%%%%%%%%%%%%%%%%%%%%%%%%%

%We now construct another point on $KP$. Construct a line from $F$ on the directrix through $S$ such that $\angle FSK = \angle KSP=\beta$ and let $P'$ be the intersection of $FS$ with $KP$ (Figure~\ref{f.other-vertex}).
%\begin{theorem}
%$P'$ is on the ellipse.
%\end{theorem}
%
%\enlargethispage{\baselineskip}
%\begin{proof}
%Since $KS$ bisects $\angle PSF$, by the exterior angle bisector theorem 
%\begin{eqn}
%\frac{SP'}{SP}&=&\frac{P'K}{PK}\\[4pt]
%\frac{SP'}{P'K}&=&\frac{SP}{PK}=e\,,
%\end{eqn}%
%and therefore $P$ is on the ellipse.\hqed
%\end{proof}

%%%%%%%%%%%%%%%%%%%%%%%%%%%%%%%%%%%%%%%%%%%%%%%%%%%%%%%%%%

%\subsection{Constructing a focal chord}
%
%\begin{definition}
%Let $P,P'$ be points on an ellipse. $PP'$ is a \emph{focal chord} of the ellipse if $PP'$ is a chord through a focus (Figure~\ref{f.focal-chord}).
%\end{definition}
%
%%%%%%%%%%%%%%%%%%%%%%%%%%%%%%%%%%%%%%%%%%%%%%%%%%%%%%%%%%%
%
%\begin{figure}
%\begin{center}
%\begin{tikzpicture}[scale=1.4]
%
%\clip (-.5,-2) rectangle +(8,4.2);
%
%% Construct directrix, focus and major axis
%\coordinate (X) at (0,0);
%\node[left] at (X) {$X$};
%\coordinate (S) at (3,0);
%\node[below right] at (S) {$S$};
%\path[name path=directrix] ($(X)+(0,2.2)$) -- ($(X)+(0,-2.2)$);
%\draw (X) -| ($(X)!2.3!(S)$) node[right] {$N$};
%
%% Locate a vertex
%\coordinate (A) at (2,0);
%\node[above left] at (A) {$A$};
%
%% Find a convenient E and construct lines through A and S
%\path[name path=findE] (S) -- +(210:4);
%\path [name intersections = {of = findE and directrix, by = {E} }];
%\node[left] at (E) {$E$};
%\path[name path=EA] (E) -- ($(E)!2.3!(A)$);
%\path[name path=ES] (E) -- ($(E)!2.3!(S)$);
%
%% Locate P
%\path[name path=SP] (S) -- +(60:3.5) coordinate (Q);
%\path [name intersections = {of = EA and SP, by = {P} }];
%\node[above] at (P) {$P$};
%
%% Construct a perpendicular through P to the directrix
%\path[name path=PK] (P) -- +(180:4.5);
%\path [name intersections = {of = PK and directrix, by = {K} }];
%\node[left] at (K) {$K$};
%
%% Locate L
%\draw[name path=PL] (K) -- ($(K)!1.8!(P)$);
%\path [name intersections = {of = PL and ES, by = {L} }];
%\node[above] at (L) {$L$};
%
%% Draw triangle PLE and directrix
%\draw (P) -- (L) -- (E) -- cycle;
%\draw (E) -- (K);
%
%% Locate A', P' and construct lines
%\coordinate (AP) at (6,0);
%\node[above] at (AP) {$A'$};
%\draw[name path=EAP,very thick] (E) -- (AP);
%\path[name path=SPP] (P) -- ($(P)!2!(S)$);
%\path [name intersections = {of = SPP and EAP, by = {PP} }];
%\draw[very thick] (P) -- (PP);
%\node[below] at (PP) {$P'$};
%
%% Locate K', L' and construct lines
%\path[name path=PKP] (PP) -- +(-2.6,0);
%\path [name intersections = {of = PKP and directrix, by = {KP} }];
%\node[left] at (KP) {$K'$};
%\draw[very thick] (PP) -- (KP);
%\path [name intersections = {of = PKP and ES, by = {LP} }];
%%\node[above,xshift=4pt,yshift=2pt] at (LP) {$L'$};
%
%\node[above,xshift=-14pt,yshift=10pt] at (LP) {$L'$};
%
%\draw[->] ($(LP)+(-10pt,8pt)$) -- +(-40:10pt);
%
%% Label angles
%\node[above right,xshift=13pt,yshift=-1pt] at (LP) {$\alpha$};
%\node[below left,xshift=-7pt,yshift=-7pt] at (S) {$\alpha$};
%\node[right,xshift=14pt,yshift=5pt] at (S) {$\alpha$};
%\node[above right,xshift=8pt,yshift=8pt] at (S) {$\alpha$};
%\node[below left,xshift=-14pt,yshift=1pt] at (L) {$\alpha$};
%
%% Draw red and blue triangles
%\draw[thick,red] (X) -- (S) -- (E);
%\draw[thick,red] (KP) -- (LP);
%\draw[thick,red] ($(E)+(-3pt,0)$) -- ($(S)+(-3pt,0)$);
%\draw[thick,blue] (S) -- (AP) -- (E) -- cycle;
%\draw[thick,blue] (LP) -- (PP);
%
%\end{tikzpicture}
%\end{center}
%\caption{Constructing a focal chord}\label{f.focal-chord}
%\end{figure}
%
%%%%%%%%%%%%%%%%%%%%%%%%%%%%%%%%%%%%%%%%%%%%%%%%%%%%%%%%%%%
%
%Let us continue with the construction of Figure~\ref{f.construct} as shown in Figure~\ref{f.focal-chord}. Construct the second vertex $A'$. Construct $EA'$ and label its intersection with $SP$ by $P'$. Construct a perpendicular from $P'$ to the directrix and label its intersection with the directrix by $K'$. Label the intersection of $P'K'$ and $EL$ by $L'$.
%\begin{theorem}
%The point $P'$ is on the ellipse and since $PP'$ intersects $S$ it is a focal chord.
%\end{theorem}
%\begin{proof}
%$\triangle SEA' \sim \triangle L'EP'$ and $\triangle XES\sim K'EL'$ are adjacent similar triangles, so
%\[
%\frac{P'L'}{P'K'}=\frac{SA'}{A'X}\;.
%\]
%By alternate interior angles and vertical angles we have $\angle SL'P'=\angle L'SP'$ so triangle $\triangle SL'P'$ is isosceles and $P'L'=SP'$. Therefore,
%\[
%\frac{SP'}{P'K'}=\frac{SA'}{A'X}=e\,,
%\]
%and $P'$ is a point on the ellipse.\hqed
%\end{proof}

%%%%%%%%%%%%%%%%%%%%%%%%%%%%%%%%%%%%%%%%%%%%%

\begin{figure}[t]
\begin{center}
\begin{tikzpicture}[scale=.9]

\clip (-1,-3.2) rectangle +(10.1,6.5);

% Construct ellipse
\draw[dashed,name path=ellipse] (5,0) ellipse[x radius=4,y radius={sqrt(8)}];

% Compute locations of P, P' on the ellipse 
\def\x{.5}
\def\y{sqrt((16-\x*\x)/2)}
\coordinate (P) at ({5+\x},{\y});
\node[above] at (P) {$P$};
\def\xp{-3.5}
\def\yp{-sqrt((16-\xp*\xp)/2)}
\coordinate (PP) at ({5+\xp},{\yp});
\node[right] at (PP) {$P'$};

% Construct focus and directrix
\coordinate (X) at (0,0);
\node[left] at (X) {$X$};
\path[name path=directrix] ($(X)+(0,3.3)$) -- ($(X)+(0,-3.3)$);
\coordinate (S) at ({5-sqrt(8)},0);
\node[right] at (S) {$S$};

% Locate F as extension of PP' to the directrix
\path[name path=PPP] (P) -- ($(P)!1.5!(PP)$);
\path[name intersections = {of = PPP and directrix, by = {F} }];
\node[left] at (F) {$F$};
\draw (PP) -- (F);

% Construct perpendiculars through K, K' to the directrix
\draw (P) -- (0,{\y}) coordinate (K);
\node[left] at (K) {$K$};
\draw[rotate=-90] (K) rectangle +(6pt,6pt);
\draw (PP) -- (0,{\yp}) coordinate (KP);
\node[left] at (KP) {$K'$};
\draw[rotate=-90] (KP) rectangle +(6pt,6pt);

% Draw triangles
\draw (S) -- (X) -- (F) -- (K);
\draw[thick,blue] (F) -- (S);
\draw[thick,red] (P) -- (S) -- (PP) -- cycle;
\draw[thick,red,dashed] (P) -- ($(P)!2!(S)$);

% Label angles
\node[xshift=20pt,yshift=-26pt] at (S) {$\beta$};
\node[xshift=-40pt,yshift=-26pt] at (S) {$\beta$};
\draw[->] ($(S)+(16pt,-26pt)$) -- +(180:32pt);
\draw[->] ($(S)+(-36pt,-26pt)$) -- +(0:14pt);

\end{tikzpicture}
\end{center}
\caption{Bisecting the angle at the focus}\label{f.bisect-angle}
\end{figure}

%%%%%%%%%%%%%%%%%%%%%%%%%%%%%%%%%%%%%%%%%%%%%%%%%%%%%

\begin{figure}[b]
\begin{center}
\begin{tikzpicture}[scale=1.1]

\clip (-1,-2) rectangle +(10,5.7);

% Construct the directrix and the focus
\coordinate (X) at (0,0);
\node[left] at (X) {$X$};
\path[name path=directrix] ($(X)+(0,4)$) -- ($(X)+(0,-2.2)$);
\coordinate (S) at (3,0);
\node[below right] at (S) {$S$};

% Locate the vertices
\coordinate (A) at (2,0);
\node[above left] at (A) {$A$};
\coordinate (AP) at (8,0);
\node[right] at (AP) {$A'$};

% Find a convenient E and construct paths through A, A', S
\path[name path=findE] (S) -- +(210:4);
\path[name intersections = {of = findE and directrix, by = {E} }];
\node[left] at (E) {$E$};
\path[name path=EA] (E) -- ($(E)!2.3!(A)$);
\path[name path=EAP] (E) -- ($(E)!1.1!(AP)$);
\path[name path=ES] (E) -- ($(E)!2!(S)$);

% Locate P, P' and draw EP
\path[name path=SP] (S) -- +(60:2.7) -- +(240:2.2);
\path[name intersections = {of = EA and SP, by = {P} }];
\node[above right] at (P) {$P$};
\path[name intersections = {of = EAP and SP, by = {PP} }];
\node[below] at (PP) {$P'$};
\vertexsmcolor{PP}{red};
\draw (E) -- (P);

% Locate F and draw lines
\path[name path=APF] (AP) -- ($(AP)!2.1!(P)$);
\path[name intersections = {of = APF and directrix, by = {F} }];
\draw (AP) -- (F) -- (E);
\node[left] at (F) {$F$};

% Draw blue angle and red triangle
\draw[name path=SF,thick,blue] (E) -- (S) -- (F);
\draw[red,thick] (P) -- (S) -- (AP) -- cycle;
\draw[red,thick,dashed] (PP) -- (S) -- (X);

% Label angles
\node[above,xshift=0pt,yshift=3pt] at (S) {$\gamma$};
\node[above left,xshift=-6pt,yshift=-1pt] at (S) {$\gamma$};
\node[below left,xshift=-16pt,yshift=1pt] at (S) {$\delta$};
\node[below,xshift=-16pt,yshift=-10pt] at (S) {$\delta$};

\draw[thick,blue,rotate=124] (S) rectangle +(5pt,5pt);

\end{tikzpicture}
\end{center}
\caption{The right angle at the focus}\label{f.right-angle}
\end{figure}

%%%%%%%%%%%%%%%%%%%%%%%%%%%%%%%%%%%%%%%%%%%%%%%%%%%%%%%%%%

\newpage

\subsection{A right angle at the focus of an ellipse}

\begin{theorem}\label{thm.bisect}
Let $P,P'$ be points on the ellipse and let $F$ be the intersection of $PP'$ with the directrix. Then $FS$ bisects the exterior angle of $\angle P'SP$(Figure~\ref{f.bisect-angle}).
\end{theorem}
\begin{proof}
Since $P,P'$ are on the ellipse 
\[
\frac{SP}{PK}=\frac{SP'}{P'K'}=e\,,
\]
and since $\triangle PFK\sim P'FK'$,
\[
\frac{SP}{SP'}=\frac{PK}{P'K'}=\frac{PF}{P'F}\,.
\]
By the exterior angle bisector theorem (Theorem~\ref{thm.exterior-angle-bisector}), $FS$ bisects the exterior angle of $\angle P'SP$.\hqed
\end{proof}

\begin{theorem}\label{thm.right-angle}
Let $P$ be a point on the ellipse and construct lines $PA,PA'$. Label their intersections with the directrix by $E$ and $F$, respectively. Then $\angle FSE$ is a right angle (Figure~\ref{f.right-angle}).
\end{theorem}

\begin{proof}
$P,A,A'$ are all points on the ellipse so Theorem~\ref{thm.bisect} applies. $FS$ bisects $\angle PSX=2\gamma$ and $ES$ bisects $\angle P'SX=2\delta$, so $2\gamma + 2\delta= 180^\circ$ and $\angle FSE=\gamma + \delta= 90^\circ$.\hqed
\end{proof}

%%%%%%%%%%%%%%%%%%%%%%%%%%%%%%%%%%%%%%%%%%%%%%%%%%%%%%%%%%%%%%%%

\subsection{Ratios of perpendiculars to the axes}

This theorem proves Theorem~\ref{thm.ratios} using Euclidean geometry.

\begin{theorem}\label{thm.ratios-besant}
Let $P$ be a point on an ellipse not on the major axis and construct perpendiculars $PN,PM$ from $P$ to the major and minor axes, respectively (Figure~\ref{f.besant-ratios}). Then
\begin{eqnlabels}
\frac{PN^2}{A'N\cdot NA}&=&\frac{BC^2}{AC^2} = \frac{b^2}{a^2}\label{eqn.pnan}\\[6pt]
\frac{PM^2}{B'N\cdot NA}&=&\frac{AC^2}{BC^2} = \frac{a^2}{b^2}\label{eqn.pmbn}\,.
\end{eqnlabels}
\end{theorem}

%%%%%%%%%%%%%%%%%%%%%%%%%%%%%%%%%%%%%%%%%%%%%%%%%%%%%%%%%%%%%%%%

\begin{proof} (Equation \ref{eqn.pnan})
$\triangle AXE\sim \triangle ANP$ since they are right triangles and the vertical angles at $A$ are equal (red). Therefore,
\begin{equation}
\frac{PN}{AN}=\frac{EX}{AX}\,.\label{eqn.triangle1}
\end{equation}%
$\triangle PA'N\sim \triangle FA'X$ (blue) so
\begin{equation}
\frac{PN}{A'N}=\frac{FX}{A'X}\,.\label{eqn.triangle2}
\end{equation}%
Multiplying Equations~\ref{eqn.triangle1} and \ref{eqn.triangle2} gives
\[
\frac{PN^2}{AN\cdot A'N}=\frac{EX\cdot FX}{AX\cdot A'X}\,.
\]
By Theorem~\ref{thm.right-angle} $\triangle FSE$ is a right triangle so by Theorem~\ref{thm.alt-hypo},
\[
\frac{PN^2}{AN\cdot A'N}=\frac{SX^2}{AX\cdot A'X}\,.
\]
Since $P$ was arbitrary this holds for any point on the ellipse, in particular, for $B$ on the minor axis, where $PN=BC$ and $AN=AN'=AC$. Therefore,
\begin{eqn}
\frac{BC^2}{AC^2}&=&\frac{SX^2}{AX\cdot A'X}\\[6pt]
\frac{PN^2}{AN\cdot A'N}&=&\frac{SX^2}{AX\cdot A'X}=\frac{BC^2}{AC^2}=\frac{b^2}{a^2}\,.\fqed
\end{eqn}%
\end{proof}

%%%%%%%%%%%%%%%%%%%%%%%%%%%%%%%%%%%%%%%%%%%%%%%%%%%%%%%%%%%%%%%%

\begin{proof} (Equation \ref{eqn.pmbn})
Since $CM=PN, PM=CN$, by Theorem~\ref{thm.dividing}, Equation~\ref{eqn.pnan} becomes 
\begin{eqnlabels}
\frac{CM^2}{AC^2-PM^2}&=&\frac{BC^2}{AC^2}\nonumber\\[6pt]
\frac{AC^2}{AC^2-PM^2}&=&\frac{BC^2}{CM^2}\label{eqn.acbc1}\,.
\end{eqnlabels}%
It follows that
\begin{equation}
\frac{AC^2}{PM^2}=\frac{BC^2}{BC^2-CM^2},\label{eqn.acbc2}
\end{equation}

By cross-multiplying the Equations~\ref{eqn.acbc1} and \ref{eqn.acbc2}. By Theorem~\ref{thm.dividing}, Equation~\ref{eqn.acbc2} implies
\[
\frac{PM^2}{BM\cdot MB'}=\frac{AC^2}{BC^2}\,.\fqed
\]
\end{proof}

%%%%%%%%%%%%%%%%%%%%%%%%%%%%%%%%%%%%%%%%%%%%%%%%%%%%%%%%%%%%%%%%

\begin{figure}
\begin{center}
\begin{tikzpicture}[scale=.9]

\clip (-7,-2.5) rectangle +(12,7);

% Draw an hemi-ellipse with center C
\coordinate (C) at (0,0);
\node[below right] at (C) {$C$};
\def\a{4.33}
\def\b{3}
\draw[name path=ellipse] (\a,0) 
  arc[start angle=0,end angle=180, x radius=\a,y radius=\b];

% Locate a focus
\coordinate (S) at ({-sqrt(\a*\a-\b*\b)},0);

% Draw axes
\coordinate (L) at +(180:{\a} and {\b});
\coordinate (R) at +(0:{\a} and {\b});
\node[below] at (R) {$A'$};
\node[above left] at (L) {$A$};
\coordinate (Top) at +(90:{\a} and {\b});
\node[above] at (Top) {$B$};
\draw[name path=major] (L) -- (R);
\draw[name path=minor] (C) -- (Top);
\draw[thick,dashed] (C) -- +(0,{-\b/3}) node[below] {$B'$};

% Construct directrix and focus
\coordinate (X) at (-6,0);
\node[left] at (X) {$X$};
\node[below right] at (S) {$S$};
\path[name path=directrix] ($(X)+(0,4.5)$) -- ($(X)+(0,-2.5)$);

% Locate P and E and draw lines
\path[name path=fromL] (L) -- +(50:4) -- +(230:3);
\path [name intersections = {of = ellipse and fromL, by = {P} }];
\node[above] at (P) {$P$};
\path [name intersections = {of = directrix and fromL, by = {E} }];
\node[left] at (E) {$E$};
\draw (P) -- (E);

% Locate F and draw lines
\path[name path=fromR] (R) -- ($(R)!1.7!(P)$);
\path [name intersections = {of = directrix and fromR, by = {F} }];
\node[left] at (F) {$F$};
\draw (R) -- (F);
\draw (F) -- (S) -- (E) -- cycle;
\draw[rotate=125] (S) rectangle +(7pt,7pt);

% Construct perpendiculars from P
\path (P) -- ($(L)!(P)!(C)$) coordinate (N) node[below] {$N$};
\draw (P) -- ($(Top)!(P)!(C)$) coordinate (M) node[right] {$M$};
\draw[rotate=180] (M) rectangle +(7pt,7pt);
\draw[rotate=90] (N) rectangle +(7pt,7pt);

% Draw red and blue similar triangles
\draw[thick,red] (X) -- (E) -- (P) -- (N) -- cycle;
\draw[thick,blue] ($(R)+(-4pt,2pt)$) -- ($(X)+(0,2pt)$) --
  (F) -- ($(F)!.99!(R)$);
\draw[thick,blue] ($(P)+(2pt,0)$) -- ($(P)!.98!($(N)+(2pt,0)$)$);

\end{tikzpicture}
\end{center}
\caption{Ratio of an ordinate}\label{f.besant-ratios}
\end{figure}

%%%%%%%%%%%%%%%%%%%%%%%%%%%%%%%%%%%%%%%%%%%%%%%%%%%%%%%%%%%%%%%%

\begin{figure}[b]
\begin{center}
\begin{tikzpicture}[scale=.75]
\def\a{4.33}
\def\b{3}

% Draw a hemi-ellipse and a circumscribing hemi-circle
\coordinate (O) at (0,0);
\node[below] at (O) {$C$};
\draw[name path=circle] (\a,0)
  arc[start angle=0,end angle=180,radius=\a];
\draw[name path=ellipse] (\a,0)
  arc[start angle=0,end angle=180, x radius=\a,y radius=\b];

% Draw axes through O
\coordinate (L) at (-\a,0);
\coordinate (R) at (\a,0);
\coordinate (T) at (0,\a);
\draw[name path=major] (L) node[below] {$A$} -- (O) -- 
  (R) node[below] {$A'$};
\draw[name path=minor] (O) -- (T);
\path [name intersections = {of = minor and ellipse, by = {B} }];
\node[above right] at (B) {$B$};

% Choose arbitrary point on major-axis, raise a perpendicular
%   and label its intersections with the ellipse and the circle
\coordinate (N) at (-2,0);
\path[name path=atN] (N) -- ($(N)+(0,\a)$);
\path [name intersections = {of = atN and circle, by = {Q} }];
\draw (N) -- (Q);

% Label X and the intersections
\node[below] at (N) {$N$};
\node[above] at (Q) {$Q$};
\path [name intersections = {of = atN and ellipse, by = {P} }];
\node[above right] at (P) {$P$};
\draw (N) rectangle +(7pt,7pt);
\draw (C) rectangle +(7pt,7pt);

\end{tikzpicture}
\caption{A circle circumscribing an ellipse}\label{f.ellipse-circle-besant}
\end{center}
\end{figure}

%%%%%%%%%%%%%%%%%%%%%%%%%%%%%%%%%%%%%%%%%%%%%%%%%%%%%%%%%%%%%%%%

\subsection{A circle circumscribing an ellipse}

This theorem proves Theorem~\ref{thm.ellipse-b-over-a} in Euclidean geometry.

Consider a circle of radius $a$ with the same center as an ellipse (Figure~\ref{f.ellipse-circle-besant}). Choose a point $N$ on the major axis and construct a perpendicular through $N$. Let its intersections with the ellipse and the circle be $P$ and $Q$, respectively.
\begin{theorem}\label{thm.ellipse-b-over-a-besant}
\[
\frac{PN}{QN}=\frac{BC}{AC}=\frac{b}{a}\,.
\]
\end{theorem}
\begin{proof} 
From Theorem~\ref{thm.ratios-besant},
\[
\frac{PN^2}{AN\cdot NA'} = \frac{BC^2}{AC^2}\,,
\]
and by Theorem~\ref{thm.alt-hypo}, $AN\cdot NA=QN^2$.\hqed
\end{proof}

%%%%%%%%%%%%%%%%%%%%%%%%%%%%%%%%%%%%%%%%%%%%%%%%%%%%%%%%%%%%%%%%

\begin{figure}[t]
\begin{center}
\begin{tikzpicture}[scale=.75]

% Size and center of the ellipse
\def\a{4.33}
\def\b{3}

% Draw an ellipse with center C
\coordinate (C) at (0,0);
\node[below] at (C) {$C$};
\draw[name path=ellipse] (0,0) ellipse[x radius=\a, y radius=\b];

% The Sun is at a focus
\coordinate (S) at ({-sqrt(\a*\a-\b*\b)},0);
\node[below right] at (S) {$S$};

% Draw axes
\coordinate (L) at (-\a,0);
\coordinate (R) at (\a,0);
\coordinate (B) at (0,-\b);
\coordinate (T) at (0,\b);
\node[above] at (T) {$B$};
\node[left] at (L) {$A$};
\node[right] at (R) {$A'$};
\draw (S) -- node[left] {$a$} (T);
\path (C) -- node[above] {$a$} (R);
\draw[name path=major] (L) -- (R);
\draw[name path=minor] (T) -- node[right] {$b$} (C);

% Locate intersections of LR with the ellipse and draw it
\path[name path=atS] (S) -- ($(S)+(0,3)$) -- ($(S)+(0,-3)$);
\path [name intersections = {of = atS and ellipse, by = {L,LP} }];
\node[above,yshift=2pt] at (L) {$L_1$};
\node[below] at (LP) {$L_2$};
\draw[thick,red] (L) -- (LP);

\end{tikzpicture}
\caption{The latus rectum of an ellipse}\label{f.ellipse-latus-rectum-besant}
\end{center}
\end{figure}

%%%%%%%%%%%%%%%%%%%%%%%%%%%%%%%%%%%%%%%%%%%%%%%%%%%%%%%%%%%%%%%%

\subsection{The latus rectum of an ellipse}

This theorem proves Theorem~\ref{thm.ellipse-lr} in Euclidean geometry.

\begin{theorem}\label{thm.ellipse-lr-besant}
$L$, the length of the latus rectum of an ellipse (Definition~\ref{def.ellipse-lr}), is 
$\displaystyle\frac{2b^2}{a}$.
\end{theorem}
\begin{proof}
By Theorem~\ref{thm.ratios-besant},
\[
\frac{SL_1^2}{AS\cdot SA'}=\frac{BC^2}{AC^2}\,.
\]
By Theorem 7.2, $SB=AC=a$, so by Pythagoras's theorem,
\[
BC^2=BS^2-SC^2=AC^2-SC^2=(AC-SC)(AC+SC)=AS\cdot SA'\,.
\]
Therefore, the length of one-half the latus rectum is
\begin{eqn}
SL_1^2&=&\frac{BC^4}{AC^2}\\[6pt]
SL_1&=&\frac{BC^2}{AC}=\frac{b^2}{a}\,.\fqed
\end{eqn}
\end{proof}

%%%%%%%%%%%%%%%%%%%%%%%%%%%%%%%%%%%%%%%%%%%%%%%%%%%%%%%%%%%%%%%%

\subsection{Areas of parallelograms}

\begin{theorem}\label{thm.perp-tangent}
Let $Y$ be the intersection the perpendicular through the focus $S$ to the tangent $TT'$ at $P$, and let $L$ be the intersection of $S'P$ and $SY$ (Figure~\ref{f.perp-focus-tangent}). Then $Y$ is on the circumscribing circle and $CY\parallel S'L$.
\end{theorem}

\begin{proof}

$\triangle STY\sim\triangle T'TC$ since they are right triangles that share the acute angle $\angle CTT'=\angle YTS$, so $\angle CT'T=\angle YST=\beta$.

By Theorem~\ref{thm.tangent-angles}, $\angle SPY=\angle S'PT'=\alpha$ since they are the angles to the foci at the tangent. $\angle S'PT'=\angle YPL=\alpha$ are vertical angles, so $\angle SPY=\angle LPY=\alpha$. 

Then $\triangle SPY\cong\triangle LPY$ since they are right triangles with an equal acute angle and a common side $PY$. Therefore, $PL=PS$ and $S'L=S'P+PL=AA'=2a$.

Since $\triangle SPY\cong\triangle LPY$, $SY=YL$, and since $S,S'$ are foci, $S'C=SC$. It follows that $\triangle CSY\sim \triangle S'SL$ and $CY\parallel S'L$. By similarity,
\[
\frac{CY}{S'L}=\frac{CS}{S'S}=\frac{CS}{2CS}\,.
\]
Therefore, $2CY=S'L=2a$ so $CY=a$ and $Y$ is on the circumscribing circle of radius $a$.\hqed
\end{proof}

%%%%%%%%%%%%%%%%%%%%%%%%%%%%%%%%%%%%%%%%%%%%%%%%%%%%%%

\begin{figure}[t]
\begin{center}
\begin{tikzpicture}

% Size and center of the ellipse
\def\a{4.33}
\def\b{3}
\coordinate (C) at (0,0);
\node[below] at (C) {$C$};

% Draw the ellipse and the circle
\draw[name path=ellipse] (\a,0) 
  arc[start angle=0,end angle=180,x radius=\a,y radius=\b];
\draw[very thick,dotted] (\a,0) 
  arc[start angle=0,end angle=180,radius=\a];

% Locate the foci
\coordinate (F1) at ({-sqrt(\a*\a-\b*\b)},0);
\node[below] at (F1) {$S'$};
\coordinate (F2) at ({+sqrt(\a*\a-\b*\b)},0);
\node[below] at (F2) {$S$};

% Draw axes AA' and BB'
\coordinate (L) at +(180:{\a} and {\b});
\coordinate (R) at +(0:{\a} and {\b});
\coordinate (Top) at +(90:{\a} and {\b});
\node[below] at (L) {$A'$};
\node[below] at (R) {$A$};
\draw[name path=major] (L) -- (R);
\draw (Top) -- (C);

% Select an arbitrary point P on the ellipse
\path[name path=fromF1p] (F1) -- +(25:8);
\path [name intersections = {of = ellipse and fromF1p, by = {P} }];
\node[above,xshift=0pt,yshift=0pt] at (P) {$P$};

% Draw tangent at P and locate its intersections with the axes
\tkzDefLine[bisector out](F1,P,F2) \tkzGetPoint{Tan1}
\path[name path=t1] ($(Tan1)!-.5!(P)$) coordinate (Z) -- 
  ($(Tan1)!1.75!(P)$) coordinate (TP);
\path[name path=ct] (C) -- ($(C)!2!(R)$);
\path[name path=bt] (C) -- ($(C)!1.5!(Top)$);
\path [name intersections = {of = t1 and bt, by = {TP} }];
\path [name intersections = {of = t1 and ct, by = {T} }];
\node[below] at (T) {$T$};
\node[above] at (TP) {$T'$};
\draw (C) -- (T) -- (TP) -- cycle;

% Construct the perpendicular through S to the tangent at Y
\draw (F2) -- ($(T)!(F2)!(TP)$) coordinate (Y);
\node[above] at (Y) {$Y$};
\draw[thick,dashed] (C) -- (Y);

% Construct lines through P and Y that meet in L
\path[name path=spp] (F1) -- ($(F1)!1.5!(P)$);
\path[name path=sy] (F2) -- ($(F2)!2.3!(Y)$);
\path [name intersections = {of = spp and sy, by = {LL} }];
\node[above left] at (LL) {$L$};
\draw (F1) -- (P) -- node[above] {$d$} (LL) -- 
  (F2) -- node[near start,left] {$d$} (P);

% Lable angles
\node[left,xshift=-3pt] at (P) {\sm{\alpha}};
\node[right,xshift=3pt] at (P) {\sm{\alpha}};
\node[below right,xshift=2pt,yshift=-2pt] at (P) {\sm{\alpha}};
\node[below right,xshift=2pt,yshift=-2pt] at (TP) {\sm{\beta}};
\node[above right,xshift=2pt,yshift=-2pt] at (F2) {\sm{\beta}};

\draw[rotate=-115]  (Y) rectangle +(7pt,7pt);
\draw  (C) rectangle +(7pt,7pt);

\draw[thick,red] ($(F2)+(0,2pt)$) -- ($(T)+(-10pt,2pt)$);
\draw[thick,red] ($(Y)+(0,-2pt)$) -- ($(T)+(-9pt,2pt)$);
\draw[thick,red] (Y) -- (F2);
\draw[thick,blue] (C) -- (TP) -- (T) -- cycle;

\end{tikzpicture}
\caption{The perpendicular from a focus to a tangent}\label{f.perp-focus-tangent}
\end{center}
\end{figure}

%%%%%%%%%%%%%%%%%%%%%%%%%%%%%%%%%%%%%%%%%%%%%%%%%%%%%%%%%%%%%%

\begin{theorem}\label{thm.cnntacac}
Let $N$ be the intersection of the perpendicular through $P$ to the major axis (Figure~\ref{f.segments-major}). Then $CN\cdot NT = AC^2=AN\cdot NA'$.
\end{theorem}

\begin{figure}[t]
\begin{center}
\begin{tikzpicture}

% Size and center of the ellipse
\def\a{4.33}
\def\b{3}
\coordinate (C) at (0,0);
\node[below] at (C) {$C$};

% Draw the ellipse
\draw[name path=ellipse] (\a,0) 
  arc[start angle=0,end angle=180, x radius=\a,y radius=\b];

% Locate the foci
\coordinate (F1) at ({-sqrt(\a*\a-\b*\b)},0);
\node[below] at (F1) {$S'$};
\coordinate (F2) at ({+sqrt(\a*\a-\b*\b)},0);
\node[below] at (F2) {$S$};

% Draw axes AA' and BB'
\coordinate (L) at +(180:{\a} and {\b});
\coordinate (R) at +(0:{\a} and {\b});
\coordinate (Top) at +(90:{\a} and {\b});
\node[below] at (L) {$A'$};
\node[below] at (R) {$A$};
\draw[name path=major] (L) -- (R);
\draw (Top) -- (C);

% Select an arbitrary point P on the ellipse
\path[name path=fromF1p] (F1) -- +(25:8);
\path [name intersections = {of = ellipse and fromF1p, by = {P} }];
\node[above,xshift=0pt,yshift=0pt] at (P) {$P$};
\draw (F1) -- (P) -- (F2);

% Draw tangent at P and find its intersections with the axes
\tkzDefLine[bisector out](F1,P,F2) \tkzGetPoint{Tan1}
\path[name path=t1] ($(Tan1)!-.5!(P)$) coordinate (Z) -- 
  ($(Tan1)!1.75!(P)$) coordinate (TP);
\path[name path=ct] (C) -- ($(C)!2!(R)$);
\path[name path=bt] (C) -- ($(C)!1.5!(Top)$);
\path [name intersections = {of = t1 and bt, by = {TP} }];
\path [name intersections = {of = t1 and ct, by = {T} }];
\node[below] at (T) {$T$};
\node[above] at (TP) {$T'$};
\draw (C) -- (T) -- (TP) -- cycle;

% Locate perpendicular through focus to the tangent at Y
\draw (F2) -- ($(T)!(F2)!(TP)$) coordinate (Y);
\node[above] at (Y) {$Y$};

% Locate perpendicular through P to the major axis
\draw (P) --  ($(L)!(P)!(R)$) coordinate (N);
\node[below] at (N) {$N$};
\draw (N) -- (Y) -- (C);

% Label the angles
\node[left,xshift=-3pt] at (Y) {\sm{\alpha}};
\node[left,xshift=-3pt] at (P) {\sm{\alpha}};
\node[below right,xshift=1pt,yshift=-2pt] at (P) {\sm{\alpha}};
\node[above right,xshift=4pt,yshift=1pt] at (N) {\sm{\alpha}};

% Draw the triangles
\draw[thick,red] ($(Y)+(-1pt,-2pt)$) -- 
  ($(C)+(8pt,2pt)$) -- ($(T)+(-4pt,2pt)$) -- (Y);
\draw[thick,blue] (C) -- (N) -- (Y) -- cycle;

\draw[rotate=-115]  (Y) rectangle +(7pt,7pt);
\draw[rotate=90]  (N) rectangle +(7pt,7pt);

\end{tikzpicture}
\caption{Ratios of segments of the major axis}\label{f.segments-major}
\end{center}
\end{figure}

%%%%%%%%%%%%%%%%%%%%%%%%%%%%%%%%%%%%%%%%%%%%%%%%%%

\begin{proof}
Continuing with the construction from Figure~\ref{f.perp-focus-tangent}, we focus on the segments $AN,NA'$ (Figure~\ref{f.segments-major}). We can deduce the angles that are shown.
\begin{itemize}
\item $CY\parallel S'P$ (Theorem~\ref{thm.perp-tangent}) so $\angle CYP = \angle S'PT'$ by corresponding angles.
\item $\angle S'PT' = \angle SPY$ by Theorem~\ref{thm.tangent-angles} since they are the angles to the foci at the tangent.
\item $\angle SYP$ and $\angle SNP$ are right angles and therefore $SYPN$ is quadrilateral that can be circumscribed by a circle whose diameter is $PS$.\footnote{It can be proven that a quadrilateral whose opposite angle are supplementary can be circumscribed by a circle. If two opposite angles are right angles that sum to $180^\circ$, then the other two angles must also sum to $180^\circ$.}
Therefore, $\angle SPY = \angle SNY$ since they are subtended by the same chord $YS$.
\end{itemize}
Since $\angle CYT\sim \angle CNY$ and $\angle 	YCT$ is a common angle,
$\triangle CYT\sim \triangle CNY$ and
\begin{eqnlabels}
\frac{CN}{CY}&=&\frac{CY}{CT}\nonumber\\[6pt]
CN\cdot CT&=&CY^2=AC^2\,,\label{eqn.cnctacac}
\end{eqnlabels}
since the perpendicular to the tangent from a focus is on the circumscribing circle (Theorem~\ref{thm.perp-tangent}). Since $NT=CT-CN$, we have $CN\cdot NT = CN\cdot CT - CN^2$ which equals $AC^2-CN^2$ by Equation~\ref{eqn.cnctacac}. This in turn equals $AN\cdot NA'$ by Theorem~\ref{thm.dividing}.\hqed
\end{proof}

%%%%%%%%%%%%%%%%%%%%%%%%%%%%%%%%%%%%%%%%%%%%%%%%%%%%%%%%%%%%%%

\begin{figure}[t]
\begin{center}
\begin{tikzpicture}[scale=.9]

% Size and center of the ellipse
\def\a{4.33}
\def\b{3}
\coordinate (C) at (0,0);
\node[below left,xshift=2pt] at (C) {$C$};

% Draw the ellipse
\draw[name path=ellipse] (C) ellipse[x radius=\a,y radius= \b];

% Locate the foci
\coordinate (F1) at ({-sqrt(\a*\a-\b*\b)},0);
\coordinate (F2) at ({+sqrt(\a*\a-\b*\b)},0);

% Draw axes
\coordinate (L) at +(180:{\a} and {\b});
\coordinate (R) at +(0:{\a} and {\b});
\coordinate (Top) at +(90:{\a} and {\b});
\coordinate (Bot) at +(-90:{\a} and {\b});
\draw[name path=major] (L) -- 
  (R) node[below right] {$A$};
\draw[name path=minor] (Bot) --
  (Top) node[above left] {$B$};

% Select an arbitrary point P on the ellipse
\path[name path=fromF1p] (F1) -- +(20:8);
\path [name intersections = {of = ellipse and fromF1p, by = {P} }];
\path[name path=ph] (P) -- (F2);
\path[name path=pc] (P) -- (C);
\node[above,xshift=0pt,yshift=0pt] at (P) {$P$};

% Draw tangent at P and locate its intersections with the minor axis
\tkzDefLine[bisector out](F1,P,F2) \tkzGetPoint{Tan1}
\path[name path=t1] ($(Tan1)!.1!(P)$) coordinate (Z) -- 
  ($(Tan1)!1.75!(P)$) coordinate (TP);
\path[name path=bt] (C) -- ($(C)!1.5!(Top)$);
\path [name intersections = {of = t1 and bt, by = {T} }];
\node[left] at (T) {$T$};

% Conjugate diameters from D through C
\path[name path=cd] (C) -- +($(P)-(Z)$);
\path[name path=dk] (C) -- +($(Z)-(P)$);
\path [name intersections = {of = cd and ellipse, by = {D} }];
\path [name intersections = {of = dk and ellipse, by = {K} }];
\node[above left,xshift=3pt,yshift=-2pt] at (D) {$D$};
\node[below right,xshift=0pt,yshift=0pt] at (K) {$K$};
\draw (D) -- (K);

% Perpendicular through P on major axis and 
%   intersection with the conjugate diameter
\draw (P) -- ($(C)!(P)!(R)$) coordinate (N);
\node[below right] at (N) {$N$};
\path[name path=pk] (P) -- ($(P)!2!(N)$);
\path [name intersections = {of = pk and dk, by = {J} }];
\node[below left,yshift=4pt] at (J) {$J$};
\draw (N) -- (J);

% Locate normal to the tangent and its 
%   intersection with the conjugate diameter and the major axis
\draw[name path=pf] (P) -- ($(D)!(P)!(K)$) coordinate (F);
\node[below left] at (F) {$F$};
\path [name intersections = {of = pf and major, by = {G} }];
\node[above left] at (G) {$G$};

% Find intersection of tangent with the major axis
\path[name path=semimajor] (C) -- ($(C)!1.6!(R)$);
\path [name intersections = {of = t1 and semimajor, by = {TP} }];
\node[below] at (TP) {$T'$};
\draw (Top) -- (T) -- (TP) -- (R);

% Draw colored triangles and quadrilateral
\path[fill,black!10!white] (N) -- (G) -- (F) -- (J) -- cycle;
\draw[thick,red] (P) -- (N) -- (G) -- cycle;
\draw[thick,red] (C) -- (G) -- (F) -- cycle;
\draw[thick,blue] (C) -- (T) -- ($(TP)+(-3pt,2pt)$);
\draw[thick,blue] ($(TP)+(-3pt,2pt)$) -- ($(C)+(0,2pt)$);
\draw[thick,blue] ($(P)+(2pt,0)$) -- ($(N)+(2pt,0)$);

% Label angles
\node[above right,xshift=2pt,yshift=-1pt] at (G) {\sm{\alpha}};
\node[below left,xshift=-1pt,yshift=1pt] at (G) {\sm{\alpha}};
\node[below right,xshift=-4pt,yshift=0pt] at (G) {\sm{180^\circ\!-\!\alpha}};
\node[below right,xshift=10pt,yshift=3pt] at (C) {\sm{\beta}};
\node[above left,xshift=0pt,yshift=4pt] at (C) {\sm{\alpha}};
\node[below,xshift=-4pt,yshift=-6pt] at (P) {\sm{\beta}};
\node[below right,xshift=-2pt,yshift=-4pt] at (T) {\sm{\alpha}};
\node[above left,xshift=2pt,yshift=2pt] at (J) {\sm{\alpha}};

\draw[rotate=-32]  (F) rectangle +(7pt,7pt);
\draw[rotate=180]   (N) rectangle +(7pt,7pt);
\draw[rotate=-212] (P) rectangle +(7pt,7pt);
\draw[rotate=0]    (C) rectangle +(7pt,7pt);

\end{tikzpicture}
\caption{Parallelograms formed by conjugate diameters}\label{f.parallelogram-conjugate-diameters}
\end{center}
\end{figure}

%%%%%%%%%%%%%%%%%%%%%%%%%%%%%%%%%%%%%%%%%%%%%%%%%%%%%%%%%%%%%%

Construct the normal to the tangent at $P$ and let its intersection with the conjugate diameter $DK$ be $F$ and its intersection with the major axis be $G$. Construct a perpendicular from $P$ to the major axis and let its intersection be $N$. Let the intersection of the tangent with the minor axis be $T$ and its intersection with the major axis be $T'$ (Figure~\ref{f.parallelogram-conjugate-diameters}).
\begin{theorem}\label{thm.parallelogram1}
\[
PF\cdot PG = BC^2\,.
\]
\end{theorem}

%%%%%%%%%%%%%%%%%%%%%%%%%%%%%%%%%%%%%%%%%%%%%%%%%%%%%%%%%%%%%%

\begin{proof}
$\triangle NPG \sim \triangle FPJ$ (rotate $\triangle NPG$ to see this) so $\angle PGN=\angle PJF = \alpha$ and
\begin{eqnlabels}
\frac{PF}{PN}&=&\frac{PJ}{PG}\nonumber\\[6pt]
PF\cdot PG &=& PJ\cdot PN\,.\label{eqn.pnpj}
\end{eqnlabels}%
By vertical angles $\angle PGN = \angle CGF = \alpha$ so 
$\triangle NPG\sim \triangle FCG$ and $\angle NPG = \angle FCG = \beta =90^\circ-\alpha$. By adding $\beta$ to the right angles $\angle BCN$ and $\angle TPF$, we get that $\angle TCJ = \angle TPJ$ and therefore $TPJC$ is a parallelogram, so $CT=PJ$ and $PF\cdot PG = CT\cdot PN$. By Equation~\ref{eqn.pnpj}, the theorem will be proven if we can show that $CT\cdot PN=BC^2$.

$\triangle TT'C\sim \triangle PT'N$ so 
\begin{eqn}
\frac{CT}{CT'}&=&\frac{PN}{NT'}\\[6pt]
\frac{CT}{PN}&=&\frac{CT'}{NT'}\,.
\end{eqn}

Multiplying each side by fractions equal to $1$ gives
\[
\frac{CT\cdot PN}{PN^2}=\frac{CT'\cdot CN}{CN \cdot NT'}\,.
\]
By Equation~\ref{eqn.cnctacac}, $CN\cdot CT' = AC^2$, and by Theorem~\ref{thm.cnntacac}, $CN\cdot NT'=AN\cdot NA'$, so
\[
\frac{CT\cdot PN}{PN^2}=\frac{AC^2}{AN\cdot NA'}\,.
\]
Multiplying by $PN^2/PN^2$ and using Theorem~\ref{thm.ratios-besant} gives
\begin{eqnlabels}
\frac{CT\cdot PN}{PN^2}&=&\frac{PN^2}{AN\cdot NA'}\cdot \frac{AC^2}{PN^2}=\frac{BC^2}{AC^2}\cdot \frac{AC^2}{PN^2}\nonumber\\[6pt]
CT\cdot PN&=&BC^2\,.\label{eqn.ctpn}\fqed
\end{eqnlabels}
\end{proof}

%%%%%%%%%%%%%%%%%%%%%%%%%%%%%%%%%%%%%%%%%%%%%%%%%%%%%%

\newpage

\begin{theorem}\label{thm.cmpnacbc}
In Figure~\ref{f.parallelogram3},
\begin{eqn}
CN^2&=&AM\cdot MA',\quad CM^2=AN\cdot NA'\\[6pt]
\frac{DM}{CN}&=&\frac{BC}{AC}\\[6pt]
\frac{CM}{PN}&=&\frac{BC}{AC}
\end{eqn}%
\end{theorem}

%%%%%%%%%%%%%%%%%%%%%%%%%%%%%%%%%%%%%%%%%%%%%%%%%%%%%%%%%%%%%%%%%%%%%%

\begin{figure}[t]
\begin{center}
\begin{tikzpicture}[scale=.8]

\clip (-8,-4) rectangle +(14,8);

% Size and center of the ellipse
\def\a{4.33}
\def\b{3}
\coordinate (C) at (0,0);
\node[below left,xshift=2pt,yshift=-4pt] at (C) {$C$};

% Draw the ellipse
\draw[name path=ellipse] (C) ellipse[x radius=\a,y radius= \b];

% Locate the foci
\coordinate (F1) at ({-sqrt(\a*\a-\b*\b)},0);
\coordinate (F2) at ({+sqrt(\a*\a-\b*\b)},0);

% Draw axes
\coordinate (L) at +(180:{\a} and {\b});
\coordinate (R) at +(0:{\a} and {\b});
\coordinate (Top) at +(90:{\a} and {\b});
\coordinate (Bot) at +(-90:{\a} and {\b});
\draw[name path=major] (L) node[below left] {$A'$} --
  (R) node[below right] {$A$};
\draw[name path=minor] (Bot) node[below] {$B'$} --
  (Top) node[above] {$B$};

% Select an arbitrary point P on the ellipse
\path[name path=fromF1p] (F1) -- +(15:9);
\path [name intersections = {of = ellipse and fromF1p, by = {P} }];
\path[name path=ph] (P) -- (F2);
\draw[name path=pc] (P) -- (C);
\node[above right] at (P) {$P$};

% Draw tangent at P and locate intersection
%  with the extension of the major axis
\path[name path=tans] ($(L)+(-4,0)$) -- ($(R)+(2,0)$);
\tkzDefLine[bisector out](F1,P,F2) \tkzGetPoint{Tan1}
\path[name path=t1] (P) -- ($(Tan1)!.2!(P)$) coordinate (Z);
\path [name intersections = {of = t1 and tans, by = {T} }];
\node[below] at (T) {$T$};
\draw (P) -- (T);

% Diameter from P through C
\draw[name path=pc] (P) -- ($(P)!2!(C)$) coordinate (G);
\node[below left] at (G) {$P'$};

% Conjugate diameter from D through C
\path[name path=cd] (C) -- +($(P)-(Z)$);
\path[name path=dk] (C) -- +($(Z)-(P)$);
\path [name intersections = {of = cd and ellipse, by = {D} }];
\path [name intersections = {of = dk and ellipse, by = {K} }];
\node[above left] at (D) {$D$};
\node[below right] at (K) {$K$};
\draw (D) -- (K);

% Draw tangent at D and locate intersection with the major axis
\tkzDefLine[bisector out](F1,D,F2) \tkzGetPoint{Tan3}
\path[name path=t3] (D) --  ($(Tan3)!4.3!(D)$);
\path [name intersections = {of = tans and t3, by = {TP} }];
\draw (D) -- (TP) node[below] {$T'$};
\draw (T) -- (TP);

% Construct perpendiculars from P, D to the major axis
\draw (P) -- ($(L)!(P)!(R)$) coordinate (N) node[below] {$N$};
\draw (D) -- ($(L)!(D)!(R)$) coordinate (MM) node[below] {$M$};

\draw[rotate=90]   (N) rectangle +(8pt,8pt);
\draw[rotate=0]    (MM) rectangle +(8pt,8pt);

\end{tikzpicture}
\caption{Ratios of perpendiculars to the major axis}\label{f.parallelogram3}
\end{center}
\end{figure}

%%%%%%%%%%%%%%%%%%%%%%%%%%%%%%%%%%%%%%%%%%%%%%%%%%%%%%

\begin{proof}
By Theorem~\ref{thm.cnntacac},
\begin{eqn}
CN\cdot CT &=& AC^2=CM\cdot CT'\\[6pt]
\frac{CM}{CN} &=& \frac{CT}{CT'}\,.
\end{eqn}%
Since $DK$ and $PP'$ are conjugate diameters, $DT'\parallel PP'$ and $\triangle T'DC \sim \triangle CPT$, so
\begin{eqn}
\frac{CM}{CN} &=& \frac{CT}{CT'} =  \frac{CN}{MT'}\\[6pt]
CN^2 &=& CM\cdot MT'\,,
\end{eqn}%
Therefore,
\begin{equation}
CN^2=CM\cdot MT'=AC^2 = AM\cdot MA'\label{eqn.cnannap}
\end{equation}%
by Theorem~\ref{thm.cnntacac}. By Theorem~\ref{thm.ratios-besant},
\[
\frac{DM^2}{AM\cdot MA'}=\frac{BC^2}{AC^2}\,,
\]
and by Equation~\ref{eqn.cnannap},
\begin{eqn}
\frac{DM^2}{CN^2}&=&\frac{BC^2}{AC^2}\\[6pt]
\frac{DM}{CN}&=&\frac{BC}{AC}\,,
\end{eqn}%
A symmetric argument shows that
\begin{eqn}
CM^2&=& AN\cdot NA'\\[6pt]
\frac{CM}{PN}&=&\frac{BC}{AC}\,\fqed.
\end{eqn}%
\end{proof}

%%%%%%%%%%%%%%%%%%%%%%%%%%%%%%%%%%%%%%%%%%%%%%%%%%%%%%%%%%%%%%%%%%%%%%

\begin{figure}[b]
\begin{center}
\begin{tikzpicture}

% Size and center of the ellipse
\def\a{4.33}
\def\b{3}
\coordinate (C) at (0,0);
\node[below left,xshift=2pt,yshift=-4pt] at (C) {$C$};

% Draw the ellipse
\draw[name path=ellipse] (C) ellipse[x radius=\a,y radius= \b];

% Locate the foci
\coordinate (F1) at ({-sqrt(\a*\a-\b*\b)},0);
\coordinate (F2) at ({+sqrt(\a*\a-\b*\b)},0);

% Draw axes AA' and BB'
\coordinate (L) at +(180:{\a} and {\b});
\coordinate (R) at +(0:{\a} and {\b});
\coordinate (Top) at +(90:{\a} and {\b});
\coordinate (Bot) at +(-90:{\a} and {\b});
\draw[name path=major] (L) node[left] {$A'$} --
  (R) node[right] {$A$};
\draw[name path=minor] (Bot) node[below] {$B'$} --
  (Top) node[above] {$B$};

% Select an arbitrary point P on the ellipse
\path[name path=fromF1p] (F1) -- +(20:8);
\path [name intersections = {of = ellipse and fromF1p, by = {P} }];
\path[name path=ph] (P) -- (F2);
\path[name path=pc] (P) -- (C);
\node[above,xshift=0pt,yshift=0pt] at (P) {$P$};

% Draw tangent at P
\tkzDefLine[bisector out](F1,P,F2) \tkzGetPoint{Tan1}
\path[name path=t1] ($(Tan1)!.15!(P)$) -- ($(Tan1)!2!(P)$);

% Conjugate diameter from P through C
\draw[name path=pc] (P) -- ($(P)!2!(C)$) coordinate (G);
\node[below left] at (G) {$P'$};

% Find a point Z on tangent for drawing the tangent
\path[name path=f1q] (F1) -- +(60:6);
\path [name intersections = {of = f1q and t1, by = {Z} }];

% Conjugate diameters from D through C
\path[name path=cd] (C) -- +($(Z)-(P)$);
\path[name path=dk] (C) -- +($(P)-(Z)$);
\path [name intersections = {of = cd and ellipse, by = {D} }];
\path [name intersections = {of = dk and ellipse, by = {K} }];
\node[above left,xshift=1pt,yshift=-5pt] at (D) {$D$};
\node[below right,xshift=0pt,yshift=0pt] at (K) {$K$};
\draw (D) -- (K);

% Draw tangent at G (continuation of PC)
\tkzDefLine[bisector out](F2,G,F1) \tkzGetPoint{Tan2}
\path[name path=t2] ($(Tan2)!.1!(G)$) -- ($(Tan2)!1.9!(G)$);

% Draw tangent at D
\tkzDefLine[bisector out](F1,D,F2) \tkzGetPoint{Tan3}
\path[name path=t3] ($(Tan3)!-1.7!(D)$) -- ($(Tan3)!3.6!(D)$);

% Draw tangent at K (continuation of DC)
\tkzDefLine[bisector out](F1,K,F2) \tkzGetPoint{Tan4}
\path[name path=t4] ($(Tan4)!.2!(K)$)  -- ($(Tan4)!1.8!(K)$);

% Draw large dashed rectangle
\draw[thick,dashed] (\a,\b) -- (-\a,\b) -- 
  (-\a,-\b) -- (\a,-\b) -- cycle;

% Get the intersections of the tangents
\path [name intersections = {of = t1 and t3, by = {J} }];
\path [name intersections = {of = t2 and t3, by = {KK} }];
\path [name intersections = {of = t1 and t4, by = {M} }];
\path [name intersections = {of = t2 and t4, by = {LL} }];

% Draw the parallelogram
\draw (J) node[above] {$L$} -- (KK) -- (LL) -- (M) -- cycle;

% Locate perpendiculars of P, D with major axes
\draw (P) -- ($(L)!(P)!(R)$) coordinate (N) node[below] {$N$};
\draw (D) -- ($(L)!(D)!(R)$) coordinate (M) node[below] {$M$};

% Locate normal from P to the conjugate diameter
%   and its intersection with the major axes
\draw[name path=pf] (P) -- ($(D)!(P)!(K)$) coordinate (F);
\node[below left] at (F) {$F$};
\path [name intersections = {of = pf and major, by = {GG} }];
\node[above left] at (GG) {$G$};

% Draw colored triangles
\draw[thick,red]  (C) -- (M) --  (D) -- cycle;
\draw[thick,red] (GG) -- (N) --  (P) -- cycle;
\draw[thick,red] (C) -- (F) --  (GG) -- cycle;

\draw[rotate=-32]  (F) rectangle +(7pt,7pt);
\draw[rotate=90]   (N) rectangle +(7pt,7pt);
\draw[rotate=-212] (P) rectangle +(7pt,7pt);
\draw[rotate=0]    (M) rectangle +(7pt,7pt);

% Draw angles
\node[below right,xshift=0pt,yshift=-6pt] at (D) {$\alpha'$};
\node[above left,xshift=-12pt,yshift=-1pt] at (C)  {$\alpha$};

\node[below right,xshift=10pt,yshift=2pt] at (C) {$\alpha$};
\node[below left,xshift=-4pt,yshift=2pt] at (GG) {$\alpha'$};

\node[above right,xshift=4pt,yshift=0pt] at (GG) {$\alpha'$};
\node[below left,xshift=3pt,yshift=-9pt] at (P) {$\alpha'$};

\end{tikzpicture}
\caption{Areas of parallelograms ($\alpha'=90^\circ\!-\!\alpha$)}\label{f.parallelogram2}
\end{center}
\end{figure}

%%%%%%%%%%%%%%%%%%%%%%%%%%%%%%%%%%%%%%%%%%%%%%%%%%%%%%%%%%%%%%%%%%%%%%

This theorem proves Theorem~\ref{thm.conj-diam-para} in Euclidean geometry.

\begin{theorem}\label{thm.area-parallelogram}
The area of the parallelogram formed by the tangents at the ends of the conjugate diameters $PP',DK$ is equal to the area of the rectangle enclosing the ellipse at the ends of the axes (Figure~\ref{f.parallelogram2}).
\end{theorem}

\begin{proof}
By the definition of conjugate diameters, it is sufficient to show that the area of $PCDL$ is $AC\cdot BC$. The area of a parallelogram is width times height so it is $CD\cdot PF$. 

By vertical angles $\angle DCM = \angle GCF$ and $\angle CGF = \angle PGN$, so $\triangle DCM\sim \triangle PGN$ 
\[
\frac{PG}{CD}=\frac{PN}{CM}\,,
\]
and by Theorem~\ref{thm.cmpnacbc}
\begin{eqnlabels}
\frac{PG}{CD}&=&\frac{AC}{BC}\nonumber\\[6pt]
\frac{CD}{AC}&=&\frac{PG}{BC}\,.\label{eqn.cdac}
\end{eqnlabels}
By Theorem~\ref{thm.parallelogram1},
\begin{equation}
\frac{PG}{BC}=\frac{BC}{PF}\,.\label{eqn.pgbc}
\end{equation}%
From Equations~\ref{eqn.cdac} and~\ref{eqn.pgbc} gives $CD\cdot PF = AC \cdot BC$.\hqed
\end{proof}

