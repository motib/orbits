% !TeX root = orbits.tex

\section*{Further reading}

Sections~\ref{s.aristarchus}--\ref{s.newton} based primarily on Hahn's book \cite{hahn-cic}. He has written a more advanced book on the orbits of planets and spacecraft \cite{hahn-orbits}. Section~\ref{s.centripetal} is based on \cite{griffiths}. Two additional articles present this aspect of Newton's work \cite{hauser-lang,stein}. Should you wish to read Newton's work, \cite{newton-cohen} is an up-to-date translation into English and Cohen's lengthy Guide will enable you to follow Newton's often terse presentation. The relevant sections of the Guide are $10.8$--$10.10$. Section~\ref{s.ellipse} on ellipses is based on \cite{wiki-ellipse}, while Section~\ref{s.geometry} is based on\cite{besant}. 

\bibliographystyle{plain}
\bibliography{orbits}
