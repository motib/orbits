% !TeX root = orbits.tex
% !TeX Program=pdfLaTeX

\chapter*{Sources and further reading}

\addcontentsline{toc}{chapter}{\large Sources and further reading}

Sections~\ref{s.aristarchus}--\ref{s.newton} are primarily on Hahn's book \cite{hahn-cic}. He has written a more advanced book on the orbits of planets and spacecraft \cite{hahn-orbits}. The proof in Chapter~\ref{s.centripetal} is from \cite{griffiths}. Two additional articles present this aspect of Newton's work \cite{hauser-lang,stein}. A modern translation of Newton's \emph{Principia} is \cite{newton-cohen} and Cohen's lengthy \textit{Guide} will facilitate understanding Newton's often terse presentation. The relevant sections of the Guide are $10.8$--$10.10$. Chapter~\ref{s.geometry} is based on Besant's textbook \cite{besant}. Drew wrote a similar shorter textbook \cite{drew}. Compare these textbooks with \cite{smith} that uses analytic geometry.\footnote{The dates given for these books are for the editions that I used.} The computations of Chapter~\ref{s.lagrange} are from \cite{stern}.

\begin{small}
\bibliographystyle{plainnat}
\bibliography{orbits}
\end{small}
