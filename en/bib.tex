% !TeX root = orbits.tex

\section*{Sources and further reading}

\addcontentsline{toc}{section}{\large Sources and further reading}

Sections~\ref{s.aristarchus}--\ref{s.newton} are primarily on Hahn's book \cite{hahn-cic}. He has written a more advanced book on the orbits of planets and spacecraft \cite{hahn-orbits}. The proof in Section~\ref{s.centripetal} is from \cite{griffiths}. Two additional articles present this aspect of Newton's work \cite{hauser-lang,stein}. Should you wish to read Newton's work, \cite{newton-cohen} is an up-to-date translation into English, and Cohen's lengthy \textit{Guide} will facilitate understanding Newton's often terse presentation. The relevant sections of the Guide are $10.8$--$10.10$. The Wikipedia entry on ellipses is very comprehensive. 

Section~\ref{s.geometry} is based on Besant's textbook that uses Euclidean geometry exclusively to prove theorems on conic sections \cite{besant}. Drew wrote a similar textbook, though shorter and more elementary \cite{drew}. Compare these textbooks one by Smith that uses analytic geometry \cite{smith}.\footnote{The dates given for all three books are for the editions that I used.}

The computations of Section~\ref{s.lagrange} are from \cite{stern}.
\begin{small}
\bibliographystyle{plain}
\bibliography{orbits}
\end{small}
