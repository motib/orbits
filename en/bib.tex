% !TeX root = orbits.tex
% !TeX Program=pdfLaTeX

\chapter*{Sources and further reading}

\addcontentsline{toc}{chapter}{\large Sources and further reading}

Sections~\ref{s.aristarchus}--\ref{s.newton} are based primarily on Hahn's book \cite{hahn-cic}. He has written a more advanced book on the orbits of planets and spacecraft \cite{hahn-orbits}. The analytic proof in Chapter~\ref{s.centripetal} is from \cite{griffiths}. The presentation of the Euclidean geometry in Chapter~\ref{s.geometry} is based on Besant's textbook \cite{besant}. Drew's (shorter) textbook \cite{drew} is similar, while Smith's textbook \cite{smith} uses analytic geometry. Feynman's proof is found in \cite{lost} and Maxwell's is in \cite[Article CXXXIII]{maxwell}. The computations of the Lagrange points are from \cite{stern}. Other expositions of Newton's work on orbits can be found in \cite{hauser-lang,stein}. 

\begin{small}
\bibliographystyle{plainnat}
\bibliography{orbits}
\end{small}
