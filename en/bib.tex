% !TeX root = orbits.tex

\section*{Further reading}

This document is based primarily on Hahn's book \cite{hahn-cic}. He has written a more advanced book on the orbits of planets and spacecraft \cite{hahn-orbits}. Section~\ref{s.centripetal} is based on \cite{griffiths}. Two additional articles present this aspect of Newton's work \cite{hauser-lang,stein}. Should you wish to read Newton's work, \cite{newton-cohen} is an up-to-date translation into English. Cohen's length Guide will enable you to follow Newton's often-terse presentation. The relevant sections of the Guide are $10.8$--$10.10$. The proofs of the theorems on ellipses is based on \cite{wiki-ellipse}. Currently, there seems to be less interest in the geometry of conic sections; a comprehensive book on the subject is \cite{besant}, although the language and notation can be unfamiliar.

\bibliographystyle{plain}
\bibliography{orbits}
