% !TeX root = orbits.tex
% !TeX Program=pdfLaTeX

\chapter{Theorems of Euclidean Geometry}\label{s.elementary}

\section{Constructing a circle from three points}

\begin{theorem}\label{thm.three-points}
Given three non-collinear points a circle can be constructed that goes through all three points.
\end{theorem}

\begin{proof}
Three non-collinear points $A,B,C$ define a triangle $\triangle ABC$ (Figure~\ref{f.kepler-circumscribed}). Construct the perpendicular bisectors of any two of its three sides, say, $AC$ and $BC$. By definition the perpendicular bisector is the locus of points equidistant from the endpoints of the segment. Let $O$ be the intersection of the two bisectors. Then the $AO=CO=BO$ is the radius of a circle centered at $O$ that goes through $A,B,C$.\hqed
\end{proof}

\begin{figure}[b]
\begin{minipage}{.48\textwidth}
\begin{center}
\begin{tikzpicture}[scale=.8]
  \clip (-1,-1.8) rectangle +(7,6.5);
  % Define the coordinates of an arbitrary triangle,
  %   label the vertices and draw dots at each vertex
  \coordinate[label = left:$A$]  (A) at ( 0, 0);
  \coordinate[label = right:$B$] (B) at (5, 0);
  \coordinate[label = above:$C$] (C) at (3.5 ,4);
  %\vertexsm{A};
  %\vertexsm{B};
  %\vertexsm{C};
  
  % Draw the triangle
  \draw (A) -- (B) -- (C) -- cycle;

  % Construct perpendicular bisectors
  \coordinate (AC) at ($(A)!.5!(C)$);
  \draw[name path=ac] ($(AC)!1.5cm!-90:(A)$) -- ($(AC)!2.5cm!90:(A)$);
  \coordinate (BC) at ($(B)!.5!(C)$);
  \draw[name path=bc] ($(BC)!3cm!-90:(B)$) -- ($(BC)!.5cm!90:(B)$);

  % Find the intersection
  \path [name intersections = {of = bc and ac, by = {M}}];
  %\vertexsm{M};
  \node[above,xshift=2pt,yshift=0pt] at (M) {$O$};

  % Construct a circumscribed circle
  \node[draw] at (M) [circle through = (A)] {};

  % Draw the right-angle squares
  \draw[rotate=-40] (AC) rectangle +(7pt,7pt);
  \draw[rotate=200] (BC) rectangle +(7pt,7pt);

\end{tikzpicture}
\caption{A circle through three points}\label{f.kepler-circumscribed}
\end{center}
\end{minipage}
\begin{minipage}{.50\textwidth}
\begin{center}
\begin{tikzpicture}[scale=.8]
  \clip (-4.6,-1.8) rectangle +(9.2,6.5);
\def\a{4.33}
\def\b{3}

% Draw a hemi-ellipse and a circumscribing hemi-circle
\coordinate (O) at (0,0);
\draw[name path=circle] (\a,0) arc(0:180:\a);

% Draw axes through O
\coordinate (L) at (-\a,0);
\coordinate (R) at (\a,0);
\draw[name path=major] (L) node[below] {$A$} -- (O) -- (R) node[below] {$A'$};

% Choose arbitrary point on major-axis, raise a perpendicular
%   and label its intersections with the ellipse and the circle
\coordinate (N) at (-2,0);
\path[name path=atN] (N) -- ($(N)+(0,\a)$);
\path [name intersections = {of = atN and circle, by = {Q} }];
\draw (N) -- (Q);
\draw (L) -- (Q) -- (R);

\draw[rotate=-120] (Q) rectangle +(8pt,8pt);
\draw (N) rectangle +(8pt,8pt);

% Label X and the intersections
\node[below] at (N) {$N$};
\node[above] at (Q) {$Q$};

\node[above right,xshift=6pt] at (L) {$90^\circ\!-\!\alpha$};
\node[below right,yshift=-16pt] at (Q) {$90^\circ\!-\!\alpha$};
\node[above left,xshift=-14pt] at (R) {$\alpha$};
\node[below left,yshift=-18pt] at (Q) {$\alpha$};

\end{tikzpicture}
\caption{Right triangle in a circle}\label{f.circle-besant}
\end{center}
\end{minipage}
\end{figure}

%%%%%%%%%%%%%%%%%%%%%%%%%%%%%%%%%%%%%%%%%%%%%%%%%%%%%%%%%%%

\begin{theorem}\label{thm.alt-hypo}
Let $Q$ be a point on a circle whose diameter is $AA'$ and construct a perpendicular $QN$ to the diameter (Figure~\ref{f.circle-besant}). Then
\[
QN^2=AN\cdot NA'\,.
\]
The equation also holds if \emph{it is given} that $\triangle AQA'$ is a right triangle.
\end{theorem}

\begin{proof}
An angle that subtends a diameter is a right angle. Since the sum of the angles of a triangle is $180^\circ$, we can label the angles as shown in the Figure, from which follows that $\triangle QNA \sim\triangle A'NQ$. Therefore,
\[
\frac{QN}{AN} = \frac{NA'}{QN}\,.\fqed
\]
\end{proof}

%%%%%%%%%%%%%%%%%%%%%%%%%%%%%%%%%%%%%%%%%%%%%%%%%%%%%%%%%%%%%%%%

\begin{figure}[b]
\begin{center}
\begin{tikzpicture}[scale=.8]
\clip (-1,-2.5) rectangle +(8.7,5.5);

% Construct a vertical line and locate three points
\coordinate (A) at (0,-2);
\node[left] at (A) {$A$};
\coordinate (E) at (0,0);
\node[left] at (E) {$E$};
\coordinate (B) at (0,2.4);
\node[left] at (B) {$B$};
\draw[thick,blue] (A) -- (B);

% Locate C and D
\coordinate (C) at (4,2.4);
\node[above] at (C) {$C$};
\coordinate (D) at (7,2.4);
\node[right] at (D) {$D$};
\draw[thick,blue] (B) -- ($(C)+(-3pt,0)$) coordinate (CX);
\draw[thick,red] (C) -- (D);

% Construct AC, AD
\draw[thick,red,name path=ad] (A) -- (D);
\draw[thick,blue,name path=ac] ($(A)+(0pt,3pt)$) -- ($(C)+(-3pt,0)$);
\draw[thick,red] (C) -- (A);

% Construct lines from E
\path[name path=eg] (E) -- ($(E)+(4,0)$);
\path [name intersections = {of = eg and ac, by = {F} }];
\path [name intersections = {of = eg and ad, by = {G} }];

% Label intersections
\node[above left,xshift=3pt] at (F) {$F$};
\node[right,xshift=4pt] at (G) {$G$};
\draw[thick,blue] (E) -- (F);
\draw[thick,red] ($(F)+(3pt,0)$) -- (G);

\end{tikzpicture}
\end{center}
\caption{Adjacent pairs of similar triangles}\label{f.adjacent}
\end{figure}

%%%%%%%%%%%%%%%%%%%%%%%%%%%%%%%%%%%%%%%%%%%%%%%%%%%%%%%%%%%%%%%%%%%%%%

\section{Adjacent pairs of similar triangles}

\begin{definition}
An \emph{adjacent pair of similar triangles} is a pair of similar triangles that share sides. In Figure~\ref{f.adjacent}, $\triangle BAC\sim \triangle EAF$ and $\triangle CAD\sim \triangle FAG$ are an adjacent pair of similar triangles.
\end{definition}
\begin{theorem}
For the adjacent pair of similar triangles in Figure~\ref{f.adjacent},
\[
\frac{AB}{AE}=\frac{AD}{AG}\,.
\]
\end{theorem}
\begin{proof} By similar triangles,
\[
\frac{AB}{AE}=\frac{AC}{AF}=\frac{AD}{AG}\,.\fqed
\]%
\end{proof}
Similar ratios hold between other sides of $\triangle BAC$ and $\triangle CAD$ by using an intermediate step with $AC$. We will use the term \emph{by an adjacent pair of similar triangles} and leave it to the reader to make the intermediate step.

\section{The angle bisector theorems}

\begin{theorem}[Interior angle bisector theorem]
\label{thm.interior-angle-bisector}
In $\triangle ABC$ let $D$ be a point on $BC$ (Figure~\ref{f.interior-angle-bisector}). Then $AD$ bisects $\angle CAB$ \emph{if and only if}
\[
\frac {BD}{CD}=\frac {AB}{AC}\,.
\]
\end{theorem}
%%%%%%%%%%%%%%%%%%%%%%%%%%%%%%%%%%%%%%%%%%%%%%%%%%%%%%%%%%%%%%%%%%%%%%

\begin{figure}[t]
\begin{center}
\begin{tikzpicture}[scale=.8]
\clip (-.6,-2.5) rectangle +(9.2,6);

% Draw base and paths of two lines at known angles
\draw (0,0) coordinate (b) node[left] {$B$} -- 
  (8,0) coordinate (c) node[right] {$C$};
\path[name path=ba] (b) -- +(30:8);
\path[name path=ca] (c) -- +(140:6);

% Get their intersection and draw lines between vertices
\path[name intersections={of=ba and ca,by=a}];
\node[above] at (a) {$A$};
\draw[name path=bc] (b) -- (c);
\draw (b) -- (a) -- (c);

% Draw bisector
\path[name path=bisector] (a) -- +(-95:5.5);
\path[name intersections={of= bc and bisector,by=d}];
\node[below left] at (d) {$D$};

% Extend bisector and construct a line parallel to AB
\path[name path=ce] (c) -- +(210:5);
\path[name intersections={of= bisector and ce,by=e}];
\node[left] at (e) {$E$};
\draw[thick,dashed] (c) -- (e);
\draw (a) -- (e);

% Label angles
\node[below left,xshift=-2pt,yshift=-8pt] at (a) {$\alpha$};
\node[below right,xshift=0pt,yshift=-8pt] at (a) {$\alpha$};
\node[above right,xshift=0pt,yshift=6pt] at (e) {$\alpha$};
\node[above left,xshift=0pt,yshift=0pt] at (d) {$\beta$};
\node[below right,xshift=0pt,yshift=0pt] at (d) {$\beta$};

\end{tikzpicture}
\end{center}
\caption{The interior angle bisector theorem}
\label{f.interior-angle-bisector}
\end{figure}

%%%%%%%%%%%%%%%%%%%%%%%%%%%%%%%%%%%%%%%%%%%%%%%%%%%%%%%%%%%%%%%%%%%%%%

\begin{proof}
Suppose that $AD$ bisects $\angle BAC$. Construct a line through $C$ parallel to $AB$ and let its intersection with $AD$ be $E$. By alternate interior angles, $\angle BAD=\angle CED=$ and by vertical angles $\angle BDA=\angle CDE$. Therefore, $\triangle ABD\sim \triangle EDC$ so
\[
\frac{BD}{CD}=\frac{AB}{CE}\,.
\]
$\triangle ECA$ is isosceles so $CE=AC$ and
\[
\frac{BD}{CD}=\frac{AB}{AC}\,.
\]
To prove the converse just ``run'' the proof backwards.\hqed
\end{proof}

%%%%%%%%%%%%%%%%%%%%%%%%%%%%%%%%%%%%%%%%%%%%%%%%%%%%%%%%%%%%%%%%%%%%%%

\begin{theorem}[Exterior angle bisector theorem]
\label{thm.exterior-angle-bisector}
In $\triangle ABC$ let $D$ be a point on the extension of $CB$ (Figure~\ref{f.exterior-angle-bisector}). Then $AD$ bisects the exterior angle of $\angle BAC$ if and only if
\[
\frac {BD}{CD}=\frac {AB}{AC}\,.
\]
\end{theorem}

%%%%%%%%%%%%%%%%%%%%%%%%%%%%%%%%%%%%%%%%%%%%%%%%%%%%%%%%%%%%%%%%%%%%%%

\begin{figure}[b]
\begin{center}
\begin{tikzpicture}[scale=1.2]
\clip (-2.5,-.8) rectangle +(11,3);

% Draw base and paths of two lines at known angles
\path (0,0) coordinate (b) node[below] {$B$} -- 
  (8,0) coordinate (c) node[below] {$C$};
\path[name path=ba] (b) -- +(50:3);
\path[name path=ca] (c) -- +(170:7.5);

% Get their intersection and draw lines between vertices
\path[name intersections={of=ba and ca,by=a}];
\node[above] at (a) {$A$};
\path[name path=bc] (c) -- ($(c)!1.5!(b)$);
\draw[red,thick] (b) -- (a) -- (c) -- cycle;

% Extend AC to F
\path (c) -- ($(c)!1.4!(a)$) coordinate (f);
\draw (a) -- (f) node[above] {$F$};

% Draw bisector to D and extend BC to intersect it
\path[name path=bisector] (a) -- +(200:5.5);
\path[name intersections={of= bc and bisector,by=d}];
\node[below] at (d) {$D$};
\draw[thick,red,dashed] (b) -- (d);
\draw[thick,blue] (a) -- (d);

% Construct parallel to AD from B and label intersection with AC by E
\path[name path=be] (b) -- +(20:4);
\path[name intersections={of= be and ca,by=e}];
\node[above] at (e) {$E$};
\draw[thick,dashed] (b) -- (e);

% Label angles
\node[left,xshift=-12pt,yshift=-2pt] at (a) {$\alpha$};
\node[below left,xshift=-10pt,yshift=-6pt] at (a) {$\alpha$};
\node[above right,xshift=10pt,yshift=6pt] at (b) {$\alpha$};
\node[left,xshift=-10pt,yshift=-2pt] at (e) {$\alpha$};

\end{tikzpicture}
\end{center}
\caption{The exterior angle bisector theorem}
\label{f.exterior-angle-bisector}
\end{figure}

%%%%%%%%%%%%%%%%%%%%%%%%%%%%%%%%%%%%%%%%%%%%%%%%%%%%%%%%%%%%%%%%%%%%%%

\begin{proof}
Suppose that $AD$ bisects $\angle BAF$. Construct a line through $B$ parallel to $AD$ and let its intersection with $AC$ be $E$. By alternate interior angles $\angle BAD=\angle ABE$ and by corresponding angles $\angle FAD=\angle AEB$. Therefore, $\triangle BCE\sim \triangle DCA$ so
\[
\frac{BD}{CD}=\frac{AE}{AC}\,.
\]
But $\triangle BAE$ is isosceles so $AE=AB$ and
\[
\frac{BD}{CD}=\frac{AB}{AC}\,.
\]
To prove the converse just ``run'' the proof backwards.\hqed
\end{proof}

\medskip

\begin{center}
\fbox{\parbox{.8\textwidth}{The exterior angle bisector theorem can be confusing to understand in a proof, because it can be hard to identify the components of a diagram. In the text the following color-coding is used: the triangle is red, the extension of one side is dashed red and the bisector is blue.}}
\end{center}

