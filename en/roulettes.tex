% !TeX root = orbits.tex
% !TeX Program=pdfLaTeX

\chapter{Ellipses as routlettes and glissettes}\label{s.roulettes}

There are many methods and tools for drawing ellipses.\footnote{You can find animations on Wikipedia and YouTube.} Modern presentations of these methods use the parametric representation of an ellipse (Definition~\ref{def.parametric}). Following Chapter~X of Besant's \emph{Conic Sections} \cite{besant}, we show three such methods, all described using Euclidean geometry only. 

Section~\ref{s.roulette} presents a roulette called the \emph{Tusi couple}. Section~\ref{s.glissette} presents a glissette called the \emph{Trammel of Archimedes}, while Section~\ref{s.triangle-glissette} presents a glissette based on a triangle.\footnote{I have not encountered this construction elsewhere.}

\section{A roulette for drawing an ellipse}\label{s.roulette}

A \emph{roulette} is a curve generated by one curve $c_1$ \emph{rolling} on another curve $c_2$. An ellipse can be generated by a circle of radius $r$ rotating around the circumference of a circle of radius $2r$.

%%%%%%%%%%%%%%%%%%%%%%%%%%%%%%%%%%%%%%%%%%%%%%%%%%%%%%%%%%%%%%%%

\begin{figure}
\begin{center}
\begin{tikzpicture}[scale=.8]
\clip (-7.5,-1) rectangle +(15,8.5);

\def\a{7}
\def\angle{60}

\coordinate (O) at (0,0);
\coordinate (AP) at ({-\a},0);
\coordinate (A) at ({\a},0);
\draw[name path=diam] (AP) -- (A);
\draw (AP) arc[start angle=180,end angle=0,radius=\a];
\node at (130:{\a+.3}) {$c_2$};

\draw[name path=smdiam] (O) -- ({\angle}:{\a}) coordinate (E);
\coordinate (C) at +({\angle}:{\a/2});
\draw[name path=circle] (C) circle[radius={\a/2}];
\path [name intersections = {of = diam and circle, by = {dummy,Q} }];
\node at ($(C)+(135:{\a/2+.3})$) {$c_1$};

\draw (C) -- (Q) -- (E);
\coordinate (P) at ($(C)!.6!(Q)$);

\draw (P) -- ($(O)!(P)!(A)$) coordinate (N);
\path[name path=rn] (N) -- ($(N)!4!(P)$);
\path [name intersections = {of = smdiam and rn, by = {R} }];
\draw (N) -- (R);

\path[name path=prp] (P) -- +(180:3);
\path [name intersections = {of = smdiam and prp, by = {RP} }];
\draw (P) -- (RP);

\draw[rotate=90] (Q) rectangle +(8pt,8pt);
\draw[rotate=90] (N) rectangle +(8pt,8pt);
\draw[rotate=90] (P) rectangle +(8pt,8pt);

\node[below] at (A) {$A$};
\node[below] at (O) {$O$};
\node[below] at (Q) {$Q$};
\node[left] at (C) {$C$};
\node[above right] at (E) {$E$};
\node[above left,xshift=-2pt] at (N) {$N$};
\node[right] at (P) {$P$};
\node[left] at (R) {$R$};
\node[left] at (RP) {$R'$};

\node[right] at (C) {$2\alpha$};
\node[above right,xshift=4pt] at (O) {$\alpha$};

\path (C) -- node[left] {$a$} (R);
\path (C) -- node[left] {$a$} (RP);
\path (C) -- node[below,xshift=-2pt] {$a$} (P);

\end{tikzpicture}
\caption{Constructing an ellipse from a roulette}\label{f.roulette1}
\end{center}
\end{figure}

%%%%%%%%%%%%%%%%%%%%%%%%%%%%%%%%%%%%%%%%%%%%%%%%%%%%%%%%%%%%%%%

\begin{theorem}
Let $P$ be an arbitrary point within with circle $c_1$ of the roulette. Then as $c_1$ rotates within $c_2$, the locus of $P$ is an ellipse (Figure~\ref{f.roulette1}).
\end{theorem}

%%%%%%%%%%%%%%%%%%%%%%%%%%%%%%%%%%%%%%%%%%%%%%%%%%

\begin{figure}
\begin{center}
\begin{tikzpicture}[scale=.85]
\clip (-5.5,-.5) rectangle +(11,5.6);

\def\a{5}
\coordinate (O) at (0,0);
\coordinate (AP) at ({-\a},0);
\coordinate (A) at ({\a},0);
\draw[name path=diam] (AP) -- (A);
\draw (AP) arc[start angle=180,end angle=0,radius=\a];
\foreach \angle in {5,10,...,85,95,100,...,175} {
  \coordinate (R) at +({\angle}:{3*\a/4});
  \coordinate (RP) at +({\angle}:{\a/4});
  \path[name path=rn] (R) -- ($(O)!(R)!(A)$) coordinate (N);
  \path[name path=rp] (RP) -- ($(R)!(RP)!(N)$);
  \path [name intersections = {of = rn and rp, by = {P} }];
  \fill[shift only,blue] (R) circle (1pt);
  \fill[shift only,green!80!black] (RP) circle (1pt);
  \fill[shift only,red] (P) circle (1pt);
}

\coordinate (a) at +(0:{3*\a/4});
\fill[shift only,red] (a) circle (1pt);
\coordinate (a) at +(0:{\a/4});
\fill[shift only,green!80!black] (a) circle (1pt);
\coordinate (a) at +(90:{3*\a/4});
\fill[shift only,blue] (a) circle (1pt);
\coordinate (a) at +(90:{\a/4});
\fill[shift only,red] (a) circle (1pt);
\coordinate (a) at +(180:{\a/4});
\fill[shift only,green!80!black] (a) circle (1pt);
\coordinate (a) at +(180:{3*\a/4});
\fill[shift only,red] (a) circle (1pt);

\end{tikzpicture}
\caption{The loci of $P$ (red), $R$ (blue), $R'$ (green)}\label{f.roulette2}
\end{center}
\end{figure}

%%%%%%%%%%%%%%%%%%%%%%%%%%%%%%%%%%%%%%%%%%%%%

\begin{proof}
Let $C$ be the center of $c_1$ and let $O$ be the center of $c_2$. Let $E$ be an arbitrary point on $c_2$ where it is contacted by $c_1$. The line segment $OE$ of length $2r$ is a chord of $c_1$ and since it equals twice the radius of $c_1$ it is a diameter. $c_1$ will intersect $OA$ at some point $Q$ and since $OE$ is a diameter of $c_1$, $\angle EQO$ is a right angle.\footnote{The diagrams assumes that $\angle EOQ$ is not a multiple of $90^\circ$, but those cases are easy to deal with.}

$\angle ECQ=2\cdot\angle EOQ=2\alpha$ because $\angle ECQ$ is a central angle of $c_1$ subtended by $EQ$ which also subtends the inscribed angle $\angle EOQ$. Therefore the arc $\widehat{EQ}$ equals $2\alpha \cdot r$. But $\angle EOQ$ is the same angle as $EOA$ and is therefore an inscribed angle of $c_2$. It follows that the arc $\widehat{EA}$ equals $\alpha \cdot 2r$ and $\widehat{EA} =\widehat{EQ}$, so as $c_1$ rotates $Q$ is always on the diameter.

Let $P$ be an arbitrary point on $CQ$ and construct $RP\parallel EQ$ and $R'P\parallel OQ$. Let $N$ be the intersection of $RP$ with $OA$. Since $CE=CQ$ are radii of $c_1$, by $RP\parallel EQ$, $\triangle PCR\sim \triangle QCE$ so $\triangle PCR$ is isosceles and $CP=CR=a$. Similarly, $\triangle PCR'$ is isosceles and $CP=CR'=a$.

Now let $c_1$ rotate within $c_2$. Since $P$ is fixed relative to $C$ and $E$ is fixed relative to $O$, $OR=r+a$ and $OR'=r-a$ are constant and their loci are circles. From $R'P\parallel ON$ we get $\triangle RPR'\sim \triangle RNO$ and
\[
\frac{PN}{RN} = \frac{OR'}{OR}=\frac{PQ}{OR}\,,
\]
since $OR'=r-a=PQ$. By Theorem~\ref{thm.ellipse-b-over-a-besant}, the locus of $P$ is an ellipse with $OR$ the semi-major axis and $OR'=PQ$ the semi-minor axis (Figure~\ref{f.roulette2}).\hqed
\end{proof}

%%%%%%%%%%%%%%%%%%%%%%%%%%%%%%%%%%%%%%%%%%%%%%%%%%%%%%%%%%%%%%%
\begin{comment}
\begin{figure}
\begin{center}
\begin{tikzpicture}[scale=.6]
\clip (-7.5,-1.2) rectangle +(15.5,9);

\def\a{7}

\coordinate (O) at (0,0);
\node[below] at (O) {$O$};
\coordinate (AP) at ({-\a},0);
\coordinate (A) at ({\a},0);
\node[below] at (A) {$A$};
\draw[name path=diam,thick,dotted] (AP) -- (A);
\draw[thick,dotted] (AP) arc[start angle=180,end angle=0,radius=\a];
\draw[<->] ($(O)+(10:\a+.6)$) 
  arc[start angle=10,end angle=80,radius=\a+.6]
  node[fill=white,midway] {$E$};

\def\angle{10}
\path[name path=smdiam] (O) -- ({\angle}:{\a}) coordinate (E);
\coordinate (C) at +({\angle}:{\a/2});
\path[name path=circle,thick,dotted] (C) circle[radius={\a/2}];
\path [name intersections = {of = diam and circle, by = {dummy,Q} }];
\draw (O) -- (C) -- (Q) -- (E);
\vertexsmcolor{C}{blue};

\def\angle{20}
\path[name path=smdiam] (O) -- ({\angle}:{\a}) coordinate (E);
\coordinate (C) at +({\angle}:{\a/2});
\path[name path=circle,thick,dotted] (C) circle[radius={\a/2}];
\path [name intersections = {of = diam and circle, by = {dummy,Q} }];
\draw (O) -- (C) -- (Q) -- (E);
\vertexsmcolor{C}{blue};

\def\angle{40}
\path[name path=smdiam] (O) -- ({\angle}:{\a}) coordinate (E);
\coordinate (C) at +({\angle}:{\a/2});
\path[name path=circle,thick,dotted] (C) circle[radius={\a/2}];
\path [name intersections = {of = diam and circle, by = {dummy,Q} }];
\draw (O) -- (C) -- (Q) -- (E);
\vertexsmcolor{C}{blue};

\def\angle{50}
\path[name path=smdiam] (O) -- ({\angle}:{\a}) coordinate (E);
\coordinate (C) at +({\angle}:{\a/2});
\path[name path=circle,thick,dotted] (C) circle[radius={\a/2}];
\path [name intersections = {of = diam and circle, by = {dummy,Q} }];
\draw (O) -- (C) -- (Q) -- (E);
\vertexsmcolor{C}{blue};

\def\angle{60}
\path[name path=smdiam] (O) -- ({\angle}:{\a}) coordinate (E);
\coordinate (C) at +({\angle}:{\a/2});
\path[name path=circle,thick,dotted] (C) circle[radius={\a/2}];
\path [name intersections = {of = diam and circle, by = {dummy,Q} }];
\draw (O) -- (C) -- (Q) -- (E);
\coordinate (P) at ($(C)!.6!(Q)$);
\vertexsmcolor{P}{red};
\vertexsmcolor{C}{blue};

\def\angle{80}
\path[name path=smdiam] (O) -- ({\angle}:{\a}) coordinate (E);
\coordinate (C) at +({\angle}:{\a/2});
\path[name path=circle,thick,dotted] (C) circle[radius={\a/2}];
\path [name intersections = {of = diam and circle, by = {dummy,Q} }];
\draw (O) -- (C) -- (Q) -- (E);
\vertexsmcolor{C}{blue};

\draw[<->] ($(O)+(0,-.8)$) -- node[fill=white] {$Q$} ($(A)+(0,-.8)$);
\end{tikzpicture}
\caption{Center $C$ (blue) rotating around $O$; eventually, $P$ (red) will be on $CQ$}\label{f.roulette3}
\end{center}
\end{figure}
\end{comment}
%%%%%%%%%%%%%%%%%%%%%%%%%%%%%%%%%%%%%%%%%%%%%%%%%%%%%%%%%%%%%%%%%%%%%%%%

\section{A glissette for drawing an ellipse}\label{s.glissette}

 A \emph{glissette} is a curve generated by one curve $c_1$ \emph{sliding} on another curve $c_2$. In Figure~\ref{f.glissette1}, the line segment $AB$ is constrained to move so that $A$ slides on $OA$ and $B$ slides on $OB$ where $OA\perp OB$. Let: $P$ be an arbitrary point on $AB$. $C$ be the bisector of $AB$, $PN$ the perpendicular to $OB$, and $Q$ intersection of $PN$ with $OC$.

%%%%%%%%%%%%%%%%%%%%%%%%%%%%%%%%%%%%%%%%%%%%%%%%%%%%%%%%%%%%%%%%%%%%%%%%

\begin{figure}
\begin{center}
\begin{tikzpicture}
\clip (-.5,-.5) rectangle +(8,5);

\def\b{7}
\def\a{4}

\coordinate (O) at (0,0);
\coordinate (A) at (0,{\a});
\coordinate (B) at ({\b},0);
\draw (A) -- (O) -- (B) -- cycle;
\fill ($(A)+(-1.5pt,-3pt)$) rectangle +(3pt,6pt);
\fill ($(B)+(-3pt,-1.5pt)$) rectangle +(6pt,3pt);

\coordinate (C) at ($(A)!.5!(B)$);
\path[name path=oc] (O) -- ($(O)!1.75!(C)$);

\coordinate (P) at ($(A)!.67!(B)$);
\path (P) -- ($(O)!(P)!(B)$) coordinate (N);

\path[name path=np] (N) --  ($(N)!3!(P)$);
\path [name intersections = {of = oc and np, by = {Q} }];
\draw (N) -- (Q) -- (O);

\node[left] at (A) {$A$};
\node[below] at (B) {$B$};
\node[below] at (O) {$O$};
\node[below left] at (P) {$P$};
\node[above] at (Q) {$Q$};
\node[above,yshift=2pt] at (C) {$C$};
\node[below] at (N) {$N$};
\draw (O) rectangle +(6pt,6pt);
\draw (N) rectangle +(6pt,6pt);
\path (A) -- node[above] {$a$} (C);
\path (O) -- node[below] {$a$} (C);
\path (C) -- node[above] {$b$} (Q);
\path (C) -- node[below] {$b$} (P);
\node[above,xshift=-5pt,yshift=2pt] at (P) {$\alpha$};
\node[below,xshift=5pt,yshift=-2pt] at (P) {$\alpha$};
\node[below,xshift=-5pt,yshift=-2pt] at (Q) {$\alpha$};

\draw[red] (Q) -- (O) -- (N) -- cycle;
\draw[red] (P) -- (N) -- (B) -- cycle;

\fill[shift only,red] (P) circle (2pt);
\fill[shift only,blue] (Q) circle (2pt);

\end{tikzpicture}
\caption{Constructing an ellipse from the glissette $AB$}\label{f.glissette1}
\end{center}
\end{figure}

%%%%%%%%%%%%%%%%%%%%%%%%%%%%%%%%%%%%%%%%%%%%%%%%%%%%%%%%%%%%%%%%%%%%%%%%

\newpage

\begin{theorem}
As $A$ slides on $OA$ and $B$ on $OB$, the locus of $Q$ is a circle and the locus of $P$ is an ellipse.
\end{theorem}
\begin{proof}
Since $OC$ is the median to the hypotenuse of a right triangle, $AC=CB=OC=a$  and therefore $\angle COA=\angle CAO$. $\triangle ACO\sim \triangle PCQ$ since $AO\parallel QP$, and $CP=CQ, OQ=AP$. $P$ is fixed (on $AB$) so $AP=a+b=OQ$ and the locus of $Q$ is a circle.

Since $\triangle PCQ$ is isosceles, $\angle CPQ=\angle CQP=\alpha$ and $\angle CPQ=\angle BPN=\alpha$ by vertical angles. Therefore, $\triangle PBN\sim \triangle QNO$ and
\[
\frac{PN}{QN}=\frac{PB}{OQ}=\frac{PB}{PA}\,.
\]
By Theorem~\ref{thm.ellipse-b-over-a-besant}, the locus of $P$ is an ellipse with $AP$ the semi-major axis and $BP$ the semi-minor axis. Figure~\ref{f.glissette2} shows the loci of $P$ and $Q$ for $AB=10$.\hqed
\end{proof}

%%%%%%%%%%%%%%%%%%%%%%%%%%%%%%%%%%%%%%%%%%%%%%%%%%%%%%%%%%%%%%%%%%%%%%%%

\begin{figure}[t]
\begin{center}
\begin{tikzpicture}[scale=.85]
\clip (-.5,-.4) rectangle +(11,9.3);

\foreach \a in {8.5,8,7.5,7,6.5,6,5.5,5,4.5,4,3.5,3,2.5,2,1.5,1} {
\def\b{sqrt(100-\a*\a)}
\coordinate (O) at (0,0);
\coordinate (A) at (0,{\a});
\coordinate (B) at ({\b},0);
\draw (A) -- (O) --(B);
\draw[thick,dotted] (A) -- (B);
\fill ($(A)+(-1.5pt,-3pt)$) rectangle +(3pt,6pt);
\fill ($(B)+(-3pt,-1.5pt)$) rectangle +(6pt,3pt);

\coordinate (C) at ($(A)!.5!(B)$);
\path[name path=oc] (O) -- ($(O)!1.75!(C)$);

\coordinate (P) at ($(A)!.67!(B)$);
\path (P) -- ($(O)!(P)!(B)$) coordinate (N);

\path[name path=np] (N) --  ($(N)!3!(P)$);
\path [name intersections = {of = oc and np, by = {Q} }];
\draw (N) -- (Q) -- (O);

\fill[shift only,red] (P) circle (2pt);
\fill[shift only,blue] (Q) circle (2pt);
}
\end{tikzpicture}
\caption{The circle ($Q$ blue) and the ellipse ($P$ red)}\label{f.glissette2}
\end{center}
\end{figure}

%%%%%%%%%%%%%%%%%%%%%%%%%%%%%%%%%%%%%%%%%%%%%%%%%%%%%%%%%%%%%%%%%%%%%%%%

\section{A triangle glissette}\label{s.triangle-glissette}

An ellipse can be drawing by sliding a given triangle $\triangle AQB$ along perpendicular lines $OA,OB$ (Figure~\ref{f.triangle1}). Let $OC$ be the median of $AB$ and extend $OC$ so that $CP=CQ$. Since $\triangle AOB$ is a right triangle, $OC=AC=BC$. The locus of $P$ is a circle with radius $OC+CP=OC+CQ$.

%%%%%%%%%%%%%%%%%%%%%%%%%%%%%%%%%%%%%%%%%%%%%%%%%%%%%%%%%%%%%%%%%%%%%%%%

\begin{figure}[b]
\begin{center}
\begin{tikzpicture}[scale=.7]
\clip (-1,-.7) rectangle +(8,9);

\def\b{8}
\def\a{sqrt(100-\b*\b)}
\def\anglea{30}
\def\angleb{15}

\coordinate (O) at (0,0);
\coordinate (A) at ({\a},0);
\coordinate (B) at ({0,\b});

\path[name path=a,blue] (A) -- +({180-atan2(\b,\a)-\anglea}:8);
\path[name path=b,red] (B) -- +({-90+atan2(\a,\b)+\angleb}:10);
\path [name intersections = {of = a and b, by = {Q} }];

\draw (B) -- (A) -- (O) -- (B) -- (Q) -- (A);
\fill ($(B)+(-1.5pt,-3pt)$) rectangle +(3pt,6pt);
\fill ($(A)+(-3pt,-1.5pt)$) rectangle +(6pt,3pt);

\coordinate (C) at ($(A)!.5!(B)$);
\draw (C) -- (Q);
\draw
  let
    \p1 = ($(Q) - (C)$),
    \n1 = {veclen(\x1, \y1)},
    \p2 = ($(O) - (C)$),
    \n2 = {veclen(\x2, \y2)}
  in 
    (O) -- ($(O) ! \n1+\n2 ! (C)$) coordinate (P);

\node[below] at (A) {$A$};
\node[left] at (B) {$B$};
\node[below] at (O) {$O$};
\node[right] at (P) {$P$};
\node[right] at (Q) {$Q$};
\node[left] at (C) {$C$};

\draw[red,thick] (B) -- (A) -- (Q) -- cycle;

\fill[shift only,red] (P) circle (2pt);
\fill[shift only,blue] (Q) circle (2pt);

\end{tikzpicture}
\caption{Constructing an ellipse from the glissette triangle (red)}\label{f.triangle1}
\end{center}
\end{figure}

%%%%%%%%%%%%%%%%%%%%%%%%%%%%%%%%%%%%%%%%%%%%%%%%%%%%%%%%%%%%%%%%%%%%%%%%

\begin{figure}[t]
\begin{center}
\begin{tikzpicture}[scale=.8]
\clip (-.5,-.6) rectangle +(10.2,8.8);

\def\b{8}
\def\a{sqrt(100-\b*\b)}
\def\anglea{30}
\def\angleb{20}
\coordinate (O) at (0,0);
\coordinate (A) at ({\a},0);
\coordinate (B) at ({0,\b});
\path[name path=a,blue] (A) -- +({180-atan2(\b,\a)-\anglea}:8);
\path[name path=b,red] (B) -- +({-90+atan2(\a,\b)+\angleb}:10);
\path [name intersections = {of = a and b, by = {Q} }];
\draw[thick,dotted] (A) -- (B) -- (Q) -- cycle;
\draw (A) -- (O) -- (B);
\fill ($(B)+(-1.5pt,-3pt)$) rectangle +(3pt,6pt);
\fill ($(A)+(-3pt,-1.5pt)$) rectangle +(6pt,3pt);
\coordinate (C) at ($(A)!.5!(B)$);
\draw
  let
    \p1 = ($(Q) - (C)$),
    \n1 = {veclen(\x1, \y1)},
    \p2 = ($(O) - (C)$),
    \n2 = {veclen(\x2, \y2)}
  in 
    (O) -- ($(O) ! \n1+\n2 ! (C)$) coordinate (P);
\node[right] at (P) {$P$};
\draw[thick,red] (O) node[below,black] {$O$} -- 
  (C) node[left,black] {$C$} -- (Q) node[right,black] {$Q$};
\fill[shift only,red] (P) circle (2pt);

\def\b{3}
\def\a{sqrt(100-\b*\b)}
\def\anglea{30}
\def\angleb{20}
\coordinate (O) at (0,0);
\coordinate (A) at ({\a},0);
\coordinate (B) at ({0,\b});
\path[name path=a,blue] (A) -- +({180-atan2(\b,\a)-\anglea}:8);
\path[name path=b,red] (B) -- +({-90+atan2(\a,\b)+\angleb}:10);
\path [name intersections = {of = a and b, by = {Q} }];
\draw[thick,dotted] (A) -- (B) -- (Q) -- cycle;
\draw (A) -- (O) -- (B);
\fill ($(B)+(-1.5pt,-3pt)$) rectangle +(3pt,6pt);
\fill ($(A)+(-3pt,-1.5pt)$) rectangle +(6pt,3pt);
\coordinate (C) at ($(A)!.5!(B)$);
\draw
  let
    \p1 = ($(Q) - (C)$),
    \n1 = {veclen(\x1, \y1)},
    \p2 = ($(O) - (C)$),
    \n2 = {veclen(\x2, \y2)}
  in 
    (O) -- ($(O) ! \n1+\n2 ! (C)$) coordinate (P);
\node[right,xshift=8pt,yshift=2pt] at (P) {$P$};
\draw[thick,blue] (O) -- (C) node[below,black] {$C$} --  
  (Q) node[right,black,xshift=5pt] {$Q$};
\fill[shift only,blue] (P) circle (2pt);

\def\b{5}
\def\a{sqrt(100-\b*\b)}
\def\anglea{30}
\def\angleb{20}
\coordinate (O) at (0,0);
\coordinate (A) at ({\a},0);
\coordinate (B) at ({0,\b});
\path[name path=a,blue] (A) -- +({180-atan2(\b,\a)-\anglea}:8);
\path[name path=b,red] (B) -- +({-90+atan2(\a,\b)+\angleb}:10);
\path [name intersections = {of = a and b, by = {Q} }];
\draw[thick,dotted] (A) -- (B) -- (Q) -- cycle;
\draw (A) -- (O) -- (B);
\fill ($(B)+(-1.5pt,-3pt)$) rectangle +(3pt,6pt);
\fill ($(A)+(-3pt,-1.5pt)$) rectangle +(6pt,3pt);
\coordinate (C) at ($(A)!.5!(B)$);
%\node[right,xshift=10pt,yshift=1pt] at (Q) {$P$};
\draw[thick,green!80!black] (O) -- 
  (C) node[above,black] {$C$} --
  (Q) node[right,black] {$Q,P$};
\fill[shift only,green!80!black] (Q) circle (2pt);

\end{tikzpicture}
\caption{There are $A,B$ such that $OCQ$ is a line segment (green)}\label{f.triangle2}
\end{center}
\end{figure}

%%%%%%%%%%%%%%%%%%%%%%%%%%%%%%%%%%%%%%%%%%%%%%%%%%%%%%%%%%%%%%%%%%%%%%%%

\begin{theorem}
The locus of $Q$ is an ellipse.
\end{theorem}

\begin{proof}
Figure~\ref{f.triangle2} shows that as $AB$ slides down, $CQ$ rotates upwards while $OC$ rotates downwards. By considering the extremes ($A$ near $O$ and $B$ near $O$), it is clear that $OCQ$ will be ``concave'' up at one extreme and ``concave'' down at the other. Therefore, there must be a position of $AB$ which $OCQ$ is a line segment (green in the Figure).

Referring now to Figure~\ref{f.triangle3}, $O[Q]$ is this line segment and $C,Q,P$ are points generated by another arbitrary position of $AB$. Construct $CE\parallel O[Q]$ which bisects $\angle PCQ$, and construct $PN\perp O[Q]$ and $CL\perp O[Q]$.

From $\triangle OCL\sim \triangle CPE$ we have
\begin{eqnarray*}
\frac{CL}{PE}&=&\frac{OC}{CP}\\[4pt]
\frac{CL}{PE}-\frac{PE}{PE}&=&\frac{OC}{CP}-\frac{CP}{CP}\\[4pt]
\frac{CL-PE}{PE}&=&\frac{OC-CP}{CP}\,,
\end{eqnarray*}
and similarly,
\[
\frac{CL+PE}{PE}=\frac{OC+CP}{CP}\,.
\]
Now,
\begin{eqnarray*}
QN &=& EN - EQ = CL - PE\\
PN &=& EN + PE = CL + PE\\[4pt]
\frac{QN}{PN}&=& \frac{CL-PE}{CL+PE}=\frac{OC-CP}{OC+CP}\,.
\end{eqnarray*}
By Theorem~\ref{thm.ellipse-b-over-a-besant}, the locus of $Q$ is an ellipse with $OC+CQ$ the semi-major axis and $OC-CQ$ the semi-minor axis.\hqed
\end{proof}

%%%%%%%%%%%%%%%%%%%%%%%%%%%%%%%%%%%%%%%%%%%%%%%%%%%%%%%%%%%%%%%%%%%%%%%%


\begin{figure}[b]
\begin{center}
\begin{tikzpicture}[scale=.85]
\clip (-.5,-.6) rectangle +(8,7);

\def\b{8}
\def\a{sqrt(100-\b*\b)}
\def\anglea{30}
\def\angleb{20}
\coordinate (O) at (0,0);
\coordinate (A) at ({\a},0);
\coordinate (B) at ({0,\b});
\path[name path=a,blue] (A) -- +({180-atan2(\b,\a)-\anglea}:8);
\path[name path=b,red] (B) -- +({-90+atan2(\a,\b)+\angleb}:10);
\path [name intersections = {of = a and b, by = {Q} }];
\coordinate (C) at ($(A)!.5!(B)$);
\draw
  let
    \p1 = ($(Q) - (C)$),
    \n1 = {veclen(\x1, \y1)},
    \p2 = ($(O) - (C)$),
    \n2 = {veclen(\x2, \y2)}
  in 
    (O) -- ($(O) ! \n1+\n2 ! (C)$) coordinate (P);
\node[right] at (P) {$P$};
\draw[thick,red] (O) node[below,black] {$O$} -- 
  (C) node[left,black] {$C$} -- (Q) node[right,black] {$Q$};

\def\b{5}
\def\a{sqrt(100-\b*\b)}
\def\anglea{30}
\def\angleb{20}
\coordinate (O) at (0,0);
\coordinate (A) at ({\a},0);
\coordinate (B) at ({0,\b});
\path[name path=a,blue] (A) -- +({180-atan2(\b,\a)-\anglea}:8);
\path[name path=b,red] (B) -- +({-90+atan2(\a,\b)+\angleb}:10);
\path [name intersections = {of = a and b, by = {QX} }];
\coordinate (CX) at ($(A)!.5!(B)$);
\draw[thick,green!80!black] (O) -- 
  (CX) --% node[below,black] {$[C]$} --
  (QX) node[right,black] {$[Q]$};

\draw (C) -- ($(O)!(C)!(CX)$) coordinate (L);
\node[below,xshift=2pt,yshift=-2pt] at (L) {$L$};
\draw (P) -- ($(O)!(P)!(QX)$) coordinate (N);
\node[below,xshift=2pt,yshift=-2pt] at (N) {$N$};
\draw (C) -- ($(P)!(C)!(N)$) coordinate (E);
\node[right,xshift=0pt,yshift=0pt] at (E) {$E$};

\node[above right,xshift=10pt,yshift=9pt] at (C) {$\alpha$};
\node[above right,xshift=14pt,yshift=2pt] at (C) {$\alpha$};

\draw[rotate=120] (L) rectangle +(6pt,6pt);
\draw[rotate=124] (N) rectangle +(6pt,6pt);
\draw[rotate=120] (E) rectangle +(6pt,6pt);
\end{tikzpicture}
\caption{$O[Q]$ is the line segment and $C,Q$ are for another arbitrary position of $AB$}\label{f.triangle3}
\end{center}
\end{figure}
