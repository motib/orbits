% !TeX root = orbits.tex
% !TeX Program=pdfLaTeX

\chapter{Marden's Theorem}\label{s.marden}

According to Dan Kalman, Marden's Theorem is ``the Most Marvelous
Theorem in Mathematics'' \cite{marden-marvelous}. This chapter explains his proof of this theorem; it assumes that the reader is familiar with complex numbers.

\begin{theorem}[Marden]\label{thm.marden}
Let $p(z)$ be a cubic polynomial with complex coefficients whose roots $z_1,z_2,z_3$ are non-collinear in the complex plane. Let $T$ be the triangle whose vertices are $z_1,z_2,z_3$. Then there is a unique ellipse $E$ (called the \emph{Steiner inellipse}) inscribed within $T$ such that the sides of $T$ are tangent to the ellipse at their midpoints. $f_1,f_2$, the foci of the ellipse, are the roots of $p'(z)$ (Figures~\ref{f.marden-axis}, \ref{f.marden-arbitrary}).
\end{theorem}
We omit the relatively simple proof that without loss of generality we can assume that two of the roots are $1$ and $-1$, and that the third root $w$ is above the $x$-axis: $w=u+vi,v>0$  \cite[p.~332]{marden}. This follows since the theorem remains true if the triangle is translated, rotated or scaled.

There are three steps to the proof. First, we show that there is \emph{an} ellipse which is tangent to one side of the triangle at its midpoint. Then we show that this ellipse is tangent to \emph{all} three sides of the triangle, and finally, that the points of contact are the midpoints of the sides.

\section{An ellipse that is tangent to one side of the triangle}

\begin{theorem}\label{thm.marden1}
Let $p(z),p'(z),T$ be as defined in the statement of Marden's theorem. Then there is \emph{an} ellipse $E$ such that one side of $T$ is tangent to $E$ at its midpoint.
\end{theorem}

%%%%%%%%%%%%%%%%%%%%%%%%%%%%%%%%%%%%%%%%%%%%%%%%%%%%%%%%%%%%%

\begin{figure}[b]
\begin{center}
\begin{tikzpicture}[scale=2]
\clip (-2.2,-.25) rectangle +(4.4,1.5);
\draw[line width=0.5pt,gray!50] (-2,0) grid (2,1);
\draw[very thick] (-2,0) -- (-1,0);
\draw[very thick] (1,0) -- (2,0);
\draw[very thick] (0,0) -- (0,1);
\coordinate (z1) at (-1,0);
\coordinate (z2) at (1,0);
\coordinate (z3) at (0,1);
\draw[thick,red] (z1) -- (z2) -- (z3) -- cycle;
\coordinate (f1) at ({-sqrt(2)/3},1/3);
\coordinate (f2) at ({sqrt(2)/3},1/3);
\draw[thick,blue] (f1) -- (f2);
\coordinate (o) at (0,1/3);
\draw[name path global=ellipse] (o)
  ellipse[x radius={1/sqrt(3)},y radius= 1/3];
\vertexsmcolor{f1}{blue};
\vertexsmcolor{f2}{blue};
\node[below] at (z1) {$(-1,0)$};
\node[below] at (z2) {$(1,0)$};
\node[above] at (z3) {$(0,i)$};
\node[below] at (0,0) {$(0,0)$};
\node[above right] at (f1) {$f_1$};
\node[above left] at (f2) {$f_2$};
\end{tikzpicture}
\caption{The third root is on the $y$-axis}\label{f.marden-axis}
\end{center}
\end{figure}

%%%%%%%%%%%%%%%%%%%%%%%%%%%%%%%%%%%%%%%%%%%%%%%%%%%%%%%%%%%%%

\paragraph{First example}
Suppose that the third root is on the $y$-axis, say at $i$ (Figure~\ref{f.marden-axis}). Then
\begin{eqn}
p(z) &=& (z+1)(z-1)(z-i) = z^3 -iz^2-z+i\\
p'(z) &=&3z^2-2iz-1=3\left(z^2-\frac{2}{3}iz-\frac{1}{3}\right)\,.
\end{eqn}%
Using the quadratic formula we can compute that the roots of $p'(z)$ are
$f_{1,2}=(1/3)(i\pm \sqrt{2})$, and it is easy to verify that the following relationship holds between the roots $f_1,f_2$ and the coefficients of $p'(z)$.
\begin{eqn}
f_1+f_2&=&\frac{2}{3}i\\
f_1f_2&=&-\frac{1}{3}\,,
\end{eqn}
$c$, one-half the distance between the foci, is $\sqrt{2}/3$, and $b$, the semi-minor axis, is $1/3$, so $a$, the semi-major axis, is
\[
a = \sqrt{
      \left(\frac{1}{3}\right)^2 + \left(\frac{\sqrt{2}}{3}\right)^2
    } =  \frac{1}{\sqrt{3}} \approx 0.577\,.
\]

%%%%%%%%%%%%%%%%%%%%%%%%%%%%%%%%%%%%%%%%%%%%%%%%%%%%%%%%%%%%%

\begin{figure}[t]
\begin{center}
\begin{tikzpicture}[scale=2]
\clip (-2.2,-.25) rectangle +(4.4,2.5);
\draw[line width=0.5pt,gray!50] (-2,0) grid (2,2);
\draw[very thick] (-2,0) -- (-1,0);
\draw[very thick] (1,0) -- (2,0);
\draw[very thick] (0,0) -- (0,2);
\coordinate (z1) at (-1,0);
\coordinate (z2) at (1,0);
\coordinate (z3) at (1,2);
\draw[thick,red] (z1) -- (z2) -- (z3) -- cycle;
\coordinate (f1) at ({(1/3)*(1+sqrt(2))},{(1/3)*(2+sqrt(2))});
\coordinate (f2) at ({(1/3)*(1-sqrt(2))},{(1/3)*(2-sqrt(2))});
\draw[thick,blue] (f1) -- (f2);
\coordinate (o) at (1/3,2/3);
\draw[rotate=45,name path global=ellipse] (o)
  ellipse[x radius={sqrt(2/3)},y radius= {sqrt(2)/3}];
\vertexsmcolor{f1}{blue};
\vertexsmcolor{f2}{blue};
\node[below] at (z1) {$(-1,0)$};
\node[below] at (z2) {$(1,0)$};
\node[above] at (z3) {$(1,2i)$};
\node[below] at (0,0) {$(0,0)$};
\node[above left,yshift=-4pt] at (f1) {$f_1$};
\node[above,yshift=4pt] at (f2) {$f_2$};

\end{tikzpicture}
\caption{The third root is not on the $y$-axis}\label{f.marden-arbitrary}
\end{center}
\end{figure}

%%%%%%%%%%%%%%%%%%%%%%%%%%%%%%%%%%%%%%%%%%%%%%%%%%%%%%%%%%%%%

\paragraph{The general case}
These equations are essentially the same for any position of the third root $w$ above the $x$-axis.
\begin{eqn}
p(z) &=& (z+1)(z-1)(z-w) = z^3 -wz^2-z+i)\\
p'(z) &=&3z^2-2wz-1=3\left(z^2-\frac{2}{3}wz-\frac{1}{3}\right)\\
f_{1,2}&=&\frac{1}{3}(w\pm \sqrt{w^2+3})\\
f_1+f_2&=&\frac{2}{3}w\\
f_1f_2&=&-\frac{1}{3}\,.
\end{eqn}

\paragraph{Second example}
Let $w=1+2i$ and substitute in the above equations for the foci.
\begin{eqn}
f_{1,2}&=&\frac{1}{3}((1+2i)\pm 2\sqrt{i})=
\frac{1}{3}((1\pm\sqrt{2})+(2\pm\sqrt{2})i)\\
f_1+f_2&=&\frac{2}{3}(1+2i)\\
f_1f_2&=&-\frac{1}{3}\,.
\end{eqn}%
where we used the fact that $\sqrt{i}=(1+i)/\sqrt{2}$ (check!).

One-half the distance between the foci is $2/3$ and the semi-minor axis is $\sqrt{2}/3$, so the semi-major axis is $\sqrt{2/3}$. The ellipse is rotated by $45^\circ$ (Figure~\ref{f.marden-arbitrary}).


\begin{proof} (Theorem~\ref{thm.marden1})\\
Case 1: $w$ is not on the $y$-axis. Let $f_1=c_1+d_1i, f_2=c_2+d_2i$ and recall that $w=u+vi,v>0$. Then $f_1+f_2=\frac{2}{3}w=\frac{2}{3}(c_1+c_2)+\frac{2}{3}(d_1+d_2)i$, so $d_1+d_2>0$ and at least one of $f_1,f_2$ is above the $x$-axis. Since $f_1f_2=-\frac{1}{3}$, the imaginary part of the product is zero. Construct the lines from the foci to the origin (Figure~\ref{f.marden-ellipses}). Using the trigonometric identity for the sum of two tangents, we have
\begin{eqn}
\frac{d_1}{c_1}+\frac{d_2}{c_2}&=&0\\
\tan \theta_1+\tan \theta_2&=& 
\frac{\sin(\theta_1+\theta_2)}{\cos\theta_1 \cos\theta_2}=0\,.
\end{eqn}%
By assumption, the foci are not on $y$-axis so the cosines are non-zero, and, therefore, $\sin(\theta_1+\theta_2)=0$. $\theta_1+\theta_2$ cannot be $0^\circ$ since $w>0$ and we conclude that $\theta_1+\theta_2=180^\circ$.

%%%%%%%%%%%%%%%%%%%%%%%%%%%%%%%%%%%%%%%%%%%%%%%%%%%%%%%%%%%%%

\begin{figure}[t]
\begin{center}
\begin{tikzpicture}[scale=2.5]
%\clip (-1.2,-.2) rectangle +(2.4,1);

\draw[line width=0.5pt,gray!50] (-2,0) grid (2,2);
\coordinate (O) at (0,0);
\coordinate (z1) at (-1,0);
\coordinate (z2) at (1,0);
\coordinate (z3) at (1,2);
\draw[thick,red] (z1) -- (z2) -- (z3) -- cycle;
\coordinate (f1) at ({(1/3)*(1+sqrt(2))},{(1/3)*(2+sqrt(2))});
\coordinate (f2) at ({(1/3)*(1-sqrt(2))},{(1/3)*(2-sqrt(2))});
\draw[thick,dotted,blue] (f1) -- (f2);
\coordinate (o) at (1/3,2/3);
\draw[thick,dotted,rotate=45,name path global=ellipse] (o)
  ellipse[x radius={sqrt(2/3)},y radius= {sqrt(2)/3}];

\draw[thick,dashed] (f1) -- (O) -- ($(O)!4!(f2)$);
\draw (.2,0) arc[start angle = 0, end angle = 54.7, radius = .2];
\draw (.35,0) arc[start angle = 0, end angle = {180-54.7}, radius = .35];
\draw (-.2,0) arc[start angle = 180, end angle = {180-54.7}, radius = .2];

\vertexsmcolor{f1}{blue};
\vertexsmcolor{f2}{blue};
\node[below] at (z1) {$(-1,0)$};
\node[below] at (z2) {$(1,0)$};
\node[below] at (0,0) {$(0,0)$};
\node[above] at (z3) {$(1,2i)$};
\node[left,yshift=-2pt] at (f1) {$f_1$};
\node at (.7,.6) {$f_1$};
\draw[->] (.7,.67) -- +(0,.1);
\node[right,xshift=4pt] at (f2) {$f_2$};
\node[above right,xshift=22pt,yshift=12pt] at (O) {\sm{\theta_1}};
\node[above left,xshift=-20pt,yshift=12pt] at (O) {\sm{\theta_1}};
\node[above,xshift=20pt,yshift=46pt] at (O) {\sm{\theta_2}};

\end{tikzpicture}
\caption{The angles of the lines from the foci to the origin}\label{f.marden-ellipses}
\end{center}
\end{figure}

Since $\theta_1, \theta_2$ are supplementary, the angle between the line from $f_2$ and the origin forms an angle $\theta_1$ with the negative $x$-axis (Figure~\ref{f.marden-ellipses}). Consider the ellipse with foci $f_1,f_2$ that passes through the origin, which is also the midpoint of one side of the triangle. Since the side on the $x$-axis makes equal angles with the lines to the foci, by Theorem~\ref{thm.tangent-angles}, this side is a tangent.

(Case 2: $w$ is on the $y$-axis. Then $w=vi, v>0$ and $f_1+f_2=\frac{1}{3}vi$, so the foci are symmetric around the $y$-axis and have the same imaginary part and the angles from the origin to the foci are equal. In case both foci are on the $y$-axis, the two angles are equal right angles.\hqed
\end{proof}

%%%%%%%%%%%%%%%%%%%%%%%%%%%%%%%%%%%%%%%%%%%%%%%%%%%%%%%%%%%%%

\section{The ellipse is tangent to all three sides of the triangle}

We showed that the side of the triangle on the $x$-axis is a tangent at its midpoin to the ellipse whose foci are the roots of $p'(z)$. To prove Marden's theorem, we must show that the other sides of the triangle are tangent at their midpoints to the \emph{same} ellipse. 

\begin{theorem}\label{thm.marden2}
Let $p(z),p'(z),T$ be as defined in the statement of Marden's theorem and the ellipse $E$ with foci $f_1,f_2$ be constructed as described in Theorem~\ref{thm.marden1}. Then the other two sides of $T$ are tangent to $E$.
\end{theorem}

\begin{proof}
By symmetry it is sufficient to prove for one of the other sides $OW$ (Figure~\ref{f.marden-tangent}). In the Figure, the vertices of the triangle on the $x$-axis are now $(0,0),(2,0)$, and the third vertex is at an arbitrary position $w=u+iv,v>0$. As before we have $f_1+f_2=\frac{2}{3}(1+w)$ and $f_1f_2=\frac{w}{3}$. From the first equation and $w>0$ we know that at least one focus is above the $x$-axis and since the ellipse is tangent to the $x$-axis the other focus is too.

Let us express $f_1f_2=w/3$ in polar coordinates (notation from Figure~\ref{f.marden-tangent}).
\[
f_1f_2=r_1e^{i\theta_1}\cdot r_2e^{i\theta_2}=r_1r_2e^{i(\theta_1+\theta_2)}=\frac{r}{3}e^{i\theta}\,,
\]
so $\theta_1+\theta_2=\theta$ and $\theta-\theta_2=\theta_1$. Therefore, $\angle f_2OV=\theta=\angle f_1OW$. But $O=(0,0)$ is \emph{external} to the ellipse because $OV$ is tangent to the ellipse at $(1,0)$. By Theorem~\ref{thm.marden-tangent}, $OW$ is tangent to the ellipse.\hqed
\end{proof}

%%%%%%%%%%%%%%%%%%%%%%%%%%%%%%%%%%%%%%%%%%%%%%%%%%%%%%%%%%%%%

\begin{figure}[t]
\begin{center}
\begin{tikzpicture}[scale=2.5]
\clip (-2.2,-.25) rectangle +(4.4,2.5);
\draw[line width=0.5pt,gray!50] (-2,0) grid (2,2);
%\draw[very thick] (-2,0) -- (-1,0);
%\draw[very thick] (1,0) -- (2,0);
%\draw[very thick] (-1,0) -- (-1,2);
\coordinate (z1) at (-1,0);
\coordinate (z2) at (1,0);
\coordinate (z3) at (1,2);
\draw[thick,red] (z1) -- (z2) -- (z3) -- cycle;
\coordinate (f1) at ({(1/3)*(1+sqrt(2))},{(1/3)*(2+sqrt(2))});
\coordinate (f2) at ({(1/3)*(1-sqrt(2))},{(1/3)*(2-sqrt(2))});
\draw[thick,blue] (f1) -- (f2);
\coordinate (o) at (1/3,2/3);
\draw[thick,dotted,rotate=45,name path global=ellipse] (o)
  ellipse[x radius={sqrt(2/3)},y radius= {sqrt(2)/3}];
\vertexsmcolor{f1}{blue};
\vertexsmcolor{f2}{blue};
\node[below] at (z1) {$O=(0,0)$};
\node[below] at (z2) {$V=(2,0)$};
\node[above] at (z3) {$W=w=(2,2)$};
\node[below] at (0,0) {$(1,0)$};
\node[above left,yshift=-4pt] at (f1) {$f_1$};
\node[right,xshift=10pt] at (f2) {$f_2$};
\coordinate (O) at (-1,0);
\draw[thick,dashed] (f1) -- (O) -- (f2);
\draw (-.3,0) arc[start angle = 0, end angle = {32}, radius = .7];
\draw (-.6,0) arc[start angle = 0, end angle = {13}, radius = .4];
\draw (-.05,0) arc[start angle = 0, end angle = {45}, radius = .95];

\node[right,xshift=32pt,yshift=4pt] at (O) {\sm{\theta_1}};
\node[above right,xshift=46pt,yshift=12pt] at (O) {\sm{\theta_2}};
\node[above right,xshift=60pt,yshift=22pt] at (O) {\sm{\theta}};
\end{tikzpicture}
\caption{The ellipse is also tangent to $OW$}\label{f.marden-tangent}
\end{center}
\end{figure}

%%%%%%%%%%%%%%%%%%%%%%%%%%%%%%%%%%%%%%%%%%%%%%%%%%%%%%%%%%%%%%%%

\begin{proof} (Theorem~\ref{thm.marden})
We must show that the tangent $OW$ contacts the ellipse at its midpoint. By construction, ellipse $E$ is tangent to the \emph{midpoint} of $OV$ and by Theorem~\ref{thm.marden2} it is also tangent to $OW,VW$. Using Theorem~\ref{thm.marden1} again, construct an ellipse $E'$ that is tangent to the $OW$ at its midpoint, which by Theorem~\ref{thm.marden2} is also tangent to the other sides of $T$. But $E,E'$ have the same foci and the same tangents, so they must be the same ellipse.\hqed
\end{proof}

\section{The Steiner circumellipse}

Any triangle can have a circle circumscribed around it and a circle inscribed within it (Figure~\ref{f.circles}). The center of the circumcircle, the \emph{circumcenter}, is the intersection of the perpendicular bisectors to the sides (blue). The center of the incircle, the \emph{incenter}, is the intersection of the angle bisectors (red). The intersection of the medians, the \emph{centroid}, the center of the unique Steiner inellipse as proved in Marden's Theorem (green), because the sides of the triangle are tangent to the inellipse at their midpoints.

%%%%%%%%%%%%%%%%%%%%%%%%%%%%%%%%%%%%%%%%%%%%%%%%%%%%%%%%%%%%%%%%

\begin{figure}[t]
\begin{center}
\begin{tikzpicture}[scale=2]
\clip (-.5,-.5) rectangle +(3,3);

\coordinate (z1) at (0,0);
\coordinate (z2) at (2,0);
\coordinate (z3) at (2,2);

\draw[name path=side1] (z1) -- (z2);
\draw[name path=side2] (z2) -- (z3);
\draw[name path=side3] (z3) -- (z1);

\path[name path=angle1] (z1) -- +(22.5:2.5);
\path[name path=angle2] (z2) -- +(135:2.5);
\path[name path=angle3] (z3) -- +(-112.5:2.5);

\path[name intersections={of=angle1 and side2,by={tan1}}];
\path[name intersections={of=angle2 and side3,by={tan2}}];
\path[name intersections={of=angle3 and side1,by={tan3}}];
\draw[red] (z1) -- (tan1);
\draw[red] (z2) -- (tan2);
\draw[red] (z3) -- (tan3);

\path [name intersections={of=angle1 and angle2,by={incenter}}];
\node[draw,red,circle through=(tan2)] at (incenter) {};

\draw[thick,name path=side1] (z1) -- (z2);
\draw[thick,name path=side2] (z2) -- (z3);
\draw[thick,name path=side3] (z3) -- (z1);

\draw[blue,name path=perp1] ($(z1)!.5!(z2)$) -- +(0,1);
\draw[blue,name path=perp2] ($(z2)!.5!(z3)$) -- +(-1,0);
\draw[blue,name path=perp3] ($(z3)!.5!(z1)$) -- +(-45:1.4);

\coordinate (circum) at (1,1);
\node[draw,blue,circle through=(z1)] at (circum) {};


\draw[name path=med1,color=green!80!black] (z1) -- (2,1); 
\draw[name path=med2,color=green!80!black] (z2) -- (1,1); 
\draw[name path=med3,color=green!80!black] (z3) -- (1,0); 

\path [name intersections={of=med1 and med2,by={centroid}}];
\vertexcolor{centroid}{green!80!black};
\vertexcolor{incenter}{red};
\vertexcolor{circum}{blue};


\end{tikzpicture}
\caption{Incenter, circumcenter, centroid}\label{f.circles}
\end{center}
\end{figure}

In a circle the centroid divides each medians into two segments: the one from the vertex to the centroid is twice as long as the one from the centroid to the side (Figure~\ref{f.two-to-one}). An ellipse has a unique inellipse, \emph{Steiner (circum)ellipse}, which is twice as large as the inellipse (Figure~\ref{f.steiner-ellipse}).

\begin{figure}[b]
\begin{minipage}{.49\textwidth}
\begin{center}
\begin{tikzpicture}[scale=2]
\clip (-1.2,-.8) rectangle +(2.5,2.8);

\path (-1,1.5) rectangle +(2,1.5); % Padding

\coordinate (z1) at (-1,0);
\coordinate (z2) at (1,0);

\path[name path=side1] (z1) -- +(60:2);
\path[name path=side2] (z2) -- +(120:2);
\path[name intersections={of=side1 and side2,by={z3}}];
\draw[red] (z1) -- (z2) -- (z3) -- cycle;

\draw[name path=med1] (z1) -- ($(z2)!.5!(z3)$) coordinate (mid1);
\draw[name path=med2] (z2) -- ($(z1)!.5!(z3)$) coordinate (mid2);
\draw[name path=med3] (z3) -- ($(z1)!.5!(z2)$) coordinate (mid3);
\path[name intersections={of=med1 and med2,by={o}}];

\vertex{o};
\node[draw,blue,circle through=(mid1)] at (o) {};
\node[draw,blue,circle through=(z1)] at (o) {};

\path (z1) -- node[fill=white] {$2$} (o) -- node[fill=white] {$1$} (mid1);
\path (z2) -- node[fill=white] {$2$} (o) -- node[fill=white] {$1$} (mid2);
\path (z3) -- node[fill=white] {$2$} (o) -- node[fill=white] {$1$} (mid3);

\path (-1,-.85) rectangle +(2,1); % Padding

\end{tikzpicture}
\caption{The two-one division of the\\medians}\label{f.two-to-one}
\end{center}
\end{minipage}
\begin{minipage}{.49\textwidth}
\begin{center}
\begin{tikzpicture}[scale=2]
\clip (-.1,-.7) rectangle +(2.8,2.8);

\coordinate (z1) at (0,0);
\coordinate (z2) at (2,0);
\coordinate (z3) at (2,2);

\draw[name path=side1] (z1) -- (z2);
\draw[name path=side2] (z2) -- (z3);
\draw[name path=side3] (z3) -- (z1);

\path[name path=angle1] (z1) -- +(22.5:2.5);
\path[name path=angle2] (z2) -- +(135:2.5);
\path[name path=angle3] (z3) -- +(-112.5:2.5);

\path[name intersections={of=angle1 and side2,by={tan1}}];
\path[name intersections={of=angle2 and side3,by={tan2}}];
\path[name intersections={of=angle3 and side1,by={tan3}}];
\path[red] (z1) -- (tan1);
\path[red] (z2) -- (tan2);
\path[red] (z3) -- (tan3);

\path [name intersections={of=angle1 and angle2,by={incenter}}];
%\node[draw,red,circle through=(tan2)] at (incenter) {};

\draw[thick,name path=side1] (z1) -- (z2);
\draw[thick,name path=side2] (z2) -- (z3);
\draw[thick,name path=side3] (z3) -- (z1);

\path[blue,name path=perp1] ($(z1)!.5!(z2)$) -- +(0,1);
\path[blue,name path=perp2] ($(z2)!.5!(z3)$) -- +(-1,0);
\path[blue,name path=perp3] ($(z3)!.5!(z1)$) -- +(-45:1.4);

\coordinate (circum) at (1,1);
%\node[draw,blue,circle through=(z1)] at (circum) {};

\path[name path=med1,color=green!80!black] (z1) -- (2,1); 
\path[name path=med2,color=green!80!black] (z2) -- (1,1); 
\path[name path=med3,color=green!80!black] (z3) -- (1,0); 

\path [name intersections={of=med1 and med2,by={centroid}}];

\coordinate (o) at (4/3,2/3);
\draw[very thick,dotted,rotate=45,name path global=ellipse] (o)
  ellipse[x radius={sqrt(2/3)},y radius= {sqrt(2)/3}];
\draw[very thick,dotted,rotate=45,name path global=ellipse] (o)
  ellipse[x radius={2*sqrt(2/3)},y radius= {2*sqrt(2)/3}];

\draw[green!80!black] (z1) -- ($(z1)!2!(centroid)$);
\draw[green!80!black] (z2) -- ($(z2)!2!(centroid)$);
\draw[green!80!black] (z3) -- ($(z3)!2!(centroid)$);

\vertexcolor{centroid}{green!80!black};
%\vertexcolor{incenter}{red};
%\vertexcolor{circum}{blue};

\end{tikzpicture}
\caption{The Steiner inellipse and circumellipse}\label{f.steiner-ellipse}
\end{center}
\end{minipage}
\end{figure}