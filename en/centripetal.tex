% !TeX root = orbits.tex
% !TeX Program=pdfLaTeX

%%%%%%%%%%%%%%%%%%%%%%%%%%%%%%%%%%%%%%%%%%%%%%%%%%%%%%%%%%%%%%%%

\chapter{A proof Proposition~XI, Problem~VI}
\label{s.centripetal}

Theorem~\ref{thm.lr-limit} is Book~I, Section~III, Proposition~XI, Problem~VI of the \emph{Principia}.

Study Figure~\ref{f.elliptical-orbit-1}:\footnote{The bottom half of the ellipse is not shown, but we still refer to lines $DC,PC$ as diameters.}
\begin{itemize}
\item Let $P,Q$ be two points on the ellipse that represent the movement of a body in an elliptical orbit separated by a time interval $\Delta t$. Construct lines from $P$ to the center $C$ and the foci $S,H$.

\item Construct the tangent at $P$ and choose $R$ on the tangent such that the body would move from $P$ to $R$ if it continued for time $\Delta t$ not subject to any force. Construct the parallelogram $PRQX$ and extend $QX$ until it intersects $PC$ at $V$.

\item Construct a line parallel to $RP$ through $H$ and let $I$ be its intersection with $PS$. 

\item Construct $DC$ the conjugate diameter to $PC$ (Definition~\ref{def.conjugate}), and let $E$ be its intersection with $PS$.
\end{itemize}

%%%%%%%%%%%%%%%%%%%%%%%%%%%%%%%%%%%%%%%%%%%%%%%%%%%%%%%%%%%%%%%%%%%%%%

\begin{figure}[b]
\begin{center}
\begin{tikzpicture}

\clip (-4.5,-.5) rectangle +(10,4.5);

% Size and center of the ellipse
\def\a{4.375}
\def\b{3.5}
\def\angle{30}

\pic{semi-ellipse={\a}/{\b}};
\pic{point-on-ellipse={\a}/{\b}/{\angle}};
\pic{tangent={.8*\a}/{.8*\b}/{\angle}};

\node[above] at (Top) {$B$};
\node[below] at (Right) {$A$};
\node[below] at (O) {$O$};
\node[below] at (F1) {$S$};
\node[below] at (F2) {$H$};

\draw[name path=fromF1p] (F1) -- (P);
\draw[name path=ph] (P) -- (F2);
\draw[name path=pc] (P) -- (O);

\path (P) -- node [above,xshift=4pt,yshift=-2pt] {$Z$} (Z);

% Select an arbitrary point Q on the ellipse and draw a line to the Sun
\coordinate (Q) at (65:{\a} and {\b});
\node[above] at (Q) {$Q$};

% Draw QR
\path[name path=qr] (Q) -- +(\angle:{\a/2} and {\b/2});
\path [name intersections = {of = t and qr, by = {R} }];
\node[right] at (R) {$R$};
\draw (R) -- (Q);

% Draw QX
\path[name path=qx] (Q) -- +({\angle-90}:{1.5*\a} and {1.5*\b});
\path [name intersections = {of = qx and fromF1p, by = {X} }];
\path [name intersections = {of = qx and pc, by = {V} }];
\draw (Q) -- (X) node[above right,xshift=-3pt] {$X$} -- 
  (V) node[below] {$V$};

% Conjugate diameter
\draw[name path=cd] (O) -- +({\angle+90}:{\a} and {\b}) coordinate (D);
\node[above left] at (D) {$D$};
\path [name intersections = {of = fromF1p and cd, by = {E} }];
\node[above left,xshift=-4pt,yshift=-2pt] at (E) {$E$};

% Draw IH
\path[name path=ih] (F2) -- +({\angle+90}:{\a} and {\b});
\path [name intersections = {of = ih and fromF1p, by = {I} }];
\node[above] at (I) {$I$};
\draw (F2) -- (I);

% Label angles
\node[left,xshift=-2pt,yshift=4pt]   at (P)  {\sm{\alpha}};
\node[below,xshift=-2pt,yshift=-4pt] at (P)  {\sm{\alpha}};
\node[right,xshift=1pt,yshift=-2pt]  at (I)  {\sm{\alpha}};
\node[above,xshift=1pt,yshift=3pt]   at (F2) {\sm{\alpha}};

% Label line segments
\path (F1) -- node[below,blue] {$c$} (O) -- node[below,blue] {$c$} (F2);
\path (F1) -- node[above,blue] {$e$} (E) -- node[above,blue] {$e$} (I);
\draw[thick,red] (I) -- node[above,near start] {$d$} (P) -- node[below] {$d$} (F2);

\draw[thick,red] ($(I)+(40:1pt)$) -- ($(F2)+(40:1pt)$);
\draw[thick,blue] (F1) -- (F2) -- (I) -- cycle;
\draw[thick,blue] (E) -- (O);

\path (R) -- node[right] {$\Delta t$} (P);

\end{tikzpicture}
\caption{Geometry of an elliptical orbit (1)}\label{f.elliptical-orbit-1}
\end{center}
\end{figure}

%%%%%%%%%%%%%%%%%%%%%%%%%%%%%%%%%%%%%%%%%%%%%%%%%%%%%%%%%%%%%%%%%%%%%%

\section{A formula for $QR$}

\begin{theorem}\label{thm.qr}
$QR = PV\cdot \displaystyle\frac{CA}{CP}$.
\end{theorem}

\begin{proof}
By Theorem~\ref{thm.tangent-angles}, $\angle RPX = \angle ZPH=\alpha$ and by alternate interior angles,
\[
\angle PHI = \angle ZPH = \alpha = \angle RPX = \angle PIH\,,
\]
so $\triangle IPH$ (red) is isosceles and $PI=PH=d$.

$SC=CH=c$ are equal because they are the distances of the foci from the center of the ellipse. Let $SE=e$. By construction $EC\parallel IH$ so $\triangle ESC \sim \triangle ISH$ (blue) and
\begin{eqn}
\frac{SC}{SE}&=&\frac{SH}{SI}\\[4pt]
SI&=&\frac{SH\cdot SE}{SC}=\frac{2c\cdot e}{c}= 2e\,.
\end{eqn}%
By definition of an ellipse $SP+PH=SI+IP+PH=2CA$ so $2e+d+d=2CA$ and $EP=e+d=CA$.

$QV \parallel EC$ so  $\triangle EPC \sim \triangle XPV$ and
\begin{eqn}
\frac{PX}{PV} &=& \frac{EP}{PC} =\frac{CA}{PC}\\[4pt]
PX &=& PV\cdot \frac{CA}{PC}\,.
\end{eqn}%
Since $PRQX$ is a parallelogram $QR=PX$, $QR= PV\cdot\displaystyle\frac{AC}{PC}$.\hqed
\end{proof}

%%%%%%%%%%%%%%%%%%%%%%%%%%%%%%%%%%%%%%%%%%%%%%%%%%%%%%%

\section{A formula for $QT$}

Construct a perpendicular from $P$ to $DC$ and label its intersection with $DC$ by $F$. Construct a perpendicular from $Q$ to $SP$ and label its intersection with $SP$ by $T$ (Figure~\ref{f.elliptical-orbit-2}).
\begin{theorem}\label{thm.qt}
\[
QT=QX\cdot \frac{FP}{CA}\,.
\]
\end{theorem}

%%%%%%%%%%%%%%%%%%%%%%%%%%%%%%%%%%%%%%%%%%%%%%%%%%%%%%%%%%%%%%%%%%%%%%

\begin{figure}[t]
\begin{center}
\begin{tikzpicture}

\clip (-4.5,-1) rectangle +(10,5);

% Size and center of the ellipse
\def\a{4.375}
\def\b{3.5}
\def\angle{30}

\pic{ellipse={\a}/{\b}};
\pic{point-on-ellipse={\a}/{\b}/{\angle}};
\pic{tangent={\a}/{\b}/{\angle}};

\node[above] at (Top) {$B$};
\node[right,xshift=-1pt] at (Right) {$A$};
\node[below left] at (O) {$O$};
\node[below] at (F1) {$S$};
\node[below] at (F2) {$H$};

\draw[name path=fromF1p] (F1) -- (P);
\draw[name path=ph] (P) -- (F2);
\draw[name path=pc] (P) -- (O);
\node[above,xshift=4pt,yshift=-2pt] at (Z) {$Z$};

% Select an arbitrary point Q on the ellipse and draw a line to the Sun
\coordinate (Q) at (65:{\a} and {\b});
\node[above] at (Q) {$Q$};

% Draw QR
\path[name path=qr] (Q) -- +(\angle:{\a/2} and {\b/2});
\path [name intersections = {of = t and qr, by = {R} }];
\node[right] at (R) {$R$};
\draw (R) -- (Q);

% Draw QX
\path[name path=qx] (Q) -- +({\angle-90}:{1.5*\a} and {1.5*\b});
\path [name intersections = {of = qx and fromF1p, by = {X} }];
\draw (Q) -- (X) node[above right,xshift=-3pt] {$X$};

% Draw T
\path (Q) -- ($(F1)!(Q)!(P)$) coordinate (T);
\node[above left,yshift=-3pt] at (T) {$T$};
\draw[rotate=13] ($(T)+(0pt,1pt)$) rectangle +(5pt,5pt);

% Conjugate diameter
\draw[name path=cd] (O) -- +({\angle+90}:{\a} and {\b}) coordinate (D);
\node[above left] at (D) {$D$};
\draw[name path=dk] (O) -- +({\angle+90}:{-1.5*\a} and {-1.5*\b});
\path [name intersections = {of = dk and ellipse, by = {K} }];

\path [name intersections = {of = fromF1p and cd, by = {E} }];
\node[above left,xshift=-4pt,yshift=-2pt] at (E) {$E$};

% Draw PF
\draw (P) -- ($(O)!(P)!(K)$) coordinate (F)
  node[right,xshift=2pt,yshift=-1pt] {$F$};
\draw[rotate=33] (F) rectangle +(5pt,5pt);

\draw[blue,thick] (E) -- (P) -- (F) -- cycle;
\draw[red,thick] ($(T)+(0pt,1pt)$) -- 
  ($(Q)+(.2pt,-1.2pt)$) -- ($(X)+(-.4pt,1pt)$);
\draw[red,thick] ($(T)+(0pt,1pt)$) -- ($(X)+(-.2pt,1pt)$);

% Label angles
\node[below right,xshift=6pt,yshift=7pt]   at (Q)  {\sm{\beta}};
\node[left,xshift=-3pt,yshift=3pt] at (P)  {\sm{\beta}};
\node[right,xshift=2pt,yshift=-4pt] at (E)  {\sm{\beta}};
\node[left,xshift=-1pt,yshift=3pt] at (X)  {\sm{\beta}};

\end{tikzpicture}
\caption{Geometry of an elliptical orbit (2)}\label{f.elliptical-orbit-2}
\end{center}
\end{figure}

%%%%%%%%%%%%%%%%%%%%%%%%%%%%%%%%%%%%%%%%%%%%%%%%%%%%%%%%%%%%%%%%%%%%%%

\begin{proof}
By construction, $QR \parallel PX$, so by alternate interior angles $\angle RQX =\angle QXT=\beta$. By construction, $QX \parallel DC$, so by alternate interior angles $\angle QXT =\angle PEF=\beta$. Since $\triangle PFE$ and $\triangle QTX$ are right triangles with an equal acute angle $\beta$, $\triangle PFE\sim\triangle QTX$. In the proof of Theorem~\ref{thm.qr} we showed that $EP=CA$ so
\begin{eqn}
\frac{QT}{QX}&=& \frac{FP}{EP}\\
QT&=&QX\cdot \frac{FP}{EP}= QX\cdot\frac{FP}{CA}\,.\fqed
\end{eqn}%
\end{proof}

%%%%%%%%%%%%%%%%%%%%%%%%%%%%%%%%%%%%%%%%%%%%%%%%%%%%%%%

\begin{figure}[b]
\begin{center}
\begin{tikzpicture}
\clip (-4.5,-3) rectangle +(9,6);

% Size and center of the ellipse
\def\a{3}
\def\b{2}
\def\angle{35}

\pic{ellipse={\a}/{\b}};
\pic{point-on-ellipse={\a}/{\b}/{\angle}};
\pic{tangent={\a}/{\b}/{\angle}};

\node[above,yshift=-1pt] at (Top) {$B$};
\node[right] at (Right) {$A$};
\node[below left,xshift=3pt,yshift=-2pt] at (O) {$O$};

% Conjugate diameter from P through C
\draw[name path=pc] (P) -- ($(P)!2!(O)$) coordinate (G);

% Conjugate diameters from D through C
\path[name path=cd] (O) -- +({\angle+90}:{1.5*\a} and {1.5*\b});
\path[name path=dk] (O) -- +({\angle+90}:{-1.5*\a} and {-1.5*\b});
\path [name intersections = {of = cd and ellipse, by = {D} }];
\path [name intersections = {of = dk and ellipse, by = {K} }];
\node[above left,xshift=1pt,yshift=-5pt] at (D) {$D$};
%\node[below right,xshift=1pt,yshift=-5pt] at (K) {$K$};
\draw (D) -- (K);

% Draw tangent at G (continuation of PC)
\draw[name path global=t2] 
  (G) -- ++ ({-\a*sin(\angle)},{\b*cos(\angle)})
  (G) -- ++ ({\a*sin(\angle)},{-\b*cos(\angle)});
% Draw tangent at D
\draw[name path global=t3] 
  (D) -- ++ ({-\a*sin(90+\angle)},{\b*cos(90+\angle)})
  (D) -- ++ ({\a*sin(90+\angle)},{-\b*cos(90+\angle)});
% Draw tangent at K (continuation of DC)
\draw[name path global=t4] 
  (K) -- ++ ({-\a*sin(90+\angle)},{\b*cos(90+\angle)})
  (K) -- ++ ({\a*sin(90+\angle)},{-\b*cos(90+\angle)});

% Draw large dashed rectangle
\draw[thick,dashed] (\a,\b) -- (-\a,\b) -- (-\a,-\b) -- (\a,-\b) -- cycle;

% Get the intersections of the tangents
\path [name intersections = {of = t  and t3, by = {J} }];
\path [name intersections = {of = t2 and t3, by = {KK} }];
\path [name intersections = {of = t  and t4, by = {M} }];
\path [name intersections = {of = t2 and t4, by = {LL} }];

% Draw the parallelogram
\draw (J) -- (KK) -- (LL) -- (M) -- cycle;

% Draw triangles
\draw[thick,blue] (O) -- (D) -- (P) -- cycle;
\draw[thick,red] (O) -- (Top) -- (Right) -- cycle;

%% Draw perpendicular to conjugate diameter
\draw[very thick,green!70!black] (P) -- ($(D)!(P)!(O)$) coordinate (F);
\node[below left,xshift=5pt,yshift=1pt] at (F) {$F$};
\draw[rotate=40] (F) rectangle +(6pt,6pt);

\end{tikzpicture}
\caption{Parallelograms formed by conjugate diameters}\label{f.conj-diam-para}
\end{center}
\end{figure}

%%%%%%%%%%%%%%%%%%%%%%%%%%%%%%%%%%%%%%%%%%%%%%%%%%%%%%%%%%%%%%%%%%%%%%

\section{A formula for $QR/QT^2$}

\begin{theorem}
\begin{equation}
\frac{QR}{QT^2}=\frac{CP\cdot CA}{CB^2}\cdot \frac{QV^2}{GV\cdot QX^2}\,.
\label{eqn.three-multiplications}
\end{equation}
\end{theorem}

%%%%%%%%%%%%%%%%%%%%%%%%%%%%%%%%%%%%%%%%%%%%%%%%%%%%%%%%%%%%%%%%%%%%%%

\begin{proof}
Let use combine the equations in Theorems~\ref{thm.qr} and ~\ref{thm.qt} to get $QR/QT^2$.
\begin{equation}
\frac{QR}{QT^2}=\frac{PV\cdot\displaystyle\frac{CA}{CP}}
{\left(QX\cdot \displaystyle\frac{FP}{CA}\right)^2}=
\frac{PV\cdot CA^3}
{QX^2\cdot CP\cdot FP^2}\,.
\end{equation}%
$DC$ and $PC$ are conjugate diameters so Theorem~\ref{thm.conj-diag} gives a formula for $PV$ that we substitute into Equation~\ref{eqn.qr-qt}.
\begin{equation}
\frac{QR}{QT^2}=
\frac{QV^2\cdot CP^2}{GV \cdot CD^2}\cdot
\frac{CA^3}
{QX^2\cdot CP\cdot FP^2}=\frac{CP\cdot CA^3}{CD^2\cdot FP^2}\cdot \frac{QV^2}{GV\cdot QX^2}\,.\label{eqn.qr-qt}
\end{equation}

%%%%%%%%%%%%%%%%%%%%%%%%%%%%%%%%%%%%%%%%%%%%%%%%


Next we show that $CD\cdot FP = CA \cdot CB$. By Theorem~\ref{thm.conj-diam-para} the areas of the parallelograms formed by the tangents to conjugate diameters are equal. By symmetry the areas of the four small parallelograms are equal, as are the triangles formed by constructing diagonals. In Figure~\ref{f.conj-diam-para} the area of the $\triangle ABC$ (red), which is $(1/2)CA\cdot CB$, is equal to the area of $\triangle PCD$ (blue), which is $(1/2)CD\cdot FP$. Substituting for $CD\cdot FP$ in Equation~\ref{eqn.qr-qt} gives
\begin{eqn}
\frac{QR}{QT^2}&=&\frac{CP\cdot CA^3}{CB^2\cdot CA^2}\cdot \frac{QV^2}{GV\cdot QX^2}
=\frac{CP\cdot CA}{CB^2}\cdot \frac{QV^2}{GV\cdot QX^2}\,.\fqed
\end{eqn}
\end{proof}

%%%%%%%%%%%%%%%%%%%%%%%%%%%%%%%%%%%%%%%%%%%%%%%%%%%%%%%%%

\begin{figure}
\begin{center}
\begin{tikzpicture}

% Size and center of the ellipse
\clip (-5.5,-.5) rectangle +(11.5,4.7);

\def\a{5.25}
\def\b{3.5}

\def\angle{30}

\pic{ellipse={\a}/{\b}};
\pic{point-on-ellipse={\a}/{\b}/{\angle}};
\pic{tangent={\a}/{\b}/{\angle}};

\draw (F1) -- (P) -- (F2);
\draw (P) -- (O);
\node[below right] (O) {$O$};
\node[below] at (F2) {$H$};
\node[below] at (F1) {$S$};

% Select an arbitrary point Q on the ellipse and draw a line to the Sun
\path [name path=fromF1q] (F1) -- +(30:9);
\path [name intersections = {of = ellipse and fromF1q, by = {Q} }];
\node[above] at (Q) {$Q$};

% Draw QR
\path [name intersections = {of = t and fromF1q, by = {R} }];
\draw (Q) -- (R);

% Draw QX
\path [name path=qx] (Q) -- +({\angle+90}:-{\a} and {-\b});
\path [name intersections = {of = qx and fromF1p, by = {X} }];
\path [name intersections = {of = qx and pc, by = {V} }];
\draw (Q) -- (X) node[above,xshift=0pt,yshift=0] {$X$} -- 
  (V) node[below,xshift=0pt] {$V$};
\vertexsmcolor{X}{red};
\vertexsmcolor{V}{blue};

\end{tikzpicture}
\caption{Geometry of an elliptical orbit (3)}\label{f.elliptical-orbit-3}
\end{center}
\end{figure}

%%%%%%%%%%%%%%%%%%%%%%%%%%%%%%%%%%%%%%%%%%%%%%%%%%%%%%%%%%%%%%%%%%%%%%

\section{Approaching the limit}

Figure~\ref{f.elliptical-orbit-3} is an enlarged diagram of part of Figure~\ref{f.elliptical-orbit-2}. As the time interval $\Delta t$ gets smaller, $Q\rightarrow P$ which implies that
\begin{itemize}
\item $X\rightarrow V$ so that $QX\rightarrow QV$.
\item $V\rightarrow P$ so that $CV\rightarrow CP$ and hence $GV\rightarrow 2CP$.
\item In the limit $QX=QV$ and $GV= 2CP$. Substituting into Equation~\ref{eqn.three-multiplications} gives
\[
\lim_{Q\rightarrow P}\frac{QR}{QT^2}=
%\frac{CP\cdot CA}{CB^2}\cdot \frac{QV^2}{GV\cdot QX^2}
\lim_{Q\rightarrow P}\frac{CP\cdot CA}{CB^2}\cdot \frac{QX^2}{2CP\cdot QX^2}=
\frac{CA}{2CB^2}=
\frac{a}{2b^2}=\frac{1}{L}\,,
\]
\end{itemize}
using the result of Theorem~\ref{thm.ellipse-lr} for the length of the latus rectum.

%%%%%%%%%%%%%%%%%%%%%%%%%%%%%%%%%%%%%%%%%%%%%%%%%%%%%%%%%%%%%%%%%%%%%%
