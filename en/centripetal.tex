% !TeX root = orbits.tex

%%%%%%%%%%%%%%%%%%%%%%%%%%%%%%%%%%%%%%%%%%%%%%%%%%%%%%%%%%%%%%%%

\section{The proof of Theorem~\ref{thm.lr-limit}}
\label{s.centripetal}

Theorem~\ref{thm.lr-limit} is Book~I, Section~III, Proposition~XI, Problem~VI of the \emph{Principia}.

Study Figure~\ref{f.elliptical-orbit-1}:\footnote{To save space the bottom half of the ellipse has been clipped, but we still refer to lines from the center to points on the ellipse as diameters.}
\begin{itemize}
\item  Given an ellipse whose center is $C$ and whose foci are $S$ and $H$, let $P,Q$ be two points on the ellipse. The points represent the movement of a body in an elliptical orbit separated by a time interval $\Delta t$.

\item Construct lines from $P$ to the center and the foci. 

\item Let $RPZ$ be the tangent at $P$ such that the line segment $PR$ is the path that the body would transverse if it continued moving for time $\Delta t$, but was not subject to any force.

\item Construct the parallelogram $PRQX$.

\item Extend $QX$ until it intersects $PC$ at $V$.

\item Construct a line parallel to $RP$ through $H$ and let $I$ be its intersection with $PS$. 

\item Construct $DC$ the conjugate diameter to $PC$(Definition~\ref{def.conjugate}). Let $E$ be its intersection with $PS$.
\end{itemize}

%%%%%%%%%%%%%%%%%%%%%%%%%%%%%%%%%%%%%%%%%%%%%%%%%%%%%%%%%%%%%%%%%%%%%%

\begin{figure}[b]
\begin{center}
\begin{tikzpicture}[scale=1.2]

% Size and center of the ellipse
\def\a{4.33}
\def\b{3.5}

% A bit of padding to center
\path ({-\a},0) -- +(-30pt,0);

% Draw an ellipse with center C
\coordinate (C) at (0,0);
\node[below] at (C) {$C$};
\draw[name path=ellipse] (\a,0) 
  arc[start angle=0,end angle=180, x radius=\a,y radius=\b];

% The Sun is at a focal point
\coordinate (F1) at ({-sqrt(\a*\a-\b*\b)},0);
\coordinate (F2) at ({+sqrt(\a*\a-\b*\b)},0);
\node[below] at (F1) {$S$};
\node[below] at (F2) {$H$};

% Draw axes whose center is C
\coordinate (L) at +(180:{\a} and {\b});
\coordinate (R) at +(0:{\a} and {\b});
\node[below] at (R) {$A$};
\coordinate (Top) at +(90:{\a} and {\b});
\draw[name path=major] (L) -- (R);
\draw[name path=minor] (C) -- (Top);

% Select an arbitrary point P on the ellipse and draw a line to the Sun
\path[name path=fromF1p] (F1) -- +(15:8);
\path [name intersections = {of = ellipse and fromF1p, by = {P} }];
\draw (F1) -- (P);
\draw[name path=ph] (P) -- (F2);
\draw[name path=pc] (P) -- (C);
\node[right] at (P) {$P$};

% Select an arbitrary point Q on the ellipse and draw a line to the Sun
\path[name path=fromF1q] (F1) -- +(40:7);
\path [name intersections = {of = ellipse and fromF1q, by = {Q} }];
\node[above] at (Q) {$Q$};

% Draw tangent at P
\path (F1) -- ($(F1)!1.1!(P)$);
\tkzDefLine[bisector out](F1,P,F2) \tkzGetPoint{Tan}
\draw[name path=t] ($(Tan)!.65!(P)$) 
  coordinate (Z) node[above,xshift=4pt,yshift=-2pt] {$Z$} -- 
  ($(Tan)!1.5!(P)$);

% Draw QR
\path[name path=qr] (Q) -- +(10:2);
\path [name intersections = {of = t and qr, by = {R} }];
\node[above right] at (R) {$R$};
\draw (R) -- (Q);

% Draw QX
\path[name path=qx] (Q) -- +($(Z)-(R)$);
\path [name intersections = {of = qx and fromF1p, by = {X} }];
\path [name intersections = {of = qx and pc, by = {V} }];
\draw (Q) -- (X) node[above right,xshift=-3pt] {$X$} -- 
  (V) node[below] {$V$};

% Conjugate diameter
\path[name path=cd] (C) -- +($(R)-(Z)$);
\path [name intersections = {of = cd and ellipse, by = {D} }];
\node[above left] at (D) {$D$};
\draw (C) -- (D);
\path [name intersections = {of = fromF1p and cd, by = {E} }];
\node[above,xshift=3pt] at (E) {$E$};

% Draw IH
\path[name path=ih] (F2) -- +($(R)-(Z)$);
\path [name intersections = {of = ih and fromF1p, by = {I} }];
\node[above] at (I) {$I$};
\draw (F2) -- (I);

% Label angles
\node[left,xshift=-2pt,yshift=4pt]   at (P)  {$\alpha$};
\node[below,xshift=-2pt,yshift=-4pt] at (P)  {$\alpha$};
\node[right,xshift=2pt,yshift=-2pt]  at (I)  {$\alpha$};
\node[above,xshift=1pt,yshift=3pt]   at (F2) {$\alpha$};

% Label line segments
\path (F1) -- node[below,blue] {$c$} (C) -- node[below,blue] {$c$} (F2);
\path (F1) -- node[above,blue] {$e$} (E) -- node[above,blue] {$e$} (I);
\draw[thick,red] (I) -- node[above,near start] {$d$} (P) -- node[below] {$d$} (F2);

\draw[thick,red] ($(I)+(40:1pt)$) -- ($(F2)+(40:1pt)$);
\draw[thick,blue] (F1) -- (F2) -- (I) -- cycle;
\draw[thick,blue] (E) -- (C);

\end{tikzpicture}
\caption{Geometry of an elliptical orbit (1)}\label{f.elliptical-orbit-1}
\end{center}
\end{figure}

%%%%%%%%%%%%%%%%%%%%%%%%%%%%%%%%%%%%%%%%%%%%%%%%%%%%%%%%%%%%%%%%%%%%%%
\newpage

\subsection{A formula for $QR$}

\begin{theorem}\label{thm.qr}
$QR = PV\cdot \displaystyle\frac{CA}{CP}$.
\end{theorem}

\begin{proof}
By Theorem~\ref{thm.tangent-angles}, $\alpha=\angle RPX = \angle ZPH$ and by alternate interior angles,
\[
\angle PHI = \angle ZPH = \alpha = \angle RPX = \angle PIH\,,
\]
so $\triangle IPH$ (red) is isoceles and $d=PI=PH$.

$SC=CH=c$ are equal because they are the distances of the foci from the center of the ellipse. By construction $EC\parallel IH$ so $\triangle ESC \sim \triangle ISH$ (blue) and 
\begin{eqn}
\frac{SC}{SE}&=&\frac{SH}{SI}\\
\frac{c}{e}&=&\frac{2c}{SI}\,.
\end{eqn}
It follows that $EI=SI-SE=2e-e=e$. By definition of an ellipse, $SP+PH=2CA$, so $e+e+d+d=2CA$ and $EP=e+d=CA$.

$QV \parallel EC$ so  $\triangle EPC \sim \triangle XPV$ and
\begin{eqn}
\frac{PX}{PV} &=& \frac{EP}{PC} =\frac{CA}{PC}\\
PX &=& PV\cdot \frac{CA}{PC}\,.
\end{eqn}
By construction $QR=PX$ proving that $QR= PV\cdot\displaystyle\frac{AC}{PC}$.
\hqed
\end{proof}

%%%%%%%%%%%%%%%%%%%%%%%%%%%%%%%%%%%%%%%%%%%%%%%%%%%%%%%%%%%%%%%%%%%%%%

\begin{figure}[b]
\begin{center}
\begin{tikzpicture}[scale=1.2]

\clip (-5,-1) rectangle +(10.5,5);

% Size and center of the ellipse
\def\a{4.33}
\def\b{3.5}

% A bit of padding to center
\path ({-\a},0) -- +(-30pt,0);

% Draw an ellipse with center C
\coordinate (C) at (0,0);
\node[below left] at (C) {$C$};
\draw[name path=ellipse] (C) ellipse[x radius=\a,y radius= \b];

% The Sun is at a focal point
\coordinate (F1) at ({-sqrt(\a*\a-\b*\b)},0);
\coordinate (F2) at ({+sqrt(\a*\a-\b*\b)},0);
\node[below] at (F1) {$S$};

% Draw axes whose center is C
\coordinate (L) at +(180:{\a} and {\b});
\coordinate (R) at +(0:{\a} and {\b});
\node[right] at (R) {$A$};
\coordinate (Top) at +(90:{\a} and {\b});
\draw[name path=major] (L) -- (R);
\draw[name path=minor] (C) -- (Top);

% Select an arbitrary point P on the ellipse and draw a line to the Sun
\path[name path=fromF1p] (F1) -- +(15:8);
\path [name intersections = {of = ellipse and fromF1p, by = {P} }];
\draw (F1) -- (P);
\path[name path=ph] (P) -- (F2);
\path[name path=pc] (P) -- (C);
\node[right] at (P) {$P$};

% Select an arbitrary point Q on the ellipse and draw a line to the Sun
\path[name path=fromF1q] (F1) -- +(40:7);
\path [name intersections = {of = ellipse and fromF1q, by = {Q} }];
\node[above] at (Q) {$Q$};

% Draw tangent at P
\path (F1) -- ($(F1)!1.1!(P)$);
\tkzDefLine[bisector out](F1,P,F2) \tkzGetPoint{Tan}
\draw[name path=t] ($(Tan)!.65!(P)$) 
  coordinate (Z) node[above,xshift=4pt,yshift=-2pt] {$Z$} -- 
  ($(Tan)!1.5!(P)$);

% Draw QR
\path[name path=qr,red] (Q) -- +(10:2);
\path [name intersections = {of = t and qr, by = {R} }];
\node[above right] at (R) {$R$};
\draw (R) -- (Q);

% Draw QX
\path[name path=qx] (Q) -- +($(Z)-(R)$);
\path [name intersections = {of = qx and fromF1p, by = {X} }];
\path [name intersections = {of = qx and pc, by = {V} }];
%\draw (Q) -- (X) -- (P) -- (R);
\node[above right,xshift=-4pt,yshift=0pt] at (X) {$X$};
%\draw (X) -- (V) node[below] {$V$};

% Draw T
\path (Q) -- ($(F1)!(Q)!(P)$) coordinate (T);
\node[below] at (T) {$T$};
\draw[rotate=13] ($(T)+(0pt,1pt)$) rectangle +(5pt,5pt);

% Conjugate diameter
\path[name path=cd] (C) -- +($(R)-(Z)$);
\path[name path=dk] (C) -- +($(Z)-(R)$);
\path [name intersections = {of = cd and ellipse, by = {D} }];
\path [name intersections = {of = dk and ellipse, by = {K} }];
\node[above left] at (D) {$D$};
\node[below right] at (K) {$K$};
\draw (D) -- (K);
\path [name intersections = {of = fromF1p and cd, by = {E} }];
\node[above,xshift=3pt] at (E) {$E$};

% Draw PF
\draw (P) -- ($(C)!(P)!(K)$) coordinate (F) node[left] {$F$};
\draw[rotate=33] (F) rectangle +(5pt,5pt);

\draw[blue,thick] (E) -- (P) -- (F) -- cycle;
\draw[red,thick] ($(T)+(0pt,1pt)$) -- 
  ($(Q)+(.2pt,-1.2pt)$) -- ($(X)+(-.4pt,1pt)$);
\draw[red,thick] ($(T)+(0pt,1pt)$) -- ($(X)+(-.2pt,1pt)$);

% Label angles
\node[below right,xshift=6pt,yshift=3pt]   at (Q)  {$\beta$};
\node[left,xshift=-4pt,yshift=4pt] at (P)  {$\beta$};
\node[right,xshift=3pt,yshift=-4pt] at (E)  {$\beta$};
\node[left,xshift=-4pt,yshift=4pt] at (X)  {$\beta$};

\end{tikzpicture}
\caption{Geometry of an elliptical orbit (2)}\label{f.elliptical-orbit-2}
\end{center}
\end{figure}

%%%%%%%%%%%%%%%%%%%%%%%%%%%%%%%%%%%%%%%%%%%%%%%%%%%%%%%

\subsection{A formula for $QT$}

Add a perpendicular from $P$ to $DC$ to the diagram and label its intersection with $DC$ by $F$ (Figure~\ref{f.elliptical-orbit-2}).

\begin{theorem}\label{thm.qt}
\[
QT=QX\cdot \frac{FP}{CA}\,.
\]
\end{theorem}

%%%%%%%%%%%%%%%%%%%%%%%%%%%%%%%%%%%%%%%%%%%%%%%%%%%%%%%%%%%%%%%%%%%%%%

\begin{proof}
By construction, $QR \parallel PX$, so by alternate interior angles $\angle RQX =\angle QXT=\beta$. By construction, $QX \parallel DC$, so by alternate interior angles $\angle QXT =\angle PEF=\beta$. Since $\triangle PFE$ and $\triangle QTX$ are right triangles with an equal acute angle $\beta$, $\triangle PFE\sim\triangle QTX$. In the proof of Theorem~\ref{thm.qr} we showed that $EP=CA$ so
\begin{eqn}
\frac{QT}{QX}&=& \frac{FP}{EP}\\
QT&=&QX\cdot \frac{FP}{EP}= \frac{FP}{CA}\,,
\end{eqn}\hqed
\end{proof}

%%%%%%%%%%%%%%%%%%%%%%%%%%%%%%%%%%%%%%%%%%%%%%%%%%%%%%%

\begin{figure}[t]
\begin{center}
\begin{tikzpicture}[scale=1]

% Size and center of the ellipse
\def\a{4.33}
\def\b{3.5}

% A bit of padding to center
\path ({-\a},0) -- +(-20pt,0);

% Draw an ellipse with center C
\coordinate (C) at (0,0);
\node[below left] at (C) {$C$};
\draw[name path=ellipse] (C) ellipse[x radius=\a,y radius= \b];

% The Sun is at a focal point
\coordinate (F1) at ({-sqrt(\a*\a-\b*\b)},0);
\coordinate (F2) at ({+sqrt(\a*\a-\b*\b)},0);
\node[below] at (F1) {$S$};
%\node[below] at (F2) {$H$};

% Draw axes whose center is C
\coordinate (L) at +(180:{\a} and {\b});
\coordinate (R) at +(0:{\a} and {\b});
\node[right] at (R) {$A$};
\coordinate (Top) at +(90:{\a} and {\b});
\draw[name path=major] (L) -- (R);
\draw[name path=minor] (C) -- (Top);

% Select an arbitrary point P on the ellipse and draw a line to the Sun
\path[name path=fromF1p] (F1) -- +(15:8);
\path [name intersections = {of = ellipse and fromF1p, by = {P} }];
\draw (F1) -- (P);
\path[name path=ph] (P) -- (F2);
\path[name path=pc] (P) -- (C);
\node[right] at (P) {$P$};

% Select an arbitrary point Q on the ellipse and draw a line to the Sun
\path[name path=fromF1q] (F1) -- +(40:7);
\path [name intersections = {of = ellipse and fromF1q, by = {Q} }];
\node[above] at (Q) {$Q$};

% Draw tangent at P
\path (F1) -- ($(F1)!1.1!(P)$);
\tkzDefLine[bisector out](F1,P,F2) \tkzGetPoint{Tan}
\draw[name path=t] ($(Tan)!.65!(P)$) 
  coordinate (Z) node[above,xshift=4pt,yshift=-2pt] {$Z$} -- 
  ($(Tan)!1.5!(P)$);

% Draw QR
\path[name path=qr,red] (Q) -- +(10:2);
\path [name intersections = {of = t and qr, by = {R} }];
\node[above right] at (R) {$R$};
\draw (R) -- (Q);

% Draw QX
\path[name path=qx] (Q) -- +($(Z)-(R)$);
\path [name intersections = {of = qx and fromF1p, by = {X} }];
\path [name intersections = {of = qx and pc, by = {V} }];
\node[above right,xshift=-4pt,yshift=0pt] at (X) {$X$};
\draw (X) -- (V) node[below right,xshift=-2pt,yshift=-4pt] {$V$};
\vertexsm{V};

% Draw T
\draw (Q) -- ($(F1)!(Q)!(P)$) coordinate (T);
\node[below] at (T) {$T$};
\draw[rotate=13] (T) rectangle +(5pt,5pt);

% Conjugate diameter
\path[name path=cd] (C) -- +($(R)-(Z)$);
\path[name path=dk] (C) -- +($(Z)-(R)$);
\path [name intersections = {of = cd and ellipse, by = {D} }];
\path [name intersections = {of = dk and ellipse, by = {K} }];
\node[above left] at (D) {$D$};
\node[below right] at (K) {$K$};
\draw (D) -- (K);
\path [name intersections = {of = fromF1p and cd, by = {E} }];
\node[above,xshift=3pt] at (E) {$E$};

% Conjugate diameter
\draw[name path=pc] (P) -- ($(P)!2!(C)$) coordinate (G) node[left] {$G$};

% Draw PF
\draw (P) -- ($(C)!(P)!(K)$) coordinate (F) node[left] {$F$};
\draw[rotate=33] (F) rectangle +(5pt,5pt);

\draw (E) -- (P) -- (F) -- cycle;
\draw (Q) -- (T) -- (X) -- cycle;

\draw[red,thick] (G) -- (V);
\draw[blue,thick] (V) -- (P);
\draw[green!80!black,thick] (Q) -- (V);
\draw[magenta,thick] (D) -- (C);

\end{tikzpicture}
\caption{Ratios on conjugate diameters}\label{f.conj-diam}
\end{center}
\end{figure}

%%%%%%%%%%%%%%%%%%%%%%%%%%%%%%%%%%%%%%%%%%%%%%%%%%%%%%%%%%%%%%%%%%%%%%

\newpage

\subsection{A formula for $QR/QT^2$}

\begin{theorem}
\[
\frac{QR}{QT^2}=\frac{CP\cdot CA}{CB^2}\cdot \frac{QV^2}{GV\cdot QX^2}\,.
\]
\end{theorem}

%%%%%%%%%%%%%%%%%%%%%%%%%%%%%%%%%%%%%%%%%%%%%%%%%%%%%%%%%%%%%%%%%%%%%%

Let use combine the equations in Theorems~\ref{thm.qr} and ~\ref{thm.qt} to get $QR/QT^2$.
\begin{equation}
\frac{QR}{QT^2}=\frac{PV\cdot\displaystyle\frac{CA}{CP}}
{\left(QX\cdot \displaystyle\frac{FP}{CA}\right)^2}=
\frac{PV\cdot CA^3}
{QX^2\cdot CP\cdot FP^2}\,.\label{eqn.qr-qt}
\end{equation}
Theorem~\ref{thm.conj-diag} gives a formula for $PV$ that we substitute into Equation~\ref{eqn.qr-qt}.
\begin{equation}
\frac{QR}{QT^2}=
\frac{QV^2\cdot CP^2}{GV \cdot PV \cdot CD^2}\cdot
\frac{CA^3}
{QX^2\cdot CP\cdot FP^2}=\frac{CP}{CD^2}\cdot \frac{CA^3}{FP^2}\cdot \frac{QV^2}{GV\cdot QX^2}\,.\label{eqn.three-multiplications}
\end{equation}

%%%%%%%%%%%%%%%%%%%%%%%%%%%%%%%%%%%%%%%%%%%%%%%%

\begin{figure}[b]
\begin{center}
\begin{tikzpicture}[scale=.6]

% Size and center of the ellipse
\def\a{4.33}
\def\b{3.5}
\coordinate (C) at (0,0);
\node[below left,xshift=2pt] at (C) {$C$};

% Draw the ellipse
\draw[name path=ellipse] (C) ellipse[x radius=\a,y radius= \b];

% Locate the focal points
\coordinate (F1) at ({-sqrt(\a*\a-\b*\b)},0);
\coordinate (F2) at ({+sqrt(\a*\a-\b*\b)},0);

% Draw axes AA' and BB'
\coordinate (L) at +(180:{\a} and {\b});
\coordinate (R) at +(0:{\a} and {\b});
\coordinate (Top) at +(90:{\a} and {\b});
\coordinate (Bot) at +(-90:{\a} and {\b});
\draw[name path=major] (L)  -- (R) node[right,xshift=-2pt] {$A$};
\draw[name path=minor] (Bot) --
  (Top) node[above,xshift=5pt,yshift=-1pt] {$B$};

% Select an arbitrary point P on the ellipse
\path[name path=fromF1p] (F1) -- +(15:8);
\path [name intersections = {of = ellipse and fromF1p, by = {P} }];
\path[name path=ph] (P) -- (F2);
\path[name path=pc] (P) -- (C);
\node[right,xshift=-2pt,yshift=3pt] at (P) {$P$};

% Draw tangent at P
\tkzDefLine[bisector out](F1,P,F2) \tkzGetPoint{Tan1}
\path[name path=t1] ($(Tan1)!.25!(P)$) -- ($(Tan1)!1.75!(P)$);

% Conjugate diameter from P through C
\draw[name path=pc] (P) -- ($(P)!2!(C)$) coordinate (G);

% Find a point Z on tangent for drawing the tangent
\path[name path=f1q] (F1) -- +(55:8);
\path [name intersections = {of = f1q and t1, by = {Z} }];

% Conjugate diameters from D through C
\path[name path=cd] (C) -- +($(Z)-(P)$);
\path[name path=dk] (C) -- +($(P)-(Z)$);
\path [name intersections = {of = cd and ellipse, by = {D} }];
\path [name intersections = {of = dk and ellipse, by = {K} }];
\node[above left,xshift=1pt,yshift=-5pt] at (D) {$D$};
\draw (D) -- (K);

%% Draw tangent at G (continuation of PC)
\tkzDefLine[bisector out](F2,G,F1) \tkzGetPoint{Tan2}
\path[name path=t2] ($(Tan2)!.2!(G)$) -- ($(Tan2)!1.7!(G)$);

%% Draw tangent at D
\tkzDefLine[bisector out](F1,D,F2) \tkzGetPoint{Tan3}
\path[name path=t3] ($(Tan3)!-1!(D)$) -- ($(Tan3)!2.8!(D)$);

% Draw tangent at K (continuation of DC)
\tkzDefLine[bisector out](F1,K,F2) \tkzGetPoint{Tan4}
\path[name path=t4] ($(Tan4)!-.1!(K)$)  -- ($(Tan4)!2!(K)$);

% Draw large dashed rectangle
\draw[thick,dashed] (\a,\b) -- (-\a,\b) -- (-\a,-\b) -- (\a,-\b) -- cycle;

% Get the intersections of the tangents
\path [name intersections = {of = t1 and t3, by = {J} }];
\path [name intersections = {of = t2 and t3, by = {KK} }];
\path [name intersections = {of = t1 and t4, by = {M} }];
\path [name intersections = {of = t2 and t4, by = {LL} }];

% Draw the parallelogram
\draw (J) -- (KK) -- (LL) -- (M) -- cycle;

% Draw triangles
\draw[thick,blue] (C) -- (D) -- (P) -- cycle;
\draw[thick,red] (C) -- (Top) -- (R) -- cycle;

%% Draw perpendicular to conjugate diameter
\draw[very thick,orange!70] (P) -- ($(D)!(P)!(C)$) coordinate (F);
\node[below left,xshift=5pt,yshift=1pt] at (F) {$F$};
\draw[rotate=33] (F) rectangle +(8pt,8pt);

\end{tikzpicture}
\caption{Parallelograms formed by conjugate diameters}\label{f.conj-diam-para}
\end{center}
\end{figure}

%%%%%%%%%%%%%%%%%%%%%%%%%%%%%%%%%%%%%%%%%%%%%%%%%%%%%%%%%

Next we show that $CD\cdot FP = CA \cdot CB$ (Figure~\ref{f.conj-diam}). Theorem~\ref{thm.conj-diam-para} proved that the areas of the parallelograms formed by the tangents conjugate diameters are equal. By symmetry the areas of the four small parallelograms are equal, as are the triangles formed by constructing corresponding diagonals. In Figure~\ref{f.conj-diam-para} the area of the $\triangle ABC'$ (red), which is $AC\cdot AB$, is equal to the area of $\triangle PCD$ (blue), which is $FP\cdot CD$.

Substituting for $CD\cdot FP$ in Equation~\ref{eqn.three-multiplications} gives
\begin{eqnlabels}
\frac{QR}{QT^2}&=&\frac{CP}{CD^2}\cdot \frac{CA^3}{FP^2}\cdot \frac{QV^2}{GV\cdot QX^2}\nonumber\\
%&=&CP\cdot \frac{CA^3}{CD^2\cdot FP^2}\cdot \frac{QV^2}{GV\cdot QX^2}\\
&=&\frac{CP\cdot CA^3}{CB^2\cdot CA^2}\cdot \frac{QV^2}{GV\cdot QX^2}\nonumber\\
&=&\frac{CP\cdot CA}{CB^2}\cdot \frac{QV^2}{GV\cdot QX^2}\,.\label{eqn.two-labels}
\end{eqnlabels}\hqed

%%%%%%%%%%%%%%%%%%%%%%%%%%%%%%%%%%%%%%%%%%%%%%%%%%%%%%%%%

\begin{figure}
\begin{center}
\begin{tikzpicture}[scale=3]

% Size and center of the ellipse
\def\a{4.33}
\def\b{3.5}

\clip (2,.95) rectangle +(2.5,2);

% Draw an ellipse with center C
\coordinate (C) at (0,0);
\node[below] at (C) {$C$};
%\draw[name path=ellipse] (C) ellipse[x radius=\a,y radius= \b];
\draw[name path=ellipse] (\a,0) 
  arc[start angle=0,end angle=180, x radius=\a,y radius=\b];

% The Sun is at a focal point
\coordinate (F1) at ({-sqrt(\a*\a-\b*\b)},0);
\coordinate (F2) at ({+sqrt(\a*\a-\b*\b)},0);

% Draw axes whose center is C
\coordinate (L) at +(180:{\a} and {\b});
\coordinate (R) at +(0:{\a} and {\b});
\coordinate (Top) at +(90:{\a} and {\b});
\draw[name path=major] (L) -- (R);
\draw[name path=minor] (C) -- (Top);

% Select an arbitrary point P on the ellipse and draw a line to the Sun
\path[name path=fromF1p] (F1) -- +(15:7);
\path [name intersections = {of = ellipse and fromF1p, by = {P} }];
\draw (P) -- ($(P)!.65!(F1)$);
\draw[name path=ph] (P) -- (F2);
\draw[name path=pc,->] (P) -- ($(P)!1.1!(C)$) node[left,yshift=-2pt] {$G$};
\node[right] at (P) {$P$};

% Select an arbitrary point Q on the ellipse and draw a line to the Sun
\path[name path=fromF1q] (F1) -- +(23:7);
\path [name intersections = {of = ellipse and fromF1q, by = {Q} }];
\node[below left,xshift=2pt,yshift=2pt] at (Q) {$Q$};

% Draw tangent at P
\draw (F1) -- ($(F1)!1.1!(P)$);
\tkzDefLine[bisector out](F1,P,F2) \tkzGetPoint{Tan}
\draw[name path=t] ($(Tan)!.75!(P)$) 
  coordinate (Z) -- node[very near end,right] {$t$} ($(Tan)!1.25!(P)$);

% Draw QR
\path[name path=qr] (Q) -- +(10:1);
\path [name intersections = {of = t and qr, by = {R} }];
\draw (R) -- (Q);

% Draw QX
\path[name path=qx] (Q) -- +($(Z)-(R)$);
\path [name intersections = {of = qx and fromF1p, by = {X} }];
\path [name intersections = {of = qx and pc, by = {V} }];
\draw (Q) -- (X) node[left,xshift=-4pt,yshift=4] {$X$} -- 
  (V) node[below,xshift=7pt] {$V$};
\vertexsmcolor{X}{red};
\vertexsmcolor{V}{blue};

% Draw arrows to clipped points
\draw[->] (2.7,1.5) -- +(195:10pt) node[left] {$S$};
\draw[->] (3.1,1.3) -- +(210:10pt) node[left,yshift=-1pt] {$C$};
\draw[->] (3.64,1.38) -- +(235:10pt) node[below] {$H$};

\end{tikzpicture}
\caption{Geometry of an elliptical orbit (3)}\label{f.elliptical-orbit-3}
\end{center}
\end{figure}

%%%%%%%%%%%%%%%%%%%%%%%%%%%%%%%%%%%%%%%%%%%%%%%%%%%%%%%%%%%%%%%%%%%%%%

\subsection{Approaching the limit}

Figure~\ref{f.elliptical-orbit-3} is an enlarged diagram of part of Figure~\ref{f.conj-diam}. Compare the two diagrams and you will see that as the time interval $\Delta t$ gets smaller, $Q\rightarrow P$ which implies that
\begin{itemize}
\item $X\rightarrow V$ so that $QX\rightarrow QV$.
\item $V\rightarrow P$ so that $CV\rightarrow CP$ and hence $GV\rightarrow 2CP$.
\item In the limit $QX=QV$ and $GV= 2CP$. Substituting into Equation~\ref{eqn.two-labels} gives
\begin{eqn}
\frac{QR}{QT^2}&=&\frac{CP\cdot CA}{CB^2}\cdot \frac{QV^2}{GV\cdot QX^2}
=\frac{CP\cdot CA}{CB^2}\cdot \frac{QX^2}{2CP\cdot QX^2}=\frac{CA}{2CB^2}=\frac{a}{2b^2}=\frac{1}{L}\,,
\end{eqn}
\end{itemize}
using the result of Theorem~\ref{thm.ellipse-lr} for the length of the latus rectum.

%%%%%%%%%%%%%%%%%%%%%%%%%%%%%%%%%%%%%%%%%%%%%%%%%%%%%%%%%%%%%%%%%%%%%%
