% !TeX Program=pdfLaTeX

\section{Marden's Theorem}\label{s.marden}

According to Dan Kalman, Marden's Theorem is ``the Most Marvelous
Theorem in Mathematics'' \cite{marden-marvelous}. This section explains his proof of this theorem; it assumes that the reader is familiar with complex numbers.

\subsection*{The statement of the theorem}

\begin{theorem}[Marden]\label{thm.marden}
Let $p(z)$ be a cubic polynomial with complex coefficients whose roots $z_1,z_2,z_3$ are non-collinear in the complex plane. Let $T$ be the triangle whose vertices are $z_1,z_2,z_3$. Then there is a unique ellipse $E$ (called the \emph{Steiner inellipse}) inscribed within $T$ such that the sides of $T$ are tangent to the ellipse at their midpoints. $f_1,f_2$, the foci of the ellipse, are the roots of $p'(z)$ (Figures~\ref{f.marden-axis}, \ref{f.marden-arbitrary}).
\end{theorem}
We omit the relatively simple proof that without loss of generality we can assume that two of the roots are $1$ and $-1$ \cite[p.~332]{marden}, and that the third root $w$ is above the $x$-axis: $w=u+vi,v>0$. This follows since the theorem remains true if the triangle is translated, rotated or scaled.

There are three steps to the proof. First, we show that there is \emph{an} ellipse which is tangent to one side of the triangle at its midpoint. Then we show that this ellipse is tangent to \emph{all} three sides of the triangle, and finally, that the points of contact are the midpoints of the side.

\subsection*{There is an ellipse that is tangent to one side of the triangle}

\begin{theorem}\label{thm.marden1}
Let $p(z),p'(z),T,E,f_1,f_2$ be as defined in the statement of Marden's theorem. Then there is \emph{an} ellipse $E$ which is tangent to one side of $T$ at its midpoint.
\end{theorem}

\paragraph{First example}
Suppose that the third root is on the $y$-axis, say at $i$ (Figure~\ref{f.marden-axis}). Then
\begin{eqn}
p(z) &=& (z+1)(z-1)(z-i) = z^3 -iz^2-z+i)\\
p'(z) &=&3z^2-2iz-1=3\left(z^2-\frac{2}{3}iz-\frac{1}{3}\right)\,.
\end{eqn}%
Using the quadratic formula we can compute the roots of $p'(z)$
$f_{1,2}=(1/3)(i\pm \sqrt{2})$, and it is easy to verify that the following relationship holds between the roots $f_1,f_2$ and the coefficients of $p'(z)$.
\begin{eqn}
f_1+f_2&=&\frac{2}{3}i\\
f_1f_2&=&-\frac{1}{3}\,,
\end{eqn}
$c$, one-half the distance between the foci, is $\sqrt{2}/3$, and $b$, the semi-minor axis, is $1/3$, so $a$, the semi-major axis, is
\[
a = \sqrt{
      \left(\frac{1}{3}\right)^2 + \left(\frac{\sqrt{2}}{3}\right)^2
    } =  \frac{1}{\sqrt{3}}\,.
\]

\begin{figure}
\begin{center}
\begin{tikzpicture}[scale=2.5]
\clip (-2.2,-.25) rectangle +(4.4,1.5);
\draw[line width=0.5pt,gray!50] (-2,0) grid (2,1);
\draw[very thick] (-2,0) -- (-1,0);
\draw[very thick] (1,0) -- (2,0);
\draw[very thick] (0,0) -- (0,1);
\coordinate (z1) at (-1,0);
\coordinate (z2) at (1,0);
\coordinate (z3) at (0,1);
\draw[thick,red] (z1) -- (z2) -- (z3) -- cycle;
\coordinate (f1) at ({-sqrt(2)/3},1/3);
\coordinate (f2) at ({sqrt(2)/3},1/3);
\draw[thick,blue] (f1) -- (f2);
\coordinate (o) at (0,1/3);
\draw[name path global=ellipse] (o)
  ellipse[x radius={1/sqrt(3)},y radius= 1/3];
\vertexsmcolor{f1}{blue};
\vertexsmcolor{f2}{blue};
\node[below] at (z1) {$(-1,0)$};
\node[below] at (z2) {$(1,0)$};
\node[above] at (z3) {$(0,i)$};
\node[below] at (0,0) {$(0,0)$};
\node[above right] at (f1) {$f_1$};
\node[above left] at (f2) {$f_2$};
\end{tikzpicture}
\caption{The third root is on the $y$-axis}\label{f.marden-axis}
\end{center}
\end{figure}

\paragraph{The general case}
The equations that were the developed at essentially the same for any position of the third root $w$ above the $x$-axis.
\begin{eqn}
p(z) &=& (z+1)(z-1)(z-w) = z^3 -wz^2-z+i)\\
p'(z) &=&3z^2-2wz-1=3\left(z^2-\frac{2}{3}wz-\frac{1}{3}\right)\\
f_{1,2}&=&\frac{1}{3}(w\pm \sqrt{w^2+3})\\
f_1+f_2&=&\frac{2}{3}w\\
f_1f_2&=&-\frac{1}{3}\,.
\end{eqn}

\paragraph{Second example}
Let $w=1+2i$ and substitute in the equations for the foci.
\begin{eqn}
f_{1,2}&=&\frac{1}{3}((1+2i)\pm 2\sqrt{i})=
\frac{1}{3}((1\pm\sqrt{2})+(2\pm\sqrt{2})i)\\
f_1+f_2&=&\frac{2}{3}(1+2i)\\
f_1f_2&=&-\frac{1}{3}\,.
\end{eqn}%
where we used the fact that $\sqrt{i}=(1+i)/\sqrt{2}$ (check!).

One-half the distance between the foci is $2/3$ and the semi-minor axis is $\sqrt{2}/3$, so the semi-major axis is $\sqrt{2/3}$. The ellipse is rotated by $45^\circ$ so that its major axis connects the foci (Figure~\ref{f.marden-arbitrary}).

\begin{figure}[b]
\begin{center}
\begin{tikzpicture}[scale=2.5]
\clip (-2.2,-.25) rectangle +(4.4,2.5);
\draw[line width=0.5pt,gray!50] (-2,0) grid (2,2);
\draw[very thick] (-2,0) -- (-1,0);
\draw[very thick] (1,0) -- (2,0);
\draw[very thick] (0,0) -- (0,2);
\coordinate (z1) at (-1,0);
\coordinate (z2) at (1,0);
\coordinate (z3) at (1,2);
\draw[thick,red] (z1) -- (z2) -- (z3) -- cycle;
\coordinate (f1) at ({(1/3)*(1+sqrt(2))},{(1/3)*(2+sqrt(2))});
\coordinate (f2) at ({(1/3)*(1-sqrt(2))},{(1/3)*(2-sqrt(2))});
\draw[thick,blue] (f1) -- (f2);
\coordinate (o) at (1/3,2/3);
\draw[rotate=45,name path global=ellipse] (o)
  ellipse[x radius={sqrt(2/3)},y radius= {sqrt(2)/3}];
\vertexsmcolor{f1}{blue};
\vertexsmcolor{f2}{blue};
\node[below] at (z1) {$(-1,0)$};
\node[below] at (z2) {$(1,0)$};
\node[above] at (z3) {$(1,2i)$};
\node[below] at (0,0) {$(0,0)$};
\node[above left,yshift=-4pt] at (f1) {$f_1$};
\node[above,yshift=4pt] at (f2) {$f_2$};

\end{tikzpicture}
\caption{The third root is not on the $y$-axis}\label{f.marden-arbitrary}
\end{center}
\end{figure}

\begin{proof} (Theorem~\ref{thm.marden1})
Assume for now that $w$ is not on the $y$-axis.

By assumption $w=u+vi, v>0$, so $f_1+f_2=\frac{2}{3}w> 0$ and at least one of $f_1,f_2$ is greater than zero and above the $x$-axis. Since $f_1f_2=-\frac{1}{3}$, the imaginary part of the product is zero. Let $f_1=c_1+d_1i, f_2=c_2+d_2i$; then the imaginary part of the product is $c_1d_2+c_2d_1=0$. Construct the lines from the foci to the origin (Figure~\ref{f.marden-ellipses}). Then,
\begin{eqn}
\frac{d_1}{c_1}+\frac{d_2}{c_2}&=&0\\
\tan \theta_1+\tan \theta_2&=& 
\frac{\sin(\theta_1+\theta_2)}{\cos\theta_1 \cos\theta_2}=0\,,
\end{eqn}%
using the trigonometric identity for the sum of two tangents.

The foci are not on $y$-axis so the cosines are non-zero, and, therefore, $\sin(\theta_1+\theta_2)=0$. $\theta_1+\theta_2$ cannot be $0^\circ$ since $w>0$ and we conclude that $\theta_1+\theta_2=180^\circ$.

%%%%%%%%%%%%%%%%%%%%%%%%%%%%%%%%%%%%%%%%%%%%%%%%%%%%%%%%%%%%%

\begin{figure}
\begin{center}
\begin{tikzpicture}[scale=2.5]
\clip (-2.2,-.25) rectangle +(4.4,2.5);

\draw[line width=0.5pt,gray!50] (-2,0) grid (2,2);
\coordinate (O) at (0,0);
\coordinate (z1) at (-1,0);
\coordinate (z2) at (1,0);
\coordinate (z3) at (1,2);
\draw[thick,red] (z1) -- (z2) -- (z3) -- cycle;
\coordinate (f1) at ({(1/3)*(1+sqrt(2))},{(1/3)*(2+sqrt(2))});
\coordinate (f2) at ({(1/3)*(1-sqrt(2))},{(1/3)*(2-sqrt(2))});
\draw[thick,dotted,blue] (f1) -- (f2);
\coordinate (o) at (1/3,2/3);
\draw[thick,dotted,rotate=45,name path global=ellipse] (o)
  ellipse[x radius={sqrt(2/3)},y radius= {sqrt(2)/3}];

\draw[thick,dashed] (f1) -- (O) -- ($(O)!4!(f2)$);
\draw (.2,0) arc[start angle = 0, end angle = 54.7, radius = .2];
\draw (.35,0) arc[start angle = 0, end angle = {180-54.7}, radius = .35];

\vertexsmcolor{f1}{blue};
\vertexsmcolor{f2}{blue};
\node[below] at (z1) {$(-1,0)$};
\node[below] at (z2) {$(1,0)$};
\node[below] at (0,0) {$(0,0)$};
\node[above] at (z3) {$(1,2i)$};
\node[above left,yshift=-2pt] at (f1) {$f_1$};
\node[above,yshift=10pt] at (f2) {$f_2$};
\node[above right,xshift=10pt,yshift=4pt] at (O) {\sm{\theta_1}};
\node[above left,xshift=-4pt,yshift=1pt] at (O) {\sm{\theta_1}};
\node[above,xshift=12pt,yshift=20pt] at (O) {\sm{\theta_2}};

\end{tikzpicture}
\caption{The ellipse is tangent to the line $OW$}\label{f.marden-ellipses}
\end{center}
\end{figure}

Referring again to the Figure, since $\theta_1, \theta_2$ are supplementary, the angle between the line from $f_2$ and the origin forms an angle $\theta_1$ with the negative $x$-axis. Consider the ellipse with foci $f_1,f_2$ that passes through the origin, which is also the midpoint of one side of the triangle. By Theorem~\ref{thm.tangent-angles}, the ellipse is tangent to this side.

If the third root is on the $y$-axis, the two angles are right angles and the theorem follows immediately.\hqed
\end{proof}

%%%%%%%%%%%%%%%%%%%%%%%%%%%%%%%%%%%%%%%%%%%%%%%%%%%%%%%%%%%%%

\subsection*{The ellipse is tangent to all three sides of the triangle}

We showed that the ellipse whose foci are the roots of $p'(z)$ has a tangent that is the side of the triangle on the $x$-axis and that the point of contact is the midpoint of that side. To prove Marden's theorem, we must show that the other sides of the triangle are tangent at their midpoints to the \emph{same} ellipse. 

\begin{theorem}\label{thm.marden2}
Let $p(z),p'(z),T,f_1,f_2$ be as defined in the statement of Marden's theorem and the ellipse $E$ constructed in Theorem~\ref{thm.marden1}. Then the other two sides of $T$ are tangents to $E$ and their points of contact are the their midpoints.
\end{theorem}

\begin{proof}
By symmetry it is sufficient to prove for one of the other sides $OW$ (Figure~\ref{f.marden-tangent}). In the Figure, the vertices of the triangle on the $x$-axis are now $(0,0),(2,0)$, and the third vertex is at an arbitrary position $w=u+iv,v>0$.

We now compute the foci as before.
\begin{eqn}
p(z)&=&p(p-1)(p-w)=z^3-(1+w)z^2+wz\\
p'(z)&=&3z^2-2(1+w)z+w\\
f_1+f_2&=&\frac{2}{3}(1+w)\\
f_1f_2&=&\frac{w}{3}\,.
\end{eqn}%
From the first equation and $w>0$ we know that at least one focus is above the $x$-axis and since the ellipse is tangent to the $x$-axis the other focus is too.

Let us express $f_1f_2=w/3$ in polar coordinates (notation as in Figure~\ref{f.marden-tangent}).
\begin{eqn}
f_1f_2&=&r_1e^{i\theta_1}\cdot r_2e^{i\theta_2}=r_1r_2e^{i(\theta_1+\theta_2)}=\frac{r}{3}e^{i\theta}\\
\theta_1+\theta_2&=&\theta\\
\theta-\theta_2&=&\theta_1\,.
\end{eqn}%
Therefore, $\angle f_2OV=\theta=\angle f_1OW$.

Now $O=(0,0)$ is \emph{external} to the ellipse because $OV$ is tangent to the ellipse at $(1,0)$. By Theorem~\ref{thm.marden-tangent}, $OW$ is tangent to the ellipse.\hqed
\end{proof}

\subsection*{The proof of Marden's Theorem}

\begin{proof}
We must show that the tangent $OW$ contacts the ellipse at its midpoint $M$. By construction, ellipse $E$ is tangent to the \emph{midpoint} of $OV$ and by Theorem~\ref{thm.marden2} it is also tangent to $OW,VW$. Using the construction of Theorem~\ref{thm.marden1} again, construct an ellipse $E'$ that is tangent to the $OW$ at $M$, which by Theorem~\ref{thm.marden2} is also tangent to the other sides of $T$. But $E,E'$ have the same foci and the same tangents, so they must be the same ellipse.\hqed
\end{proof}

%%%%%%%%%%%%%%%%%%%%%%%%%%%%%%%%%%%%%%%%%%%%%%%%%%%%%%%%%%%%%

\begin{figure}
\begin{center}
\begin{tikzpicture}[scale=2.5]
\clip (-2.2,-.25) rectangle +(4.4,2.5);
\draw[line width=0.5pt,gray!50] (-2,0) grid (2,2);
%\draw[very thick] (-2,0) -- (-1,0);
%\draw[very thick] (1,0) -- (2,0);
%\draw[very thick] (-1,0) -- (-1,2);
\coordinate (z1) at (-1,0);
\coordinate (z2) at (1,0);
\coordinate (z3) at (1,2);
\draw[thick,red] (z1) -- (z2) -- (z3) -- cycle;
\coordinate (f1) at ({(1/3)*(1+sqrt(2))},{(1/3)*(2+sqrt(2))});
\coordinate (f2) at ({(1/3)*(1-sqrt(2))},{(1/3)*(2-sqrt(2))});
\draw[thick,blue] (f1) -- (f2);
\coordinate (o) at (1/3,2/3);
\draw[thick,dotted,rotate=45,name path global=ellipse] (o)
  ellipse[x radius={sqrt(2/3)},y radius= {sqrt(2)/3}];
\vertexsmcolor{f1}{blue};
\vertexsmcolor{f2}{blue};
\node[below] at (z1) {$O=(0,0)$};
\node[below] at (z2) {$V=(2,0)$};
\node[above] at (z3) {$W=w=(2,2)$};
\node[below] at (0,0) {$(1,0)$};
\node[above left,yshift=-4pt] at (f1) {$f_1$};
\node[right,xshift=10pt] at (f2) {$f_2$};
\coordinate (O) at (-1,0);
\draw[thick,dashed] (f1) -- (O) -- (f2);
\draw (-.3,0) arc[start angle = 0, end angle = {32}, radius = .7];
\draw (-.6,0) arc[start angle = 0, end angle = {13}, radius = .4];
\draw (-.05,0) arc[start angle = 0, end angle = {45}, radius = .95];

\node[right,xshift=30pt,yshift=4pt] at (O) {\sm{\theta_1}};
\node[above right,xshift=46pt,yshift=12pt] at (O) {\sm{\theta_2}};
\node[above right,xshift=60pt,yshift=22pt] at (O) {\sm{\theta}};
\end{tikzpicture}
\caption{The ellipse is also tangent to $OW$}\label{f.marden-tangent}
\end{center}
\end{figure}
